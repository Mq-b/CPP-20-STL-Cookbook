
类模板std::function是函数的精简多态包装器,可以存储、复制和调用函数、lambda表达式或其他函数对象。在想要存储对函数或lambda的引用的地方,就很有用。使用std::function允许在同一个容器中存储具有不同签名的函数和lambda,并且其可以维护lambda捕获的上下文。

\subsubsection{How to do it…}

这个示例使用std::function类来存储vector中lambda的不同特化:

\begin{itemize}
\item 
这个示例包含在main()函数中,首先声明三个不同类型的容器:

\begin{lstlisting}[style=styleCXX]
int main() {
	deque<int> d;
	list<int> l;
	vector<int> v;
\end{lstlisting}

这些容器是,deque, list和vector,将由模板lambda引用。

\item 
我们将声明一个名为print\_c的简单lambda函数对容器的值进行打印:

\begin{lstlisting}[style=styleCXX]
auto print_c = [](auto& c) {
	for(auto i : c) cout << format("{} ", i);
	cout << '\n';
};
\end{lstlisting}

\item 
现在声明一个返回匿名lambda的lambda:

\begin{lstlisting}[style=styleCXX]
auto push_c = [](auto& container) {
	return [&container](auto value) {
		container.push_back(value);
	};
};
\end{lstlisting}

push\_c接受对容器的引用,该容器由匿名lambda捕获。匿名lambda调用捕获容器上的push\_back()成员。push\_c的返回值是匿名lambda。

\item 
现在声明一个std::function元素的vector,并用push\_c()的三个实例对其进行填充:

\begin{lstlisting}[style=styleCXX]
const vector<std::function<void(int)>>
	consumers { push_c(d), push_c(l), push_c(v) };
\end{lstlisting}

初始化器列表中的每个元素都是对push\_c的函数调用。push\_c返回匿名lambda的实例,该实例通过函数包装器存储在vector中。push\_c是用d、l和v这三个容器调用的。容器用匿名lambda作为捕获进行传递。

\item 
现在循环遍历consumers vector,并调用每个lambda元素10次,每个容器中用0-9填充三个容器:

\begin{lstlisting}[style=styleCXX]
for(auto &consume : consumers) {
	for (int i{0}; i < 10; ++i) {
		consume(i);
	}
}
\end{lstlisting}

\item 
现在我们的三个容器,deque, list和vector,已经用整数填充。现在将它们打印出来:

\begin{lstlisting}[style=styleCXX]
print_c(d);
print_c(l);
print_c(v);
\end{lstlisting}

输出应是:

\begin{tcblisting}{commandshell={}}
0 1 2 3 4 5 6 7 8 9
0 1 2 3 4 5 6 7 8 9
0 1 2 3 4 5 6 7 8 9
\end{tcblisting}
\end{itemize}

\subsubsection{How it works…}

Lambdas经常间接使用,这就是一个很好的例子。例如,push\_c返回一个匿名lambda:

\begin{lstlisting}[style=styleCXX]
auto push_c = [](auto& container) {
	return [&container](auto value) {
		container.push_back(value);
	};
};
\end{lstlisting}

这个匿名表达式存储在vector中:

\begin{lstlisting}[style=styleCXX]
const vector<std::function<void(int)>>
	consumers { push_c(d), push_c(l), push_c(v) };
\end{lstlisting}

这是consumers容器的定义,由三个元素初始化,其中每个元素都通过调用push\_c进行初始化,该调用返回一个匿名lambda。存储在vector中的是匿名表达式,而不是push\_c。

vector定义使用std::function类作为元素的类型。函数构造函数可以接受任何可调用对象,并将其引用存储为函数目标:

\begin{lstlisting}[style=styleCXX]
template< class F >
function( F&& f );
\end{lstlisting}

当函数使用函数操作符时,函数对象可以使用参数调用目标函数:

\begin{lstlisting}[style=styleCXX]
for(auto &c : consumers) {
	for (int i{0}; i < 10; ++i) {
		c(i);
	}
}
\end{lstlisting}

这将调用存储在consumers容器中的每个匿名表达式10次,从而填充d、l和v容器。

\subsubsection{There's more…}

function类的性质使它在很多方面都很有用,可以将其视为一个多态函数容器。可以存储为一个独立的函数:

\begin{lstlisting}[style=styleCXX]
void hello() {
	cout << "hello\n";
}
int main() {
	function<void(void)> h = hello;
	h();
}
\end{lstlisting}

可以存储一个成员函数,使用std::bind来绑定函数形参:

\begin{lstlisting}[style=styleCXX]
struct hello {
	void greeting() const { cout << "Hello Bob\n"; }
};
int main() {
	hello bob{};
	const function<void(void)> h =
		std::bind(&hello::greeting, &bob);
	h();
}
\end{lstlisting}

或者可以存储可执行对象:

\begin{lstlisting}[style=styleCXX]
struct hello {
	void operator()() const { cout << "Hello Bob\n"; }
};
int main() {
	const function<void(void)> h = hello();
	h();
}
\end{lstlisting}

输出如下所示:

\begin{tcblisting}{commandshell={}}
Hello Bob
\end{tcblisting}












