
当希望从用户或其他输入中选择操作时,跳转表是一种有用的模式,跳转表通常在if/else或switch结构中实现。在这个示例中,我们将只使用STL map和匿名lambda构建一个简洁的跳转表。


\subsubsection{How to do it…}

从map和lambda构建简单的跳转表很容易。map提供了简单的索引导航,lambda可以存储为有效负载。

\begin{itemize}
\item 
首先,创建一个简单的prompt()函数来从控制台获取输入:

\begin{lstlisting}[style=styleCXX]
const char prompt(const char * p) {
	std::string r;
	cout << format("{} > ", p);
	std::getline(cin, r, '\n');
	
	if(r.size() < 1) return '\0';
	if(r.size() > 1) {
		cout << "Response too long\n";
		return '\0';
	}
	return toupper(r[0]);
}
\end{lstlisting}

C-string参数用作提示符,std::getline()被调用来获取用户的输入。响应存储在r中,检查长度。若长度为一个字符,则将其转换为大写并返回。

\item 
在main()函数中,声明并初始化了lambda的map:

\begin{lstlisting}[style=styleCXX]
using jumpfunc = void(*)();
map<const char, jumpfunc> jumpmap {
	{ 'A', []{ cout << "func A\n"; } },
	{ 'B', []{ cout << "func B\n"; } },
	{ 'C', []{ cout << "func C\n"; } },
	{ 'D', []{ cout << "func D\n"; } },
	{ 'X', []{ cout << "Bye!\n"; } }
};
\end{lstlisting}

map容器装载了用于跳转表的匿名lambda。这些lambda可以很容易地调用其他函数或执行简单的任务。

使用别名是为了方便,为lambda有效负载使用函数指针类型void(*)()。若需要更大的灵活性,或者发现它更具可读性,这里可以使用std::function()。其开销很小:

\begin{lstlisting}[style=styleCXX]
using jumpfunc = std::function<void()>;
\end{lstlisting}

\item 
现在可以提示用户输入,并从map中选择一个动作:

\begin{lstlisting}[style=styleCXX]
char select{};
while(select != 'X') {
	if((select = prompt("select A/B/C/D/X"))) {
		auto it = jumpmap.find(select);
		if(it != jumpmap.end()) it->second();
		else cout << "Invalid response\n";
	}
}
\end{lstlisting}

这就是基于map的跳转表,循环直到选择“X”退出。我们使用提示字符串调用prompt(),在map对象上调用find(),然后使用it->second()它调用表达式。
\end{itemize}


\subsubsection{How it works…}

map容器是一个很好的跳转表。简洁明了,易于操作:

\begin{lstlisting}[style=styleCXX]
using jumpfunc = void(*)();
map<const char, jumpfunc> jumpmap {
	{ 'A', []{ cout << "func A\n"; } },
	{ 'B', []{ cout << "func B\n"; } },
	{ 'C', []{ cout << "func C\n"; } },
	{ 'D', []{ cout << "func D\n"; } },
	{ 'X', []{ cout << "Bye!\n"; } }
};
\end{lstlisting}

匿名lambda作为有效负载存储在映射容器中,键是动作菜单中的字符响应。

可以测试一个键的有效性,并在一个动作中选择一个lambda表达式:

\begin{lstlisting}[style=styleCXX]
auto it = jumpmap.find(select);
if(it != jumpmap.end()) it->second();
else cout << "Invalid response\n";
\end{lstlisting}

这是一个简单而优雅的解决方案,否则只能使用尴尬的分支代码了。









