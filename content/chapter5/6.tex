
可以使用一个简单的递归函数来级联lambda,这样一个的输出就是下一个的输入。这创建了一种在另一个函数上构建一个函数的简单方法。

\subsubsection{How to do it…}

这是一个简短而简单的配方,使用一个递归函数来完成大部分工作:

\begin{itemize}
\item 
首先,定义连接函数concat():

\begin{lstlisting}[style=styleCXX]
template <typename T, typename ...Ts>
auto concat(T t, Ts ...ts) {
	if constexpr (sizeof...(ts) > 0) {
		return [&](auto ...parameters) {
			return t(concat(ts...)(parameters...));
		};
	} else {
		return t;
	}
}
\end{lstlisting}

该函数返回一个匿名lambda,该lambda再次调用该函数,直到耗尽参数包。

\item 
main()函数中,我们创建了一对lambda,并使用它们调用concat()函数:

\begin{lstlisting}[style=styleCXX]
int main() {
	auto twice = [](auto i) { return i * 2; };
	auto thrice = [](auto i) { return i * 3; };
	auto combined = concat(thrice, twice,
		std::plus<int>{});
	std::cout << format("{}\n", combined(2, 3));
}
\end{lstlisting}

concat()函数调用时带有三个参数:两个lambdas和std::plus()函数。

随着递归的展开,函数称为从右到左,从plus()开始。plus()函数接受两个参数并返回和。plus()的返回值传递给twice(),它的返回值传递给threice()。

然后,使用format()将结果打印到控制台:

\begin{tcblisting}{commandshell={}}
30
\end{tcblisting}

\end{itemize}

\subsubsection{How it works…}

concat()函数很简单,但由于间接递归和返回lambda,可能会令人困惑:

\begin{lstlisting}[style=styleCXX]
template <typename T, typename ...Ts>
auto concat(T t, Ts ...ts) {
	if constexpr (sizeof...(ts) > 0) {
		return [&](auto ...parameters) {
			return t(concat(ts...)(parameters...));
		};
	} else {
		return t;
	}
}
\end{lstlisting}

concat()函数使用参数包调用。对于省略号,sizeof…运算符返回参数包中的元素数。这用于测试递归的结束。

concat()函数返回一个lambda,递归调用concat()函数。因为concat()的第一个参数不是参数包的一部分,所以每次递归调用都会剥离包的第一个元素。

外层返回语句返回lambda,内部的返回值来自于。lambda调用传递给concat()的函数并返回其值。

读者们可以把它拆开来,细细研究。这项技术还是很有价值的。





