
此示例将lambda包装在函数中,以创建用于算法谓词的自定义连接。

\subsubsection{How to do it…}

copy\_if()算法需要带有一个参数的谓词。在这个示例中,我们将从其他三个lambda创建一个谓词lambda:

\begin{itemize}
\item 
首先,编写combine()函数。这个函数返回一个用于copy\_if()算法的lambda:

\begin{lstlisting}[style=styleCXX]
template <typename F, typename A, typename B>
auto combine(F binary_func, A a, B b) {
	return [=](auto param) {
		return binary_func(a(param), b(param));
	};
}
\end{lstlisting}

combine()函数接受三个函数参数——一个二元连接符和两个谓词——并返回一个lambda,该lambda调用带有两个谓词的连接符。

\item 
main()函数中,创建了用于combine()的lambda:

\begin{lstlisting}[style=styleCXX]
int main() {
	auto begins_with = [](const string &s){
		return s.find("a") == 0;
	};
	auto ends_with = [](const string &s){
		return s.rfind("b") == s.length() - 1;
	};
	auto bool_and = [](const auto& l, const auto& r){
		return l && r;
	};
\end{lstlisting}

begin\_with和ends\_with是简单的过滤器谓词,用于分别查找以'a'开头和以'b'结尾的字符串。bool\_and可对两个参数进行与操作。

\item 
现在,可以使用combine()调用copy\_if算法:

\begin{lstlisting}[style=styleCXX]
std::copy_if(istream_iterator<string>{cin}, {},
			ostream_iterator<string>{cout, " "},
			combine(bool_and, begins_with,
				ends_with));
cout << '\n';
\end{lstlisting}

combine()函数返回一个lambda,该lambda将两个谓词与连接词组合在一起。

输出如下所示:

\begin{tcblisting}{commandshell={}}
$ echo aabb bbaa foo bar abazb | ./conjunction
aabb abazb
\end{tcblisting}
\end{itemize}

\subsubsection{How it works…}

std::copy\_if()算法需要一个带有一个形参的谓词函数,但连接需要两个形参,每个形参都需要一个参数。我们用一个函数来解决这个问题,该函数返回一个专门用于此上下文的lambda:

\begin{lstlisting}[style=styleCXX]
template <typename F, typename A, typename B>
auto combine(F binary_func, A a, B b) {
	return [=](auto param) {
		return binary_func(a(param), b(param));
	};
}
\end{lstlisting}

combine()函数从三个形参创建一个lambda,每个形参都是一个函数。返回的lambda接受谓词函数所需的一个参数。现在可以用combine()函数调用copy\_if():

\begin{lstlisting}[style=styleCXX]
std::copy_if(istream_iterator<string>{cin}, {},
			ostream_iterator<string>{cout, " "},
			combine(bool_and, begins_with, ends_with));
\end{lstlisting}

这将组合lambda传递给算法,以便其在该上下文中进行操作。









