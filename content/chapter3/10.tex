
The set container is an associative container where each element is a single value, which is used as the key. Elements in a set are maintained in sorted order and duplicate keys are not allowed.

The set container is often misunderstood, and it does have fewer and more specific uses than more general containers such as vector and map. One common use for a set is to filter duplicates from a set of values.

\subsubsection{How to do it…}

In this recipe we will read words from the standard input and filter out the duplicates.

\begin{itemize}
\item 
We'll start by defining an alias for an istream iterator. We'll use this to get input from the command line.

\begin{lstlisting}[style=styleCXX]
using input_it = istream_iterator<string>;
\end{lstlisting}

\item 
In the main() function, we'll define a set for our words:

\begin{lstlisting}[style=styleCXX]
int main() {
	set<string> words;
\end{lstlisting}

The set is defined as a set of string elements.

\item 
We define a pair of iterators for use with the inserter() function:

\begin{lstlisting}[style=styleCXX]
input_it it{ cin };
input_it end{};
\end{lstlisting}

The end iterator is initialized with its default constructor. This is known as the end-of-stream iterator. When our input ends, this iterator will compare equal with the cin iterator.

\item 
The inserter() function is used to insert elements into the set container:

\begin{lstlisting}[style=styleCXX]
copy(it, end, inserter(words, words.end()));
\end{lstlisting}

We use std::copy() to conveniently copy words from the input stream.

\item 
Now we can print out our set to see the results:

\begin{lstlisting}[style=styleCXX]
for(const string & w : words) {
	cout << format("{} ", w);
}
cout << '\n';
\end{lstlisting}

\item 
We can run the program by piping a bunch of words to its input:

\begin{tcblisting}{commandshell={}}
$ echo "a a a b c this that this foo foo foo" | ./
set-words
a b c foo that this
\end{tcblisting}
\end{itemize}

The set has eliminated the duplicates and retained a sorted list of the words that were inserted.

\subsubsection{How it works…}

The set container is the heart of this recipe. It only holds unique elements. When you insert a duplicate, that insert will fail. So, you end up with a sorted list of each unique element.

But that's not the only interesting part of this recipe.

The istream\_iterator is an input iterator that reads objects from a stream.
We instantiated the input iterator like this:

\begin{lstlisting}[style=styleCXX]
istream_iterator<string> it{ cin };
\end{lstlisting}

Now we have an input iterator of type string from the cin stream. Every time we dereference this iterator, it will return one word from the input stream.

We also instantiated another istream\_iterator:

\begin{lstlisting}[style=styleCXX]
istream_iterator<string> end{};
\end{lstlisting}

This calls the default constructor, which gives us a special end-of-stream iterator. When the input iterator reaches the end of the stream, it will become equal to the end-of-stream iterator. This is convenient for ending loops, such as the one created by the copy() algorithm.

The copy() algorithm takes three iterators, the beginning and end of the range to copy, and a destination iterator:

\begin{lstlisting}[style=styleCXX]
copy(it, end, inserter(words, words.end()));
\end{lstlisting}

The inserter() function takes a container and an iterator for the insertion point, and returns an insert\_iterator of the appropriate type for the container and its elements.

This combination of copy() and inserter() makes it easy to copy elements from a stream into the set container

