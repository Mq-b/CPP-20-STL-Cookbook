
map是存储键值对的关联容器,容器是按键排序的。键必须是唯一的,并且是const限定的,所以不能更改。

例如,若填充一个map并试图更改键,在编译时会得到一个错误:

\begin{lstlisting}[style=styleCXX]
map<int, string> mymap {
	{1, "foo"}, {2, "bar"}, {3, "baz"}
};
auto it = mymap.begin();
it->first = 47;
\end{lstlisting}

输出为:

\begin{tcblisting}{commandshell={}}
error: assignment of read-only member ...
5 | it->first = 47;
   |  ~~~~~~~~~~^~~~
\end{tcblisting}

若需要重新排序map容器,可以通过使用extract()方法交换键来实现。C++17中,extract()是map类,及其派生类中的成员函数。

它允许从序列中提取map元素,而不涉及有效负载。当提取出来时,键就不再是const限定的,并且可以修改。

\subsubsection{How to do it…}

本例中,我们将定义一个表示比赛中的选手的map。在比赛过程中的某个时刻,顺序发生了变化,需要修改map的键值。

\begin{itemize}
\item 
先从定义map类型的别名开始:

\begin{lstlisting}[style=styleCXX]
using Racermap = map<unsigned int, string>;
\end{lstlisting}

\item 
编写一个函数来打印map:

\begin{lstlisting}[style=styleCXX]
void printm(const Racermap &m)
{
	cout << "Rank:\n";
	for (const auto& [rank, racer] : m) {
		cout << format("{}:{}\n", rank, racer);
	}
}
\end{lstlisting}

可以随时将map传递给这个函数,以打印出参赛者的当前排名。

\item 
main()函数中,定义了一个具有赛车初始状态的map:

\begin{lstlisting}[style=styleCXX]
int main() {
	Racermap racers {
		{1, "Mario"}, {2, "Luigi"}, {3, "Bowser"},
		{4, "Peach"}, {5, "Donkey Kong Jr"}
	};
	printm(racers);
	node_swap(racers, 3, 5);
	printm(racers);
}
\end{lstlisting}

键是int型,表示赛车的级别,值是赛车手名字的字符串。

然后,使用printm()来打印当前排名。node\_swap()将交换两个赛车手的键,然后再次输出。

\item 
在某个时刻,一名选手落后了,而另一名选手则趁机提升了排名。node\_swap()函数将交换两个赛车手的排名:

\begin{lstlisting}[style=styleCXX]
template<typename M, typename K>
bool node_swap(M & m, K k1, K k2) {
	auto node1{ m.extract(k1) };
	auto node2{ m.extract(k2) };
	if(node1.empty() || node2.empty()) {
		return false;
	}
	swap(node1.key(), node2.key());
	m.insert(move(node1));
	m.insert(move(node2));
	return true;
}
\end{lstlisting}

这个函数使用map.extract()方法从映射中提取指定的元素。这些提取出来的元素称为节点。

节点是一个从C++17出现的新概念,可以在不涉及元素本身的情况下从map类型结构中提取元素。解除节点链接,返回节点句柄。提取后,节点句柄通过节点的key()函数提供对键的可写访问。然后,可以交换键并插入到map中,无需复制或操作有效负载。

\item 
当我们运行这段代码时,得到了节点交换前后map的输出:

\begin{tcblisting}{commandshell={}}
Rank:
1:Mario
2:Luigi
3:Bowser
4:Peach
5:Donkey Kong Jr
Rank:
1:Mario
2:Luigi
3:Donkey Kong Jr
4:Peach
5:Bowser
\end{tcblisting}
\end{itemize}

这都是通过extract()方法和新的node\_handle类实现的。让我们仔细了解一下它是如何工作的。

\subsubsection{How it works…}

该技术使用extract()函数,该函数返回一个node\_handle对象,node\_handle是节点的句柄,由关联元素及其相关结构组成。extract函数在将节点保留在原处的同时将其解除关联,并返回一个node\_handle对象。这样做的效果是从关联容器中删除节点,而不涉及数据本身。node\_handle允许访问已解除关联的节点。

node\_handle有一个成员函数key(),返回一个对节点键的可写引用。这就可以在键与容器解关联时,对键值进行修改。

\subsubsection{There's more…}

使用extract()和node\_handle时,有几个重点:

\begin{itemize}
\item 
若没有找到键,extract()函数返回一个空节点句柄。可以用empty()函数测试节点句柄是否为空:

\begin{lstlisting}[style=styleCXX]
auto node{ mapthing.extract(key) };
if(node.empty()) {
	// node handle is empty
}
\end{lstlisting}

\item 
extract()有两个重载:

\begin{lstlisting}[style=styleCXX]
node_type extract(const key_type& x);
node_type extract(const_iterator position);
\end{lstlisting}

我们使用了第一种形式,通过传递键,使用迭代器。

\item 
不能使用字面量引用,因此像extract(1)这样的调用会因段错误而使程序崩溃。

\item 
键插入映射时必须保持唯一。

例如,若将一个键更改为映射中已经存在的值:

\begin{lstlisting}[style=styleCXX]
auto node_x{ racers.extract(racers.begin()) };
node_x.key() = 5; // 5 is Donkey Kong Jr
auto status = racers.insert(move(node_x));
if(!status.inserted) {
	cout << format("insert failed, dup key: {}",
		status.position->second);
	exit(1);
}
\end{lstlisting}

插入失败,会得到错误信息:

\begin{tcblisting}{commandshell={}}
insert failed, dup key: Donkey Kong Jr
\end{tcblisting}

本例中,将begin()迭代器传递给extract()。然后,为键值分配了一个已经在使用的值(5, Donkey Kong Jr)。插入失败和结果status.inserted为false。status.position是指向已找到键的迭代器。在if()中,我使用format()来打印找到的键值。
\end{itemize}







