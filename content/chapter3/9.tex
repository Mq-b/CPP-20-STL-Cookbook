
对于有序map,键的类型必须是可排序的,必须至少支持小于<比较操作符。假设希望使用具有不可排序的自定义类型的关联容器。例如,一个向量(0,1)既不大于也不小于(1,0),只是指向不同的方向。在这种情况下,可以使用unordered\_map类型。

\subsubsection{How to do it…}

对于这个示例,我们将创建一个unordered\_map对象,该对象使用x/y坐标作为键。为此,并且需要一些功能支持。

\begin{itemize}
\item 
首先,将为坐标定义一个结构体:

\begin{lstlisting}[style=styleCXX]
struct Coord {
	int x{};
	int y{};
};
\end{lstlisting}

这是一个简单的结构,有两个元素,x和y,作为坐标。

\item 
map将使用Coord结构作为键,并使用int作为值:

\begin{lstlisting}[style=styleCXX]
using Coordmap = unordered_map<Coord, int>;
\end{lstlisting}

为了方便使用map,我们使用了using别名。

\item 
要使用Coord结构体作为键,需要几个重载。这些是使用unordered\_map所必需的。首先,定义一个相等比较运算符:

\begin{lstlisting}[style=styleCXX]
bool operator==(const Coord& lhs, const Coord& rhs) {
	return lhs.x == rhs.x && lhs.y == rhs.y;
}
\end{lstlisting}

这是一个很简单的函数,用来比较x元素和y元素。

\item 
还需要一个std::hash类特化,这就可以使用键检索map元素:

\begin{lstlisting}[style=styleCXX]
namespace std {
	template<>
	struct hash<Coord> {
		size_t operator()(const Coord& c) const {
			return static_cast<size_t>(c.x)
			+ static_cast<size_t>(c.y);
		}
	};
}
\end{lstlisting}

这为std::unordered\_map类使用的默认哈希类提供了特化,必须在std命名空间中。

\item 
还将编写一个print函数来打印Coordmap对象:

\begin{lstlisting}[style=styleCXX]
void print_Coordmap(const Coordmap& m) {
	for (const auto& [key, value] : m) {
		cout << format("{{ ({}, {}): {} }} ",
		key.x, key.y, value);
	}
	cout << '\n';
}
\end{lstlisting}

这使用C++ 20 format()函数来打印x/y键和值。注意使用双大括号\{\{和\}\}来打印单个大括号。

\item 
现在有了所有的支持函数,就可以编写main()函数了。

\begin{lstlisting}[style=styleCXX]
int main() {
	Coordmap m {
		{ {0, 0}, 1 },
		{ {0, 1}, 2 },
		{ {2, 1}, 3 }
	};
	print_Coordmap(m);
}
\end{lstlisting}

输出:

\begin{tcblisting}{commandshell={}}
{ (2, 1): 3 } { (0, 1): 2 } { (0, 0): 1 }
\end{tcblisting}

此时,已经定义了一个Coordmap对象,可接受键的Coord对象,并将它们映射到值。

\item 
也可以基于Coord键访问单个成员:

\begin{lstlisting}[style=styleCXX]
Coord k{ 0, 1 };
cout << format("{{ ({}, {}): {} }}\n", k.x, k.y,
m.at(k));
\end{lstlisting}

输出:

\begin{tcblisting}{commandshell={}}
{ (0, 1): 2 }
\end{tcblisting}

这里,定义了一个名为k的Coord对象,并使用at()函数从unordered\_map中检索值。

\end{itemize}

\subsubsection{How it works…}

unordered\_map类依赖于哈希类从键中查找元素,通常会这样实例化对象:

\begin{lstlisting}[style=styleCXX]
std::unordered_map<key_type, value_type> my_map;
\end{lstlisting}

因为我们没有创建哈希类,所以其使用了默认的哈希类。unordered\_map类的完整模板类型定义如下:

\begin{lstlisting}[style=styleCXX]
template<
	class Key,
	class T,
	class Hash = std::hash<Key>,
	class KeyEqual = std::equal_to<Key>,
	class Allocator = std::allocator< std::pair<const Key,
		T> >
> class unordered_map;
\end{lstlisting}

该模板为Hash、KeyEqual和Allocator提供了默认值,因此通常不会在定义中包含它们。在示例中,为默认std::hash类提供了特化。

STL包含std::hash对大多数标准类型的特化,如string、int等。为了与我们的类一起工作,其需要特化。

可以向模板形参传递一个函数:

\begin{lstlisting}[style=styleCXX]
std::unordered_map<coord, value_type, my_hash_type> my_map;
\end{lstlisting}

这会有用。但在我看来,特化会更通用。





