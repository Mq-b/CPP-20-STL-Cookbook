
这是一个简单的程序,用于计算目录及其子目录中每个文件的大小,可以在POSIX/Unix和Windows文件系统上运行。

\subsubsection{How to do it…}

这个示例是一个程序,用于报告目录及其子目录中每个文件的大小,以及总数。我们将重用在本章其他地方使用过的一些函数:

\begin{itemize}
\item 
先从几个别名开始说起:

\begin{lstlisting}[style=styleCXX]
namespace fs = std::filesystem;
using dit = fs::directory_iterator;
using de = fs::directory_entry;
\end{lstlisting}

\item 
还对fs::path对象使用格式化特化:

\begin{lstlisting}[style=styleCXX]
template<>
struct std::formatter<fs::path>:
std::formatter<std::string> {
	template<typename FormatContext>
	auto format(const fs::path& p, FormatContext& ctx) {
		return format_to(ctx.out(), "{}", p.string());
	}
};
\end{lstlisting}

\item 
为了输出目录的大小,将使用make\_comma()函数:

\begin{lstlisting}[style=styleCXX]
string make_commas(const uintmax_t& num) {
	string s{ std::to_string(num) };
	for(long l = s.length() - 3; l > 0; l -= 3) {
		s.insert(l, ",");
	}
	return s;
}
\end{lstlisting}

我们以前用过这个,其从结尾开始,每三个字符前插入一个逗号。

\item 
要对目录进行排序,需要一个小写字符串的函数:

\begin{lstlisting}[style=styleCXX]
string strlower(string s) {
	auto char_lower = [](const char& c) -> char {
		if(c >= 'A' && c <= 'Z') return c + ('a' –
			'A');
		else return c;
	};
	std::transform(s.begin(), s.end(), s.begin(),
		char_lower);
	return s;
}
\end{lstlisting}

\item 
需要一个比较谓词,根据路径名的小写字母对directory\_entry对象进行排序:

\begin{lstlisting}[style=styleCXX]
bool dircmp_lc(const de& lhs, const de& rhs) {
	const auto lhstr{ lhs.path().string() };
	const auto rhstr{ rhs.path().string() };
	return strlower(lhstr) < strlower(rhstr);
}
\end{lstlisting}

\item 
size\_string()返回用于报告文件大小的缩写值,单位为千兆字节、兆字节、千字节或字节:

\begin{lstlisting}[style=styleCXX]
string size_string(const uintmax_t fsize) {
	constexpr const uintmax_t kilo{ 1024 };
	constexpr const uintmax_t mega{ kilo * kilo };
	constexpr const uintmax_t giga{ mega * kilo };
	
	if(fsize >= giga ) return format("{}{}",
		(fsize + giga / 2) / giga, 'G');
	else if (fsize >= mega) return format("{}{}",
		(fsize + mega / 2) / mega, 'M');
	else if (fsize >= kilo) return format("{}{}",
		(fsize + kilo / 2) / kilo, 'K');
	else return format("{}B", fsize);
}
\end{lstlisting}

\item 
entry\_size()返回文件的大小,若是目录,则返回目录的递归大小:

\begin{lstlisting}[style=styleCXX]
uintmax_t entry_size(const fs::path& p) {
	if(fs::is_regular_file(p)) return
		fs::file_size(p);
	uintmax_t accum{};
	if(fs::is_directory(p) && ! fs::is_symlink(p)) {
		for(auto& e : dit{ p }) {
			accum += entry_size(e.path());
		}
	}
	return accum;
}
\end{lstlisting}

\item 
main()中,从声明开始,并测试是否有一个有效的目录要搜索:

\begin{lstlisting}[style=styleCXX]
int main(const int argc, const char** argv) {
	auto dir{ argc > 1 ?
		fs::path(argv[1]) : fs::current_path() };
	vector<de> entries{};
	uintmax_t accum{};
	if (!exists(dir)) {
		cout << format("path {} does not exist\n",
		dir);
		return 1;
	}
	if(!is_directory(dir)) {
		cout << format("{} is not a directory\n",
		dir);
		return 1;
	}
	cout << format("{}:\n", absolute(dir));
\end{lstlisting}

对于目录路径dir,若有参数,则使用argv[1]。否则,对当前目录使用current\_path()。然后,为用法计数器设置环境:

\begin{itemize}
\item 
directory\_entry对象的vector用于排序。

\item 
accum用于累积文件最终总大小的值。

\item 
继续检查目录之前,需要确保dir存在,并且的确是一个目录。
\end{itemize}

\item 
接下来,一个简单的循环填充vector。填充完成后,使用dircmp\_lc()函数作为比较谓词对条目进行排序:

\begin{lstlisting}[style=styleCXX]
for (const auto& e : dit{ dir }) {
	entries.emplace_back(e.path());
}
std::sort(entries.begin(), entries.end(), dircmp_lc);
\end{lstlisting}

\item 
现在一切都设置好了,可以使用排序vector中,directory\_entry对象累积的结果了:

\begin{lstlisting}[style=styleCXX]
for (const auto& e : entries) {
	fs::path p{ e };
	uintmax_t esize{ entry_size(p) };
	string dir_flag{};
	accum += esize;
	if(is_directory(p) && !is_symlink(p)) dir_flag =
	" *";
	cout << format("{:>5} {}{}\n",
	size_string(esize), p.filename(), dir_flag);
}
cout << format("{:->25}\n", "");
cout << format("total bytes: {} ({})\n",
	make_commas(accum), size_string(accum));
\end{lstlisting}

对entry\_size()的调用返回directory\_entry对象中表示的文件或目录的大小。

若当前entry是一个目录(而不是一个符号链接),就添加一个符号来表示它是一个目录。我选择了一个倒三角,你可以选择你喜欢的符号。[译者注:由于倒三角不好显示,这里选择了使用*表示]

循环完成后,用逗号显示两个字节的累积大小,以及size\_string()的缩写符号。

输出为:

\begin{tcblisting}{commandshell={}}
/home/billw/working/cpp-stl-wkbk/chap10:
327K bwgrep
  3K bwgrep.cpp
199K dir
  4K dir.cpp
176K formatter
905B formatter.cpp
  0B include
  1K Makefile
181K path-ops
  1K path-ops.cpp
327K rerename
  2K rerename.cpp
 11K testdir *
 11K testdir-backup *
203K working
  3K working.cpp
-------------------------
total bytes: 1,484,398 (1M)
\end{tcblisting}

\end{itemize}

\subsubsection{How it works…}

fs::file\_size()函数的作用是:返回一个uintmax\_t值,该值表示文件的大小,为给定平台上最大的自然无符号整数。虽然,在大多数64位系统上这通常是一个64位整数,但Windows是一个例外,其使用32位整数。虽然,size\_t可能在某些系统上适用于这个值,但在Windows上无法编译,因为这里可能试图将64位值提升为32位值。

entry\_size()函数接受一个路径对象,并返回uintmax\_t值:

\begin{lstlisting}[style=styleCXX]
uintmax_t entry_size(const fs::path& p) {
	if(fs::is_regular_file(p)) return fs::file_size(p);
	uintmax_t accum{};
	if(fs::is_directory(p) && !fs::is_symlink(p)) {
		for(auto& e : dit{ p }) {
			accum += entry_size(e.path());
		}
	}
	return accum;
}
\end{lstlisting}

该函数检查常规文件并返回文件的大小。否则,将检查一个目录是否也是符号链接。我们只想知道目录中文件的大小,因此不希望有符号链接。(符号链接可能会出现引用循环)

若找到一个目录,就循环遍历,为遇到的每个文件使用entry\_size()。这是一个递归循环,所以我们最终会得到整个目录的大小。


























