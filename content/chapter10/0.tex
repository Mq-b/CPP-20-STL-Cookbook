STL文件系统库的目的是标准化跨平台的文件系统操作。文件系统库寻求规范化操作,而非如同POSIX/Unix、Windows和其他文件系统之间那样的不标准桥梁。

文件系统库采用了相应的Boost库,并与C++17集成到STL中。撰写本文时,其在一些系统上的实现仍然存在差距,但本章中的实力已经在Linux、Windows和macOS文件系统上进行了测试,并分别使用GCC、MSVC和Clang编译器的最新可用版本进行了编译。

标准库使用<filesystem>头文件,std::filesystem命名空间通常别名为fs:

\begin{lstlisting}[style=styleCXX]
namespace fs = std::filesystem;
\end{lstlisting}

fs::path类是文件系统库的核心,在不同的环境中提供了规范化的文件名和目录路径表示。路径对象可以表示文件、目录或对象中的对象,甚至是不存在或不可能的对象。


在下面的主题中,我们将介绍使用文件系统库处理文件和目录的工具:

\begin{itemize}
\item 
为路径类特化std::formatter

\item 
使用带有路径的操作函数

\item 
列出目录中的文件

\item 
使用grep实用程序搜索目录和文件

\item 
使用regex和directory\_iterator重命名文件

\item 
创建磁盘使用计数器
\end{itemize}











