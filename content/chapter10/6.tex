

这是一个使用正则表达式重命名文件的简单实用程序,可使用directory\_iterator查找目录中的文件,并使用fs::rename()重命名。

\subsubsection{How to do it…}

在这个示例中,我们创建了一个使用正则表达式的文件重命名实用程序:

\begin{itemize}
\item 
先从定义一些别名开始:

\begin{lstlisting}[style=styleCXX]
namespace fs = std::filesystem;
using dit = fs::directory_iterator;
using pat_v = vector<std::pair<regex, string>>;
\end{lstlisting}

别名pat\_v是一个用于正则表达式的vector。

\item 
我们还继续为path对象使用格式化特化:

\begin{lstlisting}[style=styleCXX]
template<>
struct std::formatter<fs::path>:
std::formatter<std::string> {
	template<typename FormatContext>
	auto format(const fs::path& p, FormatContext& ctx) {
		return format_to(ctx.out(), "{}", p.string());
	}
};
\end{lstlisting}

\item 
我们有一个函数用于对文件名字符串应用正则表达式替换:

\begin{lstlisting}[style=styleCXX]
string replace_str(string s, const pat_v& replacements) {
	for(const auto& [pattern, repl] : replacements) {
		s = regex_replace(s, pattern, repl);
	}
	return s;
}
\end{lstlisting}

注意,我们循环遍历模式/替换对的vector,依次应用正则表达式。这使得我们可以堆叠替换。

\item 
main()中,首先检查命令行参数:

\begin{lstlisting}[style=styleCXX]
int main(const int argc, const char** argv) {
	pat_v patterns{};
	if(argc < 3 || argc % 2 != 1) {
		fs::path cmdname{ fs::path{argv[0]}.filename() };
		cout << format(
			"usage: {} [regex replacement] ...\n",
			cmdname);
		return 1;
	}
\end{lstlisting}

命令行接受一个或多个字符串对。每对字符串都包含一个正则表达式(正则表达式),后面跟着一个替换。

\item 
现在,用regex和string对象填充vector:

\begin{lstlisting}[style=styleCXX]
for(int i{ 1 }; i < argc; i += 2) {
	patterns.emplace_back(argv[i], argv[i + 1]);
}
\end{lstlisting}

pair构造函数根据在命令行上传递的C-string构造适当的regex和string对象,其通过emplace\_back()方法添加到vector中。

\item 
使用directory\_iterator对象搜索当前目录:

\begin{lstlisting}[style=styleCXX]
for(const auto& entry : dit{fs::current_path()}) {
	fs::path fpath{ entry.path() };
	string rname{
		replace_str(fpath.filename().string(),
		patterns) };
	if(fpath.filename().string() != rname) {
		fs::path rpath{ fpath };
		rpath.replace_filename(rname);
		if(exists(rpath)) {
			cout << "Error: cannot rename - destination
			file exists.\n";
		} else {
			fs::rename(fpath, rpath);
			cout << format(
			"{} -> {}\n",
			fpath.filename(),
			rpath.filename());
		}
	}
}
\end{lstlisting}

这个for循环中,调用replace\_str()来获取替换的文件名,然后检查新名称是不是目录中某个文件的副本。在path对象上使用replace\_filename()方法创建具有新文件名的路径,并使用fs::rename()重命名文件。

\item 
为了测试这个工具,我创建了一个目录,里面有几个文件用于重命名:

\begin{tcblisting}{commandshell={}}
$ ls
bwfoo.txt bwgrep.cpp chrono.cpp dir.cpp formatter.cpp
path-ops.cpp working.cpp
\end{tcblisting}

\item 
可以做一些简单的事情,比如把.cpp改成.Cpp:

\begin{tcblisting}{commandshell={}}
$ ../rerename .cpp .Cpp
dir.cpp -> dir.Cpp
path-ops.cpp -> path-ops.Cpp
bwgrep.cpp -> bwgrep.Cpp
working.cpp -> working.Cpp
formatter.cpp -> formatter.Cpp
\end{tcblisting}

再把它们改回来:

\begin{tcblisting}{commandshell={}}
$ ../rerename .Cpp .cpp
formatter.Cpp -> formatter.cpp
bwgrep.Cpp -> bwgrep.cpp
dir.Cpp -> dir.cpp
working.Cpp -> working.cpp
path-ops.Cpp -> path-ops.cpp
\end{tcblisting}

\item 
使用标准的正则表达式语法,我可以在每个文件名的开头添加“bw”:

\begin{tcblisting}{commandshell={}}
$ ../rerename '^' bw
bwgrep.cpp -> bwbwgrep.cpp
chrono.cpp -> bwchrono.cpp
formatter.cpp -> bwformatter.cpp
bwfoo.txt -> bwbwfoo.txt
working.cpp -> bwworking.cpp
\end{tcblisting}

注意,它重命名了开头已经有“bw”的文件。我们不想让它这样做。所以,先恢复文件名:

\begin{tcblisting}{commandshell={}}
$ ../rerename '^bw' ''
bwbwgrep.cpp -> bwgrep.cpp
bwworking.cpp -> working.cpp
bwformatter.cpp -> formatter.cpp
bwchrono.cpp -> chrono.cpp
bwbwfoo.txt -> bwfoo.txt
\end{tcblisting}

现在,使用一个正则表达式来检查文件名是否已经以"bw"开头:

\begin{tcblisting}{commandshell={}}
$ ../rerename '^(?!bw)' bw
chrono.cpp -> bwchrono.cpp
formatter.cpp -> bwformatter.cpp
working.cpp -> bwworking.cpp
\end{tcblisting}

因为,使用了一个regex/replacement字符串的vector,所以可以堆叠几个替换:

\begin{tcblisting}{commandshell={}}
$ ../rerename foo bar '\.cpp$' '.xpp' grep grok
bwgrep.cpp -> bwgrok.xpp
bwworking.cpp -> bwworking.xpp
bwformatter.cpp -> bwformatter.xpp
bwchrono.cpp -> bwchrono.xpp
bwfoo.txt -> bwbar.txt
\end{tcblisting}

\end{itemize}

\subsubsection{How it works…}

这个示例的文件系统部分使用directory\_iterator为当前目录中的每个文件,返回一个directory\_entry对象:

\begin{lstlisting}[style=styleCXX]
for(const auto& entry : dit{fs::current_path()}) {
	fs::path fpath{ entry.path() };
	...
}
\end{lstlisting}

然后,使用directory\_entry对象构造一个path对象来处理文件。

我们在path对象上使用replace\_filename()方法来创建重命名操作的目标:

\begin{lstlisting}[style=styleCXX]
fs::path rpath{ fpath };
rpath.replace_filename(rname);
\end{lstlisting}

这里,创建一个副本并更改它的名称:

\begin{lstlisting}[style=styleCXX]
fs::rename(fpath, rpath);
\end{lstlisting}

示例的正则表达式部分,regex\_replace()使用正则表达式语法在字符串中执行替换:

\begin{lstlisting}[style=styleCXX]
s = regex_replace(s, pattern, repl);
\end{lstlisting}

正则表达式语法非常强大,允许替换包含搜索字符串的部分:

\begin{tcblisting}{commandshell={}}
$ ../rerename '(bw)(.*\.)(.*)$' '$3$2$1'
bwgrep.cpp -> cppgrep.bw
bwfoo.txt -> txtfoo.bw
\end{tcblisting}

通过在搜索模式中使用括号,可以轻松地重新排列文件名的各个部分。



