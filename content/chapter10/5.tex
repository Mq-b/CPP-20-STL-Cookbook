
为了演示遍历和搜索目录结构,创建了一个工作方式类似Unix grep的简单程序。这个程序使用recursive\_directory\_iterator遍历嵌套目录,并使用正则表达式搜索文件以查找匹配。

\subsubsection{How to do it…}

这个示例中,我们编写了一个简单的grep程序,可以遍历目录以使用正则表达式搜索文件:

\begin{itemize}
\item 
先从一些别名开始:

\begin{lstlisting}[style=styleCXX]
namespace fs = std::filesystem;
using de = fs::directory_entry;
using rdit = fs::recursive_directory_iterator;
using match_v = vector<std::pair<size_t, std::string>>;
\end{lstlisting}

match\_v是一个正则表达式匹配结果的vector。

\item 
继续使用path对象的格式化特化:

\begin{lstlisting}[style=styleCXX]
template<>
struct std::formatter<fs::path>:
std::formatter<std::string> {
	template<typename FormatContext>
	auto format(const fs::path& p, FormatContext& ctx) {
		return format_to(ctx.out(), "{}", p.string());
	}
};
\end{lstlisting}

\item 
我们有一个简单的函数,用于从文件中获取正则表达式匹配:

\begin{lstlisting}[style=styleCXX]
match_v matches(const fs::path& fpath, const regex& re) {
	match_v matches{};
	std::ifstream instrm(fpath.string(),
		std::ios_base::in);
	string s;
	for(size_t lineno{1}; getline(instrm, s); ++lineno) {
		if(std::regex_search(s.begin(), s.end(), re)) {
			matches.emplace_back(lineno, move(s));
		}
	}
	return matches;
}
\end{lstlisting}

在这个函数中,使用ifstream打开文件,使用getline()从文件中读取行,并使用regex\_search()匹配正则表达式。在vector中收集结果并返回。

\item 
现在可以在main()中使用这个函数:

\begin{lstlisting}[style=styleCXX]
int main() {
	constexpr const char * fn{ "working.cpp" };
	constexpr const char * pattern{ "path" };
	
	fs::path fpath{ fn };
	regex re{ pattern };
	auto regmatches{ matches(fpath, re) };
	for(const auto& [lineno, line] : regmatches) {
		cout << format("{}: {}\n", lineno, line);
	}
	cout << format("found {} matches\n", regmatches.
	size());
}
\end{lstlisting}

在本例中,使用常量作为文件名和正则表达式模式。创建path和regex对象,调用matches()函数,并打印结果。

输出行号和匹配行的字符串:

\begin{tcblisting}{commandshell={}}
25: struct std::formatter<fs::path>:
std::formatter<std::string> {
27: auto format(const fs::path& p, FormatContext& ctx) {
32: match_v matches(const fs::path& fpath, const regex& re) {
34: std::ifstream instrm(fpath.string(), std::ios_base::in);
62: constexpr const char * pattern{ "path" };
64: fs::path fpath{ fn };
66: auto regmatches{ matches(fpath, re) };
\end{tcblisting}

\item 
程序需要接受regex模式和文件名的命令行参数,其能够遍历目录或获取文件名列表(这可能是命令行通配符展开的结果)。这需要main()函数中的一些逻辑处理。

首先,需要一个辅助函数:

\begin{lstlisting}[style=styleCXX]
size_t pmatches(const regex& re, const fs::path& epath,
const fs::path& search_path) {
	fs::path target{epath};
	auto regmatches{ matches(epath, re) };
	auto matchcount{ regmatches.size() };
	if(!matchcount) return 0;
	
	if(!(search_path == epath)) {
		target =
		epath.lexically_relative(search_path);
	}
	for (const auto& [lineno, line] : regmatches) {
		cout << format("{} {}: {}\n", target, lineno,
		line);
	}
	return regmatches.size();
}
\end{lstlisting}

这个函数调用matches()函数并输出结果,其接受一个regex对象和两个path对象。epath是目录搜索的结果,而search\_path是搜索目录本身。我们将在main()中设置这些参数。

\item 
在main()中,使用argc和argv命令行参数,并声明了几个变量:

\begin{lstlisting}[style=styleCXX]
int main(const int argc, const char** argv) {
	const char * arg_pat{};
	regex re{};
	fs::path search_path{};
	size_t matchcount{};
	...
\end{lstlisting}

这里需要声明的变量是:

\begin{itemize}
\item
arg\_pat用于命令行中的正则表达式模式

\item
re是正则表达式对象

\item
search\_path命令行搜索路径是参数

\item
matchcount是用来计数匹配的行
\end{itemize}

\item 
若没有参数,则打印一个简短的用法字符串:

\begin{lstlisting}[style=styleCXX]
if(argc < 2) {
	auto cmdname{ fs::path(argv[0]).filename() };
	cout << format("usage: {} pattern [path/file]\n",
		cmdname);
	return 1;
}
\end{lstlisting}

argv[1]始终是来自命令行的调用命令。cmdname使用filename()方法,会返回只包含调用命令路径的文件名。

\item 
接下来,解析正则表达式。使用try-catch块来捕获来自正则表达式解析器的错误:

\begin{lstlisting}[style=styleCXX]
arg_pat = argv[1];
try {
	re = regex(arg_pat, std::regex_constants::icase);
} catch(const std::regex_error& e) {
	cout << format("{}: {}\n", e.what(), arg_pat);
	return 1;
}
\end{lstlisting}

使用icase标志告诉正则表达式解析器忽略大小写。

\item 
若argc == 2,只有一个参数,将其视为正则表达式模式,并且使用当前目录作为搜索路径:

\begin{lstlisting}[style=styleCXX]
if(argc == 2) {
	search_path = ".";
	for (const auto& entry : rdit{ search_path }) {
		const auto epath{ entry.path() };
		matchcount += pmatches(re, epath,
		search_path);
	}
}
\end{lstlisting}

rdit是recursive\_directory\_iterator类的别名,该类从起始路径遍历目录树,为遇到的每个文件返回一个directory\_entry对象。然后,创建一个path对象,并调用pmatches()遍历文件,从而打印所有正则表达式匹配项。

\item 
此时在main()中,argc是>=2。现在,处理命令行上有一个或多个文件路径的情况:

\begin{lstlisting}[style=styleCXX]
int count{ argc - 2 };
while(count-- > 0) {
	fs::path p{ argv[count + 2] };
	if(!exists(p)) {
		cout << format("not found: {}\n", p);
		continue;
	}
	if(is_directory(p)) {
		for (const auto& entry : rdit{ p }) {
			const auto epath{ entry.path() };
			matchcount += pmatches(re, epath, p);
		}
	} else {
		matchcount += pmatches(re, p, p);
	}
}
\end{lstlisting}

while循环处理命令行上搜索模式以外的一个或多个参数,其检查每个文件名以确保其存在。然后,若其是一个目录,将为recursive\_directory\_iterator类使用rdit别名,来遍历目录并调用pmatches()来打印文件中的模式匹配。

若是单个文件,则可在该文件上调用pmatches()。

\item 
可以用一个参数作为搜索模式运行我们的grep:

\begin{tcblisting}{commandshell={}}
$ ./bwgrep using
dir.cpp 12: using std::format;
dir.cpp 13: using std::cout;
dir.cpp 14: using std::string;
...
formatter.cpp 10: using std::cout;
formatter.cpp 11: using std::string;
formatter.cpp 13: using namespace std::filesystem;
found 33 matches
\end{tcblisting}

可以用第二个参数作为目录来运行:

\begin{tcblisting}{commandshell={}}
$ ./bwgrep using ..
chap04/iterator-adapters.cpp 12: using std::format;
chap04/iterator-adapters.cpp 13: using std::cout;
chap04/iterator-adapters.cpp 14: using std::cin;
...
chap01/hello-version.cpp 24: using std::print;
chap01/chrono.cpp 8: using namespace std::chrono_
literals;
chap01/working.cpp 15: using std::cout;
chap01/working.cpp 34: using std::vector;
found 529 matches
\end{tcblisting}

注意,这里通过遍历目录树来查找子目录中的文件。

或者,可以用一个文件参数运行它:

\begin{tcblisting}{commandshell={}}
$ ./bwgrep using bwgrep.cpp
bwgrep.cpp 13: using std::format;
bwgrep.cpp 14: using std::cout;
bwgrep.cpp 15: using std::string;
...
bwgrep.cpp 22: using rdit = fs::recursive_directory_
iterator;
bwgrep.cpp 23: using match_v = vector<std::pair<size_t,
std::string>>;
found 9 matches
\end{tcblisting}

\end{itemize}

\subsubsection{How it works…}

虽然,这个程序的主要任务是正则表达式匹配,但我们主要关注递归处理文件目录的技术。

recursive\_directory\_iterator对象可与directory\_iterator对象互换,不同的是recursive\_directory\_iterator对象对每个子目录的所有条目进行递归操作。

\subsubsection{See also…}

有关正则表达式的更多信息,请参见第7章的相关章节。












