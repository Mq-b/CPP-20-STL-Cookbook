\begin{flushright}
	\zihao{1} 前言
\end{flushright}

\hspace*{\fill} \\ %插入空行
\noindent
\textbf{关于本书}

本书介绍了使用C++ STL(标准模板库)的方法,以及C++20加入的新特性。

C++是一门强大的底层语言,其建立在C语言的基础上,具有类型安全、泛型编程和面向对象编程的语法扩展等特性。STL提供了一组高级类、函数和算法,可使编程工作更容易、更有效、更不容易出错。

我认为五种语言和特性拼凑成了C++,包括:
1) C语言,
2) 神秘且功能强大的宏预处理器,
3) 功能丰富的类/对象模型,
4) 模板的泛型编程模型,
5) 建立在C++类和模板之上的STL。

\hspace*{\fill} \\ %插入空行
\noindent
\textbf{预备知识}

本书假定读者对C++有基本的了解,包括语法、结构、数据类型、类和对象、模板和STL。

本书中例子假定已经包含某些头文件来使用库函数。示例通常不会列出所有必要的头文件,而更倾向于关注当前描述的技术。建议读者下载示例代码,其中包含完整的代码段。

可以从GitHub下载示例代码: \url{https://github.com/ PacktPublishing/CPP-20-STL-Cookbook}。

\hspace*{\fill} \\ %插入空行
\noindent
\textbf{本书使用C++20}

C++语言由国际标准化组织(ISO)以约三年为一个周期进行标准化,目前的标准称为C++20(在此之前有C++17、C++14和C++11)。C++20于2020年9月获得批准。

C++20为语言和STL添加了许多重要的特性。格式(format)、模块(module)、范围(range)等新特性,将对使用STL的方式产生重大影响。

还有一些变化。例如,想删除vector的匹配的元素,可以使用erase-remove方法:

\begin{lstlisting}[style=styleCXX]
auto it = std::remove(vec1.begin(), vec1.end(), value);
vec1.erase(it, vec1.end());
\end{lstlisting}

从C++20起,可以直接使用新的std::erase函数,并在一个函数调用中完成这些工作:

\begin{lstlisting}[style=styleCXX]
std::erase(vec1, value);
\end{lstlisting}

C++20有许多细微的和实质性的改进。本书中,介绍了其中的大部分内容,特别是与STL相关的特性。

\hspace*{\fill} \\ %插入空行
\noindent
\textbf{大括号初始化}

本书中的方法经常使用大括号初始化,来代替更熟悉的复制初始化。

\begin{lstlisting}[style=styleCMake]
std::string name{ "Jimi Hendrix" }; // braced initialization
std::string name = "Jimi Hendrix"; // copy initialization
\end{lstlisting}

赋值操作符(=)作为赋值操作符和复制操作符,具有双重作用,并且为大家所熟悉的。

但赋值操作符有其缺点,它也是一个复制构造函数,所以数据类型可能会进行隐式窄向转换。这不仅效率低,并且可能导致意外的类型转换,会使程序变得很难调试。

大括号初始化使用列表初始化操作符\{\}(C++11引入),可以避免这些问题。

同样值得注意的是,T\{\}的特殊情况保证为“零初始化”。

\begin{lstlisting}[style=styleCXX]
int x; // uninitialized bad :(
int x = 0; // zero (copy constructed) good :)
int x{}; // zero (zero-initialized) best :D
\end{lstlisting}

空大括号0初始化为初始化新变量,提供了一种更简单快捷的方式。

\hspace*{\fill} \\ %插入空行
\noindent
\textbf{隐藏std::命名空间}

本书中的练习将隐藏std::命名空间,这主要是出于排版和可读性的考虑。我们都知道大多数STL标识符都在std::命名空间中。所以这里会使用某种形式的using声明,以避免使用重复的前缀使示例变得混乱。例如,使用cout时,可以假设已经包含了一个using声明,就像这样:

\begin{lstlisting}[style=styleCXX]
using std::cout; // cout is now sans prefix
cout << "Hello, Jimi!\n";
\end{lstlisting}

示例代码中不会显示using声明,这可以使我们能更加专注于示例所要表达的内容。

但是,代码中导入整个std::命名空间是一种糟糕的做法。应该避免这样使用:

\begin{lstlisting}[style=styleCXX]
using namespace std; // bad. don't do that.
cout << "Hello, Jimi!\n";
\end{lstlisting}

命名空间包含数千个标识符,没有理由让命名空间充满标识符。冲突的可能性还是有的,并且对于调试来说风险更高。当使用一个没有std::前缀的名称时,首选的方法是一次导入一个命名空间。

为了进一步避免命名空间冲突,我经常为重用的类使用单独的命名空间。我倾向于使用bw作为我自己的命名空间,你也可以选择适合你的命名空间名称。

\hspace*{\fill} \\ %插入空行
\noindent
\textbf{使用using进行类型别名}

本书对类型别名使用的是using,而不是typedef。

STL类和类型有时会很冗长。例如,模板化迭代器类可能长这样:

\begin{lstlisting}[style=styleCXX]
std::vector<std::pair<int,std::string>>::iterator
\end{lstlisting}

长类型名不仅很难输入,而且容易出错。

一种常见的方式是用typedef缩写长类型名:

\begin{lstlisting}[style=styleCXX]
typedef std::vector<std::pair<int,std::string>>::iterator
vecit_t
\end{lstlisting}

这为迭代器类型声明了一个别名。typedef继承自C语言,其语法反映了这一点。

从C++11开始,using关键字可以用来创建类型别名:

\begin{lstlisting}[style=styleCXX]
using vecit_t =
std::vector<std::pair<int,std::string>>::iterator;
\end{lstlisting}

通常,using别名等同于typedef。最重要的区别是using别名可以模板化:

\begin{lstlisting}[style=styleCXX]
template<typename T>
using v = std::vector<T>;
v<int> x{};
\end{lstlisting}

本书更倾向于使用using来处理类型别名。

\hspace*{\fill} \\ %插入空行
\noindent
\textbf{简化函数模板}

从C++20开始,可以在没有模板文件的情况下指定简化函数模板。例如:

\begin{lstlisting}[style=styleCXX]
void printc(const auto& c) {
	for (auto i : c) {
		std::cout << i << '\n';
	}
}
\end{lstlisting}

参数列表中的auto类型类似于匿名模板typename:

\begin{lstlisting}[style=styleCXX]
template<typename C>
void printc(const C& c) {
	for (auto i : c) {
		std::cout << i << '\n';
	}
}
\end{lstlisting}

虽然才在C++20标准中出现,但主流编译器早已支持简化函数模板了。本书将在许多示例中使用简化函数模板。

\hspace*{\fill} \\ %插入空行
\noindent
\textbf{C++20的format()}

C++20前,我们可以选择使用传统的printf()或STL cout来格式化文本。虽然两者都有缺陷,但确实好用。从C++20开始,format()函数的加入提供了文本格式化,灵感来自Python3。

\hspace*{\fill} \\ %插入空行
\noindent
\textbf{使用STL解决现实问题}

本书中的方法使用STL为实际问题提供解决方案,仅依赖于STL和C++标准库。这将使读者可以轻松地进行试验和学习,而无需安装和配置第三方依赖库。

现在,让我们与STL一起快乐的玩耍吧!

\hspace*{\fill} \\ %插入空行
\noindent
\textbf{适读人群}

本书是为中高级C++开发者所著,这些开发者更多的想要了解C++20标准模板库的内容。当然,基本的编码知识和C++概念必须要有,从而才能更愉快的阅读本书。

\hspace*{\fill} \\ %插入空行
\noindent
\textbf{本书内容}

\textit{第1章,C++20的新特性},介绍了C++20中新STL的特性,熟悉新的语言功能。

\textit{第2章,STL的泛型特性},讨论在新C++标准中添加的STL。

\textit{第3章,STL容器},介绍了整个STL容器库。

\textit{第4章,兼容迭代器},展示了如何使用和创建STL兼容迭代器。

\textit{第5章,Lambda表达式},将STL函数和算法与Lambda表达式一起使用。

\textit{第6章,STL算法},提供了使用和创建STL兼容算法的方法。

\textit{第7章,字符串、流和格式化},描述了STL字符串和格式化工具类。

\textit{第8章,实用工具类},涵盖了日期和时间、智能指针、optional类等STL实用工具。

\textit{第9章,并发和并行},描述了对并发的支持,包括线程、异步、原子类型等。

\textit{第10章,文件系统},介绍了std::filesystem,以及如何将其与C++20的新特性结合起来使用。

\textit{第11章,一些的想法},提供了一些解决方案,包括trie类型,字符串分割等。展示了如何将STL用于实际问题的高级示例。

\hspace*{\fill} \\ %插入空行
\noindent
\textbf{本书中使用的GCC编译器}

除非另有说明,本书中的大多数示例都是使用GCC编译器11.2版本开发和测试的,11.2版本是撰写本文时最新的稳定版本。

在撰写本书时,C++20仍然很新,没有在任何编译器上完全实现。在GCC(GNU)、MSVC(微软)和Clang(苹果)这三个主要编译器中,MSVC编译器在实现新标准方面走得最快。不过,可能还是会遇到在MSVC或其他编译器上实现的特性,但GCC还没实现。这时,会明确说明使用的编译器。若某个特性还没有在可用的编译器上实现,会告诉读者“目前无法测试这个特性”。

\begin{table}[H]
\centering
\begin{tabular}{|cc|}
\hline
\multicolumn{2}{|c|}{代码可能已经在一个或多个这样的编译器上进行过测试} \\ \hline
\multicolumn{1}{|c|}{GCC 11.2}                  & Debian Linux 5.16.11    \\ \hline
\multicolumn{1}{|c|}{LLVM/Clang 13.1.6}         & macOS 12.13/Darwin 21.4 \\ \hline
\multicolumn{1}{|c|}{Microsoft C++ 19.32.31302} & Windows 10              \\ \hline
\end{tabular}
\end{table}

这里,强烈建议您安装GCC以遵循本书中的示例。GCC是根据GNU通用公共许可证(GPL)免费提供的。获得最新版本GCC的最简单方法是安装Debian Linux(也是GPL)操作系统,并使用apt进行安装GCC。

\textbf{若正在使用这本书的数字版本,建议自己输入代码或从本书的GitHub库访问代码(下一节有链接)。这样做将避免与复制和粘贴代码相关的错误。}

\hspace*{\fill} \\ %插入空行
\noindent
\textbf{下载示例}

可以从GitHub上下载这本书的示例代码文件\url{https://github.com/PacktPublishing/CPP-20-STL-Cookbook}。若代码有更新,会在GitHub库中更新。



















