
C++17中引入的std::variant类可以保存不同的值,每次一个,其中每个值必须适合相同的分配内存空间。对于保存可供在单个上下文中使用的替代类型非常有用。

\subsubsection{与union的区别}

variant类是一个带标记的联合,其与原始union结构的不同之处在于,一次只能有一种类型起作用。

原始union类型继承自C,是一种结构,在这种结构中,相同的数据可以作为不同的类型访问。例如:

\begin{lstlisting}[style=styleCXX]
union ipv4 {
	struct {
		uint8_t a; uint8_t b; uint8_t c; uint8_t d;
	}quad;
	uint32_t int32;
} addr;
addr.int32 = 0x2A05A8C0;
cout << format("ip addr dotted quad: {}.{}.{}.{}\n",
	addr.quad.a, addr.quad.b, addr.quad.c, addr.quad.d);
cout << format("ip addr int32 (LE): {:08X}\n", addr.int32);
\end{lstlisting}

输出为:

\begin{tcblisting}{commandshell={}}
ip addr dotted quad: 192.168.5.42
ip addr int32 (LE): 2A05A8C0
\end{tcblisting}

这个例子中,union有两个成员,类型是struct和uint32\_t,其中struct有四个uint8\_t成员。这为我们提供了相同32位内存空间的两种不同视角,可以将相同的ipv4地址视为一个32位无符号整数(Little Endian或LE)或四个8位无符号整数。这提供了在系统级别的位多态。

variant不是这样的。variant类是一个带标记的union,其中每个数据都带有其类型的标记。若将一个值存储为uint32\_t,只能以uint32\_t的形式访问。这使得variant类型安全,但并不是union的替代品。

\subsubsection{How to do it…}

示例中,将展示std::variant的使用方式。

\begin{itemize}
\item 
我们将从一个简单的类开始,先来保存一个Animal:

\begin{lstlisting}[style=styleCXX]
class Animal {
	string_view _name{};
	string_view _sound{};
	Animal();
public:
	Animal(string_view n, string_view s)
	: _name{ n }, _sound{ s } {}
	void speak() const {
		cout << format("{} says {}\n", _name, _sound);
	}
	void sound(string_view s) {
		_sound = s;
	}
}
\end{lstlisting}

动物的名字和动物发出的声音被传递到构造函数中。

\item 
不同的物种类继承自Animal:

\begin{lstlisting}[style=styleCXX]
class Cat : public Animal {
	public:
	Cat(string_view n) : Animal(n, "meow") {}
};
class Dog : public Animal {
	public:
	Dog(string_view n) : Animal(n, "arf!") {}
};
class Wookie : public Animal {
	public:
	Wookie(string_view n) : Animal(n, "grrraarrgghh!") {}
};
\end{lstlisting}

这些类中的每一个都通过调用父构造函数,来为其物种类设置声音。

\item 
现在,可以在别名中定义变量类型:

\begin{lstlisting}[style=styleCXX]
using v_animal = std::variant<Cat, Dog, Wookie>;
\end{lstlisting}

这个变体可以容纳任何类型,猫,狗或伍基。

\item 
在main()中,使用v\_animal别名作为类型创建了一个列表:

\begin{lstlisting}[style=styleCXX]
int main() {
	list<v_animal> pets{
		Cat{"Hobbes"}, Dog{"Fido"}, Cat{"Max"},
		Wookie{"Chewie"}
	};
	...
\end{lstlisting}

列表中的每个元素都是变量定义中包含的类型。

\item 
variant类提供了几种访问元素的不同方法,先来看看visit()。

visit()调用带有当前包含在变体中的对象的函子。首先,定义一个接受宠物的函子:

\begin{lstlisting}[style=styleCXX]
struct animal_speaks {
	void operator()(const Dog& d) const { d.speak(); }
	void operator()(const Cat& c) const { c.speak(); }
	void operator()(const Wookie& w) const {
		w.speak(); }
};
\end{lstlisting}

这是一个简单的函子类,每个Animal子类都有重载。使用visit()调用它,并使用列表的每个元素:

\begin{lstlisting}[style=styleCXX]
for (const v_animal& a : pets) {
	visit(animal_speaks{}, a);
}
\end{lstlisting}

会得到这样的输出:

\begin{tcblisting}{commandshell={}}
Hobbes says meow
Fido says arf!
Max says meow
Chewie says grrraarrgghh!
\end{tcblisting}

\item 
variant类还提供了一个index()方法:

\begin{lstlisting}[style=styleCXX]
for(const v_animal &a : pets) {
	auto idx{ a.index() };
	if(idx == 0) get<Cat>(a).speak();
	if(idx == 1) get<Dog>(a).speak();
	if(idx == 2) get<Wookie>(a).speak();
}
\end{lstlisting}

可以得到这样的输出:

\begin{tcblisting}{commandshell={}}
Hobbes says meow
Fido says arf!
Max says meow
Chewie says grrraarrgghh!
\end{tcblisting}

每个变量对象都根据模板参数中声明类型的顺序建立索引。av\_animal类型定义为std::variant<Cat, Dog, Wookie>,这些类型的索引顺序为0 - 2。

\item 
get\_if<T>()函数的作用是:根据类型测试给定元素:

\begin{lstlisting}[style=styleCXX]
for (const v_animal& a : pets) {
	if(const auto c{ get_if<Cat>(&a) }; c) {
		c->speak();
	} else if(const auto d{ get_if<Dog>(&a) }; d) {
		d->speak();
	} else if(const auto w{ get_if<Wookie>(&a) }; w) {
		w->speak();
	}
}
\end{lstlisting}

输出为:

\begin{tcblisting}{commandshell={}}
Hobbes says meow
Fido says arf!
Max says meow
Chewie says grrraarrgghh!
\end{tcblisting}

get\_if<T>()函数的作用是:若元素类型与T匹配,则返回一个指针;否则,返回nullptr。

\item 
最后,hold\_alternative <T>()函数返回true或false。可以使用this来测试一个类型是否对应一个元素,而不返回该元素:

\begin{lstlisting}[style=styleCXX]
size_t n_cats{}, n_dogs{}, n_wookies{};
for(const v_animal& a : pets) {
	if(holds_alternative<Cat>(a)) ++n_cats;
	if(holds_alternative<Dog>(a)) ++n_dogs;
	if(holds_alternative<Wookie>(a)) ++n_wookies;
}
cout << format("there are {} cat(s), "
				"{} dog(s), "
				"and {} wookie(s)\n",
				n_cats, n_dogs, n_wookies);
\end{lstlisting}

输出为:

\begin{tcblisting}{commandshell={}}
there are 2 cat(s), 1 dog(s), and 1 wookie(s)
\end{tcblisting}
\end{itemize}

\subsubsection{How it works…}

std::variant类是一个单对象容器。variant<X, Y, Z>实例的类型,必须是X, Y或Z对象类型的其中之一,并可以同时保存当前对象的值和类型。

index()方法告诉我们当前对象的类型:

\begin{lstlisting}[style=styleCXX]
if(v.index() == 0) // if variant is type X
\end{lstlisting}

holds\_alternative<T>()非成员函数如果T是当前对象的类型,则返回true:

\begin{lstlisting}[style=styleCXX]
if(holds_alternative<X>(v)) // if current variant obj is type X
\end{lstlisting}

可以使用get()非成员函数检索当前对象:

\begin{lstlisting}[style=styleCXX]
auto o{ get<X>(v) }; // current variant obj must be type X
\end{lstlisting}

可以将类型和检索测试与get\_if()非成员函数结合起来:

\begin{lstlisting}[style=styleCXX]
auto* p{ get_if<X>(v) }; // nullptr if current obj not type X
\end{lstlisting}

visit()非成员函数使用一个可调用对象,将当前变量对象作为其唯一形参:

\begin{lstlisting}[style=styleCXX]
visit(f, v); // calls f(v) with current variant obj
\end{lstlisting}

visit()函数是在不测试对象类型的情况下,检索对象的唯一方法。结合一个可以处理每种类型的函子,就非常灵活了:

\begin{lstlisting}[style=styleCXX]
struct animal_speaks {
	void operator()(const Dog& d) const { d.speak(); }
	void operator()(const Cat& c) const { c.speak(); }
	void operator()(const Wookie& v) const { v.speak(); }
};
main() {
	for (const v_animal& a : pets) {
		visit(animal_speaks{}, a);
	}
}
\end{lstlisting}

输出为:

\begin{tcblisting}{commandshell={}}
Hobbes says meow
Fido says arf!
Max says meow
Chewie says grrraarrgghh!
\end{tcblisting}


