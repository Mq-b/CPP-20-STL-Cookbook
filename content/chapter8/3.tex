
C++17中引入了std::any类,可为任何类型的单个对象提供了类型安全的容器。

例如,这是一个默认构造的any对象:

\begin{lstlisting}[style=styleCXX]
any x{};
\end{lstlisting}

该对象无值,可以使用has\_value()方法进行测试:

\begin{lstlisting}[style=styleCXX]
if(x.has_value()) cout << "have value\n";
else cout << "no value\n";
\end{lstlisting}

输出为:

\begin{tcblisting}{commandshell={}}
no value
\end{tcblisting}

我们使用赋值操作符为any对象赋值:

\begin{lstlisting}[style=styleCXX]
x = 42;
\end{lstlisting}

现在,any对象有一个值和一个类型:

\begin{lstlisting}[style=styleCXX]
if(x.has_value()) {
	cout << format("x has type: {}\n", x.type().name());
	cout << format("x has value: {}\n", any_cast<int>(x));
} else {
	cout << "no value\n";
}
\end{lstlisting}

输出为:

\begin{tcblisting}{commandshell={}}
x has type: i
x has value: 42
\end{tcblisting}

type()方法返回一个type\_info对象。type\_info::name()方法返回C-string类型的实现定义的名称。在本例中,GCC中i表示int。

我们使用any\_cast<type>()非成员函数强制转换值进行使用。

可以用不同类型的不同值重新赋值:

\begin{lstlisting}[style=styleCXX]
x = "abc"s;
cout << format("x is type {} with value {}\n",
	x.type().name(), any_cast<string>(x))
\end{lstlisting}

输出为:

\begin{tcblisting}{commandshell={}}
x is type NSt7__cxx1112basic_string... with value abc
\end{tcblisting}

我用GCC缩写了这个长类型名,相同的任何对象,曾经持有int,现在包含一个STL字符串对象。

any类的主要用途是创建一个多态函数。让我们来看看如何在这个示例中做到这一点:

\subsubsection{How to do it…}

这个示例中,我们将使用any类构建一个多态函数。多态函数是可以在其参数中接受不同类型对象的函数:

\begin{itemize}
\item 
我们的多态函数可以接受任意对象并,打印其类型和值:

\begin{lstlisting}[style=styleCXX]
void p_any(const any& a) {
	if (!a.has_value()) {
		cout << "None.\n";
	} else if (a.type() == typeid(int)) {
		cout << format("int: {}\n", any_cast<int>(a));
	} else if (a.type() == typeid(string)) {
		cout << format("string: \"{}\"\n",
		any_cast<const string&>(a));
	} else if (a.type() == typeid(list<int>)) {
		cout << "list<int>: ";
		for(auto& i : any_cast<const list<int>&>(a))
			cout << format("{} ", i);
		cout << '\n';
	} else {
		cout << format("something else: {}\n",
		a.type().name());
	}
}
\end{lstlisting}

p\_any()函数首先测试对象是否有值。然后,针对各种类型测试type()方法,并对每种类型采取适当的操作。

在any类之前,我们必须为这个函数编写四种不同的专门化,并且仍然不能轻松地处理默认情况。

\item 
我们在main()调用这个函数,如下所示:

\begin{lstlisting}[style=styleCXX]
p_any({});
p_any(47);
p_any("abc"s);
p_any(any(list{ 1, 2, 3 }));
p_any(any(vector{ 1, 2, 3 }));
\end{lstlisting}

输出为:

\begin{tcblisting}{commandshell={}}
one.
int: 47
string: "abc"
list<int>: 1 2 3
something else: St6vectorIiSaIiEE
\end{tcblisting}

\end{itemize}

我们的多态函数用最少的代码处理了各种类型的变量。

\subsubsection{How it works…}

any复制构造函数和赋值操作符,可以直接初始化来生成目标对象的非const副本作为包含对象。所包含对象的类型作为typeid对象单独存储。

初始化后,any对象有以下方法:

\begin{itemize}
\item 
emplace()替换包含的对象,在适当的位置构造新对象。

\item 
reset() 替换包含的对象,在适当的位置构造新对象。

\item 
has\_value() 若有包含对象,则返回true。

\item 
type() 返回typeid对象,表示所包含对象的类型。

\item 
operator=() 用复制或移动操作替换包含的对象。
\end{itemize}

any类还支持以下非成员函数:

\begin{itemize}
\item 
any\_cast<T>(),模板函数,提供对包含对象的类型安全访问。

请记住,any\_cast<T>()函数返回包含对象的副本。可以使用any\_cast<T\&>()来返回引用。

\item 
std::swap()专门处理std::swap算法。
\end{itemize}

若用错误的类型强制转换any对象,它会抛出一个bad\_any\_cast异常:

\begin{lstlisting}[style=styleCXX]
try {
	cout << any_cast<int>(x) << '\n';
} catch(std::bad_any_cast& e) {
	cout << format("any: {}\n", e.what());
}
\end{lstlisting}

输出为:

\begin{tcblisting}{commandshell={}}
any: bad any_cast
\end{tcblisting}

















