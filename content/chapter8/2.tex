
C++17引入的std::optional类保存了一个可选值。

假设有一个函数,它可能返回值,也可能不返回值——例如,一个函数检查一个数字是否是质数,但若存在第一个因数,则返回第一个因数。此函数应返回值或bool状态。我们可以创建一个同时包含值和状态的结构体:

\begin{lstlisting}[style=styleCXX]
struct factor_t {
	bool is_prime;
	long factor;
};
factor_t factor(long n) {
	factor_t r{};
	for(long i = 2; i <= n / 2; ++i) {
		if (n % i == 0) {
			r.is_prime = false;
			r.factor = i;
			return r;
		}
	}
	r.is_prime = true;
	return r;
}
\end{lstlisting}

这是一个笨拙的解决方案,但它是有效的,而且很常见。

optional类可以让这个任务变得简单很多:

\begin{lstlisting}[style=styleCXX]
optional<long> factor(long n) {
	for (long i = 2; i <= n / 2; ++i) {
		if (n % i == 0) return {i};
	}
	return {};
}
\end{lstlisting}

使用optional,可以返回值或非值。

我们可以这样使用:

\begin{lstlisting}[style=styleCXX]
long a{ 42 };
long b{ 73 };
auto x = factor(a);
auto y = factor(b);
if(x) cout << format("lowest factor of {} is {}\n", a, *x);
else cout << format("{} is prime\n", a);
if(y) cout << format("lowest factor of {} is {}\n", b, *y);
else cout << format("{} is prime\n", b);
\end{lstlisting}

输出是:

\begin{tcblisting}{commandshell={}}
lowest factor of 42 is 2
73 is prime
\end{tcblisting}

\subsubsection{How to do it…}

这个示例中,来看一些如何使用optional类的例子:

\begin{itemize}
\item 
optional类非常简单。我们使用标准模板表示法构造一个optional值:

\begin{lstlisting}[style=styleCXX]
optional<int> a{ 42 };
cout << *a << '\n';
\end{lstlisting}

使用指针的解引用操作符访问可选对象的值。

输出为:

\begin{tcblisting}{commandshell={}}
42
\end{tcblisting}

\item 
使用bool操作符来测试optional是否有值:

\begin{lstlisting}[style=styleCXX]
if(a) cout << *a << '\n';
else cout << "no value\n";
\end{lstlisting}

若构造a时不带值:

\begin{lstlisting}[style=styleCXX]
optional<int> a{};
\end{lstlisting}

输出将输出else的内容:

\begin{tcblisting}{commandshell={}}
no value
\end{tcblisting}

\item 
我们可以通过声明类型别名来进一步简化:

\begin{lstlisting}[style=styleCXX]
using oint = std::optional<int>;
oint a{ 42 };
oint b{ 73 };
\end{lstlisting}

\item 
若想对oint对象进行操作,可以对操作符进行重载:

\begin{lstlisting}[style=styleCXX]
oint operator+(const oint& a, const oint& b) {
	if(a && b) return *a + *b;
	else return {};
}
oint operator+(const oint& a, const int b) {
	if(a) return *a + b;
	else return {};
}
\end{lstlisting}

现在,可以直接操作oint对象:

\begin{lstlisting}[style=styleCXX]
auto sum{ a + b };
if(sum) {
	cout << format("{} + {} = {}\n", *a, *b, *sum);
} else {
	cout << "NAN\n";
}
\end{lstlisting}

输出为:

\begin{lstlisting}[style=styleCXX]
42 + 73 = 115
\end{lstlisting}

\item 
假设用默认构造函数声明b:

\begin{lstlisting}[style=styleCXX]
oint b{};
\end{lstlisting}

现在,可得到else分支的输出:

\begin{lstlisting}[style=styleCXX]
NAN
\end{lstlisting}
\end{itemize}


\subsubsection{How it works…}

std::optional类是为了简单而创建的,其为许多常用函数提供操作符重载。还包括成员函数,可提供更进一步的灵活性。

optional类提供了一个操作符bool重载,用于确定对象是否有值:

\begin{lstlisting}[style=styleCXX]
optional<int> n{ 42 };
if(n) ... // has a value
\end{lstlisting}

或者,可以使用has\_value()成员函数:

\begin{lstlisting}[style=styleCXX]
if(n.has_value()) ... // has a value
\end{lstlisting}

要访问该值,可以使用解引用操作符的重载:

\begin{lstlisting}[style=styleCXX]
x = *n; // * retruns the value
\end{lstlisting}

或者,可以使用value()成员函数:

\begin{lstlisting}[style=styleCXX]
x = n.value(); // * retruns the value
\end{lstlisting}

reset()成员函数可以销毁该值,并重置可选对象的状态:

\begin{lstlisting}[style=styleCXX]
n.reset(); // no longer has a value
\end{lstlisting}


\subsubsection{There's more…}

optional类通过value()方法提供异常支持:

\begin{lstlisting}[style=styleCXX]
b.reset();
try {
	cout << b.value() << '\n';
} catch(const std::bad_optional_access& e) {
	cout << format("b.value(): {}\n", e.what());
}
\end{lstlisting}

输出为:

\begin{tcblisting}{commandshell={}}
b.value(): bad optional access
\end{tcblisting}

\begin{tcolorbox}[colback=webgreen!5!white,colframe=webgreen!75!black,title=Note]
只有value()方法会抛出异常。对于无效值,解引用操作符的行为未定义。
\end{tcolorbox}







