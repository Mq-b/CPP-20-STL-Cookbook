
std::shared\_ptr类提供了一个别名构造函数,来共享由另一个不相关指针管理的指针:

\begin{lstlisting}[style=styleCXX]
shared_ptr( shared_ptr<Y>&& ref, element_type* ptr ) noexcept;
\end{lstlisting}

这将返回一个别名shared\_ptr对象,该对象使用ref的资源,但返回一个指向ptr的指针,use\_count和deleter与ref共享,但是get()返回ptr。这可以在不共享整个对象的情况下共享托管对象的一个成员,并且在仍然使用该成员时不允许删除对象。

\subsubsection{How to do it…}

在这个示例中,我们创建了一个托管对象并共享该对象的成员:

\begin{itemize}
\item 
从托管对象的类开始:

\begin{lstlisting}[style=styleCXX]
struct animal {
	string name{};
	string sound{};
	animal(const string& n, const string& a)
	: name{n}, sound{a} {
		cout << format("ctor: {}\n", name);
	}
	~animal() {
		cout << format("dtor: {}\n", name);
	}
};
\end{lstlisting}

该类有两个成员,animal对象的名称和声音的字符串类型。构造函数和析构函数也有print语句。

\item 
现在,需要一个函数来创建一个animal,但只分享它的名字和声音:

\begin{lstlisting}[style=styleCXX]
auto make_animal(const string& n, const string& s) {
	auto ap = make_shared<animal>(n, s);
	auto np = shared_ptr<string>(ap, &ap->name);
	auto sp = shared_ptr<string>(ap, &ap->sound);
	return tuple(np, sp);
}
\end{lstlisting}

这个函数创建与动物对象shared\_ptr,由名称和声音构造。然后,为名称和声音创建别名shared\_ptr对象。当返回name和sound指针时,animal指针将超出作用域。因为别名指针可以防止使用计数达到零,所以它不会删除。

\item 
main()函数中,使用make\_animal()并检查结果:

\begin{lstlisting}[style=styleCXX]
int main() {
	auto [name, sound] =
		make_animal("Velociraptor", "Grrrr!");
	cout << format("The {} says {}\n", *name, *sound);
	cout << format("Use count: name {}, sound {}\n",
		name.use_count(), sound.use_count());
}
\end{lstlisting}

输出为:

\begin{tcblisting}{commandshell={}}
ctor: Velociraptor
The Velociraptor says Grrrr!
Use count: name 2, sound 2
dtor: Velociraptor
\end{tcblisting}

每个别名指针的use\_count值为2。当make\_animal()函数创建别名指针时,其会增加animal指针的使用次数。当函数结束时,animal指针超出作用域,使其使用计数为2,这反映在别名指针中。别名指针在main()的末尾超出了作用域,这可以销毁animal指针。

\end{itemize}

\subsubsection{How it works…}

别名共享指针看起来有点抽象,但它比看起来要简单。

共享指针使用一个控制块来管理它的资源。一个控制块与一个托管对象相关联,并在共享该对象的指针之间共享。控制块一般包含:

\begin{itemize}
\item 
指向管理对象的指针

\item 
删除器

\item 
分配器

\item 
拥有托管对象的shared\_ptr对象的数量(这是使用计数)

\item 
引用管理对象的weak\_ptr对象的数量
\end{itemize}

在使用别名共享指针的情况下,控制块包括指向别名对象的指针,其他不变。

别名共享指针参与使用计数,就像非别名共享指针一样,防止托管对象在使用计数为零之前被销毁。删除器没有改变,所以它会破坏管理对象。

\begin{tcolorbox}[colback=webgreen!5!white,colframe=webgreen!75!black,title=重要的Note]
可以使用任何指针来构造别名共享指针。通常,指针指向别名对象中的成员。若别名指针没有引用托管对象的元素,则需要分别管理其构造和销毁。
\end{tcolorbox}




























