

当从用户接收输入时,通常会在字符串中出现过多的连续空格字符。本节提供了一个删除连续空格的函数,即使其中包含制表符或其他空白字符.

\subsubsection{How to do it…}

这个函数利用std::unique()算法,从字符串中移除连续的空白字符。

\begin{itemize}
\item 
bw命名空间中,先从一个检测空白的函数说起:

\begin{lstlisting}[style=styleCXX]
template<typename T>
bool isws(const T& c) {
	constexpr const T whitespace[]{ " \t\r\n\v\f" };
	for(const T& wsc : whitespace) {
		if(c == wsc) return true;
	}
	return false;
}
\end{lstlisting}

这个模板化的isws()函数应该适用于任何字符类型。

\item 
delws()函数使用std::unique()来擦除字符串中连续的空格:

\begin{lstlisting}[style=styleCXX]
string delws(const string& s) {
	string outstr{s};
	auto its = unique(outstr.begin(), outstr.end(),
		[](const auto &a, const auto &b) {
			return isws(a) && isws(b);
		});
	outstr.erase(its, outstr.end());
	outstr.shrink_to_fit();
	return outstr;
}
\end{lstlisting}

delws()复制输入字符串,删除连续的空格,并返回新字符串。

\item 
可以使用main()中的字符串调用:

\begin{lstlisting}[style=styleCXX]
int main() {
	const string s{ "big bad \t wolf" };
	const string s2{ bw::delws(s) };
	cout << format("[{}]\n", s);
	cout << format("[{}]\n", s2);
	return 0;
}
\end{lstlisting}

输出为:

\begin{tcblisting}{commandshell={}}
[big     bad                  wolf]
[big bad wolf]
\end{tcblisting}

\end{itemize}

\subsubsection{How it works…}

这个函数使用std::unique()算法和一个比较lambda来查找字符串对象中的连续空格。

比较lambda调用isws()函数来确定是否找到了连续的空格:

\begin{lstlisting}[style=styleCXX]
auto its = unique(outstr.begin(), outstr.end(),
	[](const auto &a, const auto &b) {
		return isws(a) && isws(b);
	});
\end{lstlisting}

可以使用标准库中的isspace()函数,但它是一个标准C函数,依赖于从int到char的窄类型转换。这可能会在一些现代C++编译器上触发警告,并且不能保证在没有显式强制转换的情况下正常工作。我们的isws()函数使用了模板化的类型,应该适用于任何系统,以及std::string的任何特化。


















