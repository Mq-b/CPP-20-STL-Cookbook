
gather是一个利用现有算法的算法示例。

gather算法接受一对容器迭代器,并将满足谓词的元素移动到序列中的相应(枢轴)位置,返回一对包含满足谓词的元素的迭代器。

例如,可以使用收集算法将所有偶数排序到vector的中点:

\begin{lstlisting}[style=styleCXX]
vector<int> vint{ 0, 1, 2, 3, 4, 5, 6, 7, 8, 9 };
gather(vint.begin(), vint.end(), mid(vint), is_even);
for(const auto& el : vint) cout << el;
\end{lstlisting}

输出为:

\begin{tcblisting}{commandshell={}}
1302468579
\end{tcblisting}

注意,偶数都在输出的中间。

在这个示例中,我们将使用标准STL算法实现一个gather算法。

\subsubsection{How to do it…}

收集算法使用std::stable\_partition()算法,将项移动到枢轴迭代器之前,并再次将项移动到枢轴迭代器之后。

\begin{itemize}
\item 
将算法放在bw命名空间中以避免冲突。

\begin{lstlisting}[style=styleCXX]
namespace bw {
	using std::stable_partition;
	using std::pair;
	using std::not_fn;
	
	template <typename It, typename Pred>
	pair<It, It> gather(It first, It last, It pivot,
	Pred pred) {
		return {stable_partition(first, pivot, not_fn(pred)),
			stable_partition(pivot, last, pred)};
	}
};
\end{lstlisting}

gather()算法返回一对迭代器,由两次调用stable\_partition()返回。

\item 
还包括了一些辅助lambda:

\begin{lstlisting}[style=styleCXX]
constexpr auto midit = [](auto& v) {
	return v.begin() + (v.end() - v.begin()) / 2;
};
constexpr auto is_even = [](auto i) {
	return i % 2 == 0;
};
constexpr auto is_even_char = [](auto c) {
	if(c >= '0' && c <= '9') return (c - '0') % 2 == 0;
	else return false;
};
\end{lstlisting}

这三个lambda如下:

\begin{itemize}
\item 
midit返回容器中点的迭代器,用作枢轴点。

\item 
is\_even如果值为偶数则返回布尔值true,用作谓词。

\item 
is\_even\_char如果值是'0'和'9'之间的字符,则返回布尔值true,并且是偶数,用于谓词。
\end{itemize}

\item 
可以在main()函数中调用gather(),使用int vector,如下所示:

\begin{lstlisting}[style=styleCXX]
int main() {
	vector<int> vint{ 0, 1, 2, 3, 4, 5, 6, 7, 8, 9 };
	auto gathered_even = bw::gather(vint.begin(),
		vint.end(), bw::midit(vint), bw::is_even);
	for(const auto& el : vint) cout << el;
	cout << '\n';
}
\end{lstlisting}

输出显示偶数聚集在了中间:

\begin{tcblisting}{commandshell={}}
1302468579
\end{tcblisting}

gather()函数返回一对只包含偶数值的迭代器:

\begin{lstlisting}[style=styleCXX]
auto& [it1, it2] = gathered_even;
for(auto it{ it1 }; it < it2; ++it) cout << *it;
cout << '\n';
\end{lstlisting}

输出为:

\begin{tcblisting}{commandshell={}}
02468
\end{tcblisting}

\item 
可以将枢轴点设置为begin()或end()迭代器:

\begin{lstlisting}[style=styleCXX]
bw::gather(vint.begin(), vint.end(), vint.begin(),
	bw::is_even);
for(const auto& el : vint) cout << el;
cout << '\n';
bw::gather(vint.begin(), vint.end(), vint.end(),
	bw::is_even);
for(const auto& el : vint) cout << el;
cout << '\n';
\end{lstlisting}

输出为:

\begin{tcblisting}{commandshell={}}
0246813579
1357902468
\end{tcblisting}

\item 
因为gather()是基于迭代器的,所以可以将它用于任何容器。这是一串字符数字:

\begin{lstlisting}[style=styleCXX]
string jenny{ "867-5309" };
bw::gather(jenny.begin(), jenny.end(), jenny.end(),
	bw::is_even_char);
for(const auto& el : jenny) cout << el;
cout << '\n';
\end{lstlisting}

这将把所有偶数移动到字符串的末尾:

输出为:

\begin{tcblisting}{commandshell={}}
7-539860
\end{tcblisting}

\end{itemize}

\subsubsection{How it works…}

gather()函数使用std::stable\_partition()算法将与谓词匹配的元素移动到枢轴点。

gather()有两个对stable\_partition()的调用,一个带有谓词,另一个使用谓词取反结果:

\begin{lstlisting}[style=styleCXX]
template <typename It, typename Pred>
pair<It, It> gather(It first, It last, It pivot, Pred pred) {
	return { stable_partition(first, pivot, not_fn(pred)),
		stable_partition(pivot, last, pred) };
}
\end{lstlisting}

从两个stable\_partition()调用返回的迭代器以pair形式返回。










