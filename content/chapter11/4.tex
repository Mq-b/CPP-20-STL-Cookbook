
STL具有丰富的算法库。然而,有时可能会发现它缺少自己需要的东西。一个常见的需求是split函数。

split函数在字符分隔符上拆分字符串。例如,下面是标准Debian安装的Unix /etc/passwd文件:

\begin{tcblisting}{commandshell={}}
root:x:0:0:root:/root:/bin/bash
daemon:x:1:1:daemon:/usr/sbin:/usr/sbin/nologin
bin:x:2:2:bin:/bin:/usr/sbin/nologin
sys:x:3:3:sys:/dev:/usr/sbin/nologin
sync:x:4:65534:sync:/bin:/bin/sync
\end{tcblisting}

每个字段用冒号:character分隔,其中字段为:


\begin{enumerate}
\item 
账号

\item 
可加密的密码

\item 
用户名

\item 
组名

\item 
用户名或评论

\item 
主目录

\item 
可选的命令解释器
\end{enumerate}

这是基于POSIX的操作系统中的标准文件,还有其他类似的文件。大多数脚本语言都包含一个内置函数,用于在分隔符上分隔字符串。C++中有一些简单的方法可以做到这一点,但std::string只是STL中的另一个容器,在分隔符上分割容器的通用算法可能是有用补充。让我们来创建一个。

\subsubsection{How to do it…}

在这个示例中,我们构建了一个通用算法,它在分隔符上拆分容器,并将结果放入目标容器中。

\begin{itemize}
\item 
我们的算法在bw命名空间中,以避免与std冲突:

\begin{lstlisting}[style=styleCXX]
namespace bw {
	template<typename It, typename Oc, typename V,
	typename Pred>
	It split(It it, It end_it, Oc& dest,
	const V& sep, Pred& f) {
		using SliceContainer = typename
		Oc::value_type;
		while(it != end_it) {
			SliceContainer dest_elm{};
			auto slice{ it };
			while(slice != end_it) {
				if(f(*slice, sep)) break;
				dest_elm.push_back(*slice++);
			}
			dest.push_back(dest_elm);
			if(slice == end_it) return end_it;
			it = ++slice;
		}
		return it;
	}
};
\end{lstlisting}

split()算法在容器中搜索分隔符,并将分离的切片收集到新的输出容器中,其中每部分都是输出容器中的容器。

我希望split()算法尽可能通用,就像算法库中的算法一样。这样,所有的参数都是模板化的,代码可以处理各种各样的参数类型。

首先,来看看模板参数:

\begin{itemize}
\item 
It是源容器的输入迭代器类型。

\item 
Oc输出容器类型。这是容器中的容器。

\item 
V分隔符类型。

\item 
Pred为谓词函数。
\end{itemize}

输出类型是容器的容器,需要容纳切片的容器。可以是vector<string>,其中字符串值是切片,也可以是vector<vector<int>{}>,其中内部的vector<int>包含切片,所以需要从输出容器类型派生内部容器的类型。可以通过函数体中的using声明来做到这一点。

\begin{lstlisting}[style=styleCXX]
using SliceContainer = typename Oc::value_type;
\end{lstlisting}

这也是为什么不能为输出形参使用输出迭代器的原因。根据定义,输出迭代器不能确定其内容的类型,并且将其value\_type设置为void。

使用SliceContainer定义一个临时容器,可以将其添加到输出容器中:

\begin{lstlisting}[style=styleCXX]
dest.push_back(dest_elm);
\end{lstlisting}

\item 
谓词是一个二元操作符,用于比较输入元素和分隔符。在bw命名空间中包含了一个默认的相等操作符:

\begin{lstlisting}[style=styleCXX]
constexpr auto eq = [](const auto& el, const auto& sep) {
	return el == sep;
};
\end{lstlisting}

\item 
还包含了split()的特化,默认使用eq操作符:

\begin{lstlisting}[style=styleCXX]
template<typename It, typename Oc, typename V>
It split(It it, const It end_it, Oc& dest, const V& sep)
{
	return split(it, end_it, dest, sep, eq);
}
\end{lstlisting}

\item 
因为分割字符串对象是这个算法的常见用例,包含了一个用于特定目的的辅助函数:

\begin{lstlisting}[style=styleCXX]
template<typename Cin, typename Cout, typename V>
Cout& strsplit(const Cin& str, Cout& dest, const V& sep)
{
	split(str.begin(), str.end(), dest, sep, eq);
	return dest;
}
\end{lstlisting}

\item 
我们测试split算法main()函数:

\begin{lstlisting}[style=styleCXX]
int main() {
	constexpr char strsep{ ':' };
	const string str
		{ "sync:x:4:65534:sync:/bin:/bin/sync" };
	vector<string> dest_vs{};
	bw::split(str.begin(), str.end(), dest_vs, strsep,
		bw::eq);
	for(const auto& e : dest_vs) cout <<
		format("[{}] ", e);
	cout << '\n';
}
\end{lstlisting}

使用/etc/passwd文件中的字符串来测试算法,结果如下:

\begin{tcblisting}{commandshell={}}
[sync] [x] [4] [65534] [sync] [/bin] [/bin/sync]
\end{tcblisting}

\item 
使用strsplit()帮助函数会更简单:

\begin{lstlisting}[style=styleCXX]
vector<string> dest_vs2{};
bw::strsplit(str, dest_vs2, strsep);
for(const auto& e : dest_vs2) cout << format("[{}] ", e);
cout << '\n';
\end{lstlisting}

输出为:

\begin{tcblisting}{commandshell={}}
[sync] [x] [4] [65534] [sync] [/bin] [/bin/sync]
\end{tcblisting}

这样就可以很容易地解析/etc/passwd文件。

\item 
当然,可以对任何容器使用相同算法:

\begin{lstlisting}[style=styleCXX]
constexpr int intsep{ -1 };
vector<int> vi{ 1, 2, 3, 4, intsep, 5, 6, 7, 8, intsep,
	9, 10, 11, 12 };
vector<vector<int>> dest_vi{};
bw::split(vi.begin(), vi.end(), dest_vi, intsep);
for(const auto& v : dest_vi) {
	string s;
	for(const auto& e : v) s += format("{}", e);
	cout << format("[{}] ", s);
}
cout << '\n';
\end{lstlisting}

输出为:

\begin{tcblisting}{commandshell={}}
[1234] [5678] [9101112]
\end{tcblisting}
\end{itemize}

\subsubsection{How it works…}

split算法本身比较简单,这个示例的神奇之处在于,模板可使其通用。

using声明中的派生类型,可以创建与输出容器一起使用的容器:

\begin{lstlisting}[style=styleCXX]
using SliceContainer = typename Oc::value_type;
\end{lstlisting}

这里别名一个SliceContainer类型,可以用它为切片创建容器:

\begin{lstlisting}[style=styleCXX]
SliceContainer dest_elm{};
\end{lstlisting}

这是一个临时容器,可将每个切片添加到输出容器中:

\begin{lstlisting}[style=styleCXX]
dest.push_back(dest_elm);
\end{lstlisting}








