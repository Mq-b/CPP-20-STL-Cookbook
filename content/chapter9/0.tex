并发性和并行性指的是在不同的执行线程中运行代码的能力。并发性是在后台运行线程的能力,并行性是在处理器的不同内核中同时运行线程的能力。

运行时库以及主机操作系统,将为给定硬件环境中的线程,在并发和并行执行模型之间进行选择。

在现代多任务操作系统中,main()函数已经代表了一个执行线程。当一个新线程启动时,可由现有的线程派生。

C++标准库中,std::thread类提供了线程执行的基本单元。其他类构建在线程之上,以提供锁、互斥和其他并发模式。根据系统架构的不同,执行线程可以在一个处理器上并发运行,也可以在不同的内核上并行运行。


\begin{itemize}
\item 
休眠一定的时间

\item 
std::thread——实现并发

\item 
std::async——实现并发

\item 
STL算法与执行策略

\item 
互斥锁和锁——安全地共享数据

\item 
std::atomic——共享标志和值

\item 
std::call\_once——初始化线程

\item 
std::condition\_variable——解决生产者-消费者问题

\item 
实现多个生产者和消费者
\end{itemize}