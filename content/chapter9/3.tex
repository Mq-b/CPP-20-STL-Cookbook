
线程是并发的单位,main()函数可以看作是执行的主线程。在操作系统上下文中,主线程与其他进程拥有的其他线程并发运行。

thread类是STL中并发性的根本,所有其他并发特性都建立在线程类的基础上。

本节中,我们将了解std::thread的基础知识,以及join()和detach()如何确定其执行上下文。

\subsubsection{How to do it…}

在这个示例中,我们创建了一些std::thread对象,并试验了它们的执行选项。

\begin{itemize}
\item 
从线程休眠函数开始,单位是毫秒:

\begin{lstlisting}[style=styleCXX]
void sleepms(const unsigned ms) {
	using std::chrono::milliseconds;
	std::this_thread::sleep_for(milliseconds(ms));
}
\end{lstlisting}

sleep\_for()函数接受一个duration对象,并在指定的时间内阻塞当前线程的执行。sleepms()函数作为一个方便的包装器,其接受一个unsigned值,表示休眠的毫秒数。

\item 
现在,需要线程的一个函数。这个函数休眠的毫秒数是可变的,基于一个整数参数:

\begin{lstlisting}[style=styleCXX]
void fthread(const int n) {
	cout << format("This is t{}\n", n);
	for(size_t i{}; i < 5; ++i) {
		sleepms(100 * n);
		cout << format("t{}: {}\n", n, i + 1);
	}
	cout << format("Finishing t{}\n", n);
}
\end{lstlisting}

fthread()调用sleepms() 5次,每次休眠100 * n毫秒。

\item 
可以在一个单独的线程中(比如主线程)运行std::thread:

\begin{lstlisting}[style=styleCXX]
int main() {
	thread t1(fthread, 1);
	cout << "end of main()\n";
}
\end{lstlisting}

可以编译,但运行时会有错误:

\begin{tcblisting}{commandshell={}}
terminate called without an active exception Aborted
\end{tcblisting}

(错误信息会有所不同。这是在Debian和GCC上的错误信息。)

问题在于,当线程对象超出作用域时,操作系统不知道该如何处理。我们必须指定调用者是否等待线程,或者它是否分离并独立运行。

\item 
使用join()来指示调用者将等待线程完成:

\begin{lstlisting}[style=styleCXX]
int main() {
	thread t1(fthread, 1);
	t1.join();
	cout << "end of main()\n";
}
\end{lstlisting}

输出为:

\begin{tcblisting}{commandshell={}}
This is t1
t1: 1
t1: 2
t1: 3
t1: 4
t1: 5
Finishing t1
end of main()
\end{tcblisting}

现在,main()会等待线程完成。

\item 
若使用detach(),那么main()不会等待,程序在线程运行之前就结束了:

\begin{lstlisting}[style=styleCXX]
thread t1(fthread, 1);
t1.detach();
\end{lstlisting}

输出为:

\begin{tcblisting}{commandshell={}}
end of main()
\end{tcblisting}

\item 
当线程分离后,需要给它一些运行的时间:

\begin{lstlisting}[style=styleCXX]
thread t1(fthread, 1);
t1.detach();
cout << "main() sleep 2 sec\n";
sleepms(2000);
\end{lstlisting}

输出为:

\begin{tcblisting}{commandshell={}}
main() sleep 2 sec
This is t1
t1: 1
t1: 2
t1: 3
t1: 4
t1: 5
Finishing t1
end of main()
\end{tcblisting}

\item 
分离第二个线程,看看会发生什么:

\begin{lstlisting}[style=styleCXX]
int main() {
	thread t1(fthread, 1);
	thread t2(fthread, 2);
	t1.detach();
	t2.detach();
	cout << "main() sleep 2 sec\n";
	sleepms(2000);
	cout << "end of main()\n";
}
\end{lstlisting}

输出为:

\begin{tcblisting}{commandshell={}}
main() sleep 2 sec
This is t1
This is t2
t1: 1
t2: 1
t1: 2
t1: 3
t2: 2
t1: 4
t1: 5
Finishing t1
t2: 3
t2: 4
t2: 5
Finishing t2
end of main()
\end{tcblisting}

因为fthread()函数使用它的形参作为sleepms()的乘法器,所以第二个线程比第一个线程运行得慢一些。我们可以看到计时器在输出中交错。

\item 
若使用join(),而不是detatch()来执行此操作,会得到类似的结果:

\begin{lstlisting}[style=styleCXX]
int main() {
	thread t1(fthread, 1);
	thread t2(fthread, 2);
	t1.join();
	t2.join();
	cout << "end of main()\n";
}
\end{lstlisting}

输出为:

\begin{tcblisting}{commandshell={}}
This is t1
This is t2
t1: 1
t2: 1
t1: 2
t1: 3
t2: 2
t1: 4
t1: 5
Finishing t1
t2: 3
t2: 4
t2: 5
Finishing t2
end of main()
\end{tcblisting}

因为join()等待线程完成,所以不再需要main()中2秒的sleepms()来等待线程完成。
\end{itemize}

\subsubsection{How it works…}

std::thread对象表示一个执行线程,对象和线程之间是一对一的关系。一个线程对象表示一个线程,一个线程由一个线程对象表示。线程对象不能复制或赋值,但可以移动。

其构造函数是这样:

\begin{lstlisting}[style=styleCXX]
explicit thread( Function&& f, Args&&… args );
\end{lstlisting}

线程是用一个函数指针和零个或多个参数构造的。函数会立即调用:

\begin{lstlisting}[style=styleCXX]
thread t1(fthread, 1);
\end{lstlisting}

这将创建对象t1,并立即以字面值1作为参数调用函数fthread(int)。

创建线程后,必须在线程上使用join()或detach():

\begin{lstlisting}[style=styleCXX]
t1.join();
\end{lstlisting}

join()方法阻塞调用线程的执行,直到t1线程完成:

\begin{lstlisting}[style=styleCXX]
t1.detach();
\end{lstlisting}

detach()方法允许调用线程独立于t1线程继续运行。

\subsubsection{There's more…}

C++20提供了std::jthread,其会自动在作用域的末尾汇入创建线程:

\begin{lstlisting}[style=styleCXX]
int main() {
	std::jthread t1(fthread, 1);
	cout "< "end of main("\n";
}
\end{lstlisting}

输出为:

\begin{tcblisting}{commandshell={}}
end of main()
This is t1
t1: 1
t1: 2
t1: 3
t1: 4
t1: 5
Finishing t1
\end{tcblisting}

t1线程可以独立执行,然后在其作用域的末尾会自动汇入main()线程。











