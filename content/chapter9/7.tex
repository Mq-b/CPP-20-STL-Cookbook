
std::atomic类封装了单个对象,并保证对其的操作是原子的。

写入原子对象由内存顺序策略控制,读取可能同时发生。通常用于同步不同线程之间的访问,std::atomic从模板类型定义原子类型。类型必须普通。

若类型占用连续的内存,没有用户定义的构造函数,也没有虚成员函数,那就是普通类型。所有的基本类型都是普通类型。

虽然可以构造普通类型,但std::atomic最常用于简单的基本类型,如bool、int、long、float和double。

\subsubsection{How to do it…}

这个示例使用一个简单的函数,循环计数器来演示共享原子对象。我们将生成一群这样的循环,作为共享原子值的线程:

\begin{itemize}
\item 
原子对象通常放置在全局命名空间中,可以让线程访问:

\begin{lstlisting}[style=styleCXX]
std::atomic<bool> ready{};
std::atomic<uint64_t> g_count{};
std::atomic_flag winner{};
\end{lstlisting}

ready对象是一个bool类型,当所有线程都准备好开始计数时,该类型设置为true。

g\_count对象是一个全局计数器,由每个线程进行递增。

winner对象是一个特殊的atomic\_flag类型,用于指示哪个线程先完成。

\item 
我们使用几个常量来控制线程数和每个线程的循环数:

\begin{lstlisting}[style=styleCXX]
constexpr int max_count{1000 * 1000};
constexpr int max_threads{100};
\end{lstlisting}

我将它设置为运行100个线程,并在每个线程中计算1,000,000次迭代。

\item 
countem()函数为每个线程生成,循环max\_count次,并在每次循环迭代时增加g\_count。这是我们使用原子值的地方:

\begin{lstlisting}[style=styleCXX]
void countem (int id) {
	while(!ready) std::this_thread::yield();
	for(int i{}; i < max_count; ++i) ++g_count;
	if(!winner.test_and_set()) {
		std::cout << format("thread {:02} won!\n",
		id);
	}
};
\end{lstlisting}

ready原子值用于同步线程。每个线程将调用yield(),直到就绪值设置为true。yield()函数的作用是:将执行传递给其他线程。

for循环的每次迭代都会增加g\_count原子值。最终值应该等于max\_count * max\_threads。

循环完成后,获胜者对象的test\_and\_set()方法用于报告获胜线程。test \_and\_set()是atomic\_flag类的一个方法,设置标志并返回设置之前的bool值。

\item 
我们以前使用过make\_comma()函数,可以显示一个带有数千个分隔符的数字:

\begin{lstlisting}[style=styleCXX]
string make_commas(const uint64_t& num) {
	string s{ std::to_string(num) };
	for(long l = s.length() - 3; l > 0; l -= 3) {
		s.insert(l, ",");
	}
	return s;
}
\end{lstlisting}

\item 
main()函数生成线程并显示结果:

\begin{lstlisting}[style=styleCXX]
int main() {
	vector<std::thread> swarm;
	cout << format("spawn {} threads\n", max_threads);
	for(int i{}; i < max_threads; ++i) {
		swarm.emplace_back(countem, i);
	}
	ready = true;
	for(auto& t : swarm) t.join();
	cout << format("global count: {}\n",
		make_commas(g_count));
	return 0;
}
\end{lstlisting}

\end{itemize}

在这里,创建一个vector<std::thread>对象来保存线程。

for循环中,使用emplace\_back()在vector中创建线程。

线程生成后,设置ready标志,以便线程可以开始循环。

输出为:

\begin{tcblisting}{commandshell={}}
spawn 100 threads
thread 67 won!
global count: 100,000,000
\end{tcblisting}

每次运行时,都会有不同的线程获胜。

\subsubsection{How it works…}

std::atomic类封装了一个对象来同步多个线程之间的访问。

封装的对象必须是普通类型,必须使用连续的内存,没有用户定义的构造函数,也没有虚成员函数。所有的基本类型都是普通类型。

atomic可以使用一个简单的结构体:

\begin{lstlisting}[style=styleCXX]
struct Trivial {
	int a;
	int b;
};
std::atomic<Trivial> triv1;
\end{lstlisting}

虽然这种用法是可能的,但并不实际。除了设置和检索复合值之外,并没有体现原子性的价值,最终需要一个互斥锁。并且,原子类最适合用于标量值。

\noindent
\textbf{特化}

原子类的专门化有以下几个不同的目的:

\begin{itemize}
\item 
指针和智能指针:std::atomic<U*>特化包括对原子指针算术操作的支持,包括用于加法的fetch\_add()和用于减法的fetch\_sub()。

\item 
浮点类型:当与浮点类型float、double和long double一起使用时,std::atomic包括对原子浮点算术操作的支持,包括用于加法的fetch\_add()和用于减法的fetch\_sub()。

\item 
整型类型:当与整型类型一起使用时,std::atomic提供了对其他原子操作的支持,包括fetch\_add()、fetch\_sub()、fetch\_and()、fetch\_or()和fetch\_xor()。

\end{itemize}


\noindent
\textbf{标准的别名}

STL为所有标准标量整型提供类型别名。所以在我们的代码中,不是这些声明:

\begin{lstlisting}[style=styleCXX]
std::atomic<bool> ready{};
std::atomic<uint64_t> g_count{};
\end{lstlisting}

我们可以用:

\begin{lstlisting}[style=styleCXX]
std::atomic_bool ready{};
std::atomic_uint64_t g_count{};
\end{lstlisting}

有46个标准的别名,每一个代表标准的整型:

\begin{table}[H]
\centering
\begin{tabular}{|l|l|}
\hline
atomic\_bool      & atomic\_uint64\_t        \\ \hline
atomic\_char      & atomic\_int\_least8\_t   \\ \hline
atomic\_schar     & atomic\_uint\_least8\_t  \\ \hline
atomic\_uchar     & atomic\_int\_least16\_t  \\ \hline
atomic\_short     & atomic\_uint\_least16\_t \\ \hline
atomic\_ushort    & atomic\_int\_least32\_t  \\ \hline
atomic\_int       & atomic\_uint\_least32\_t \\ \hline
atomic\_uint      & atomic\_int\_least64\_t  \\ \hline
atomic\_long      & atomic\_uint\_least64\_t \\ \hline
atomic\_ulong     & atomic\_int\_fast8\_t    \\ \hline
atomic\_llong     & atomic\_uint\_fast8\_t   \\ \hline
atomic\_ullong    & atomic\_int\_fast16\_t   \\ \hline
atomic\_char8\_t  & atomic\_uint\_fast16\_t  \\ \hline
atomic\_char16\_t & atomic\_int\_fast32\_t   \\ \hline
atomic\_char32\_t & atomic\_uint\_fast32\_t  \\ \hline
atomic\_wchar\_t  & atomic\_int\_fast64\_t   \\ \hline
atomic\_int8\_t   & atomic\_uint\_fast64\_t  \\ \hline
atomic\_uint8\_t  & atomic\_intptr\_t        \\ \hline
atomic\_int16\_t  & atomic\_uintptr\_t       \\ \hline
atomic\_uint16\_t & atomic\_size\_t          \\ \hline
atomic\_int32\_t  & atomic\_ptrdiff\_t       \\ \hline
atomic\_uint32\_t & atomic\_intmax\_t        \\ \hline
atomic\_int64\_t  & atomice\_uintmax\_t      \\ \hline
\end{tabular}
\end{table}

\noindent
\textbf{无锁版本}

大多数现代体系结构为执行原子操作提供了原子CPU指令,原子指令应该在硬件支持的情况下使用硬件支持。某些硬件可能不支持某些原子类型,std::atomic可以使用互斥来确保那些特化的线程安全操作,导致线程在等待其他线程完成操作时阻塞。使用硬件支持的特化因为它们不需要互斥锁,所以称为无锁式方法。

is\_lock\_free()方法检查特化是否无锁:

\begin{lstlisting}[style=styleCXX]
cout << format("is g_count lock-free? {}\n",
	g_count.is_lock_free());
\end{lstlisting}

输出为:

\begin{tcblisting}{commandshell={}}
is g_count lock-free? true
\end{tcblisting}

这个结果对于大多数现代架构都可以正常处理。

有一些保证std::atomic的无锁版本可用。这些特化保证为每个目的使用最有效的硬件原子操作:

\begin{itemize}
\item 
std::atomic\_signed\_lock\_free 有符号整型最有效的无锁特化的别名。

\item 
std::atomic\_unsigned\_lock\_free 是无符号整型最有效的无锁特化的别名。

\item 
The std::atomic\_flag 类提供了一个无锁原子布尔类型。
\end{itemize}

\begin{tcolorbox}[colback=webgreen!5!white,colframe=webgreen!75!black,title=重要的Note]
当前Windows系统不支持64位硬件整数,即使在64位系统上也是如此。当在我的实验室中的一个系统上测试这段代码时,将std::atomic<uint64\_t>替换为std::atomic\_unsigned\_lock\_free,性能提高了3倍。在64位Linux和Mac系统上,性能没什么变化。
\end{tcolorbox}

\subsubsection{There's more…}

当多个线程同时读写变量时,线程可能以不同于写入变量的顺序观察变化。memory\_order指定内存访问如何围绕原子操作排序。

std::atomic提供了访问和更改其托管值的方法。与关联操作符不同,这些访问方法为要指定的memory\_order提供参数。例如:

\begin{lstlisting}[style=styleCXX]
g_count.fetch_add(1, std::memory_order_seq_cst);
\end{lstlisting}

这种情况下,memory\_order\_seq\_cst指定顺序一致的排序。因此,对fetch\_add()的调用将按顺序一致地将g\_count的值加1。

可能的memory\_order常量是:

\begin{itemize}
\item 
memory\_order\_relaxed: 这是一个松散的操作。没有同步或排序约束,只有操作的原子性得到保证。

\item 
memory\_order\_consume: 这是一个消费型操作。在此加载之前,不能对依赖于该值的当前线程中的访问进行重排序。这只会影响编译器优化。

\item 
memory\_order\_acquire: 这是一个获取型操作。在此加载之前,访问不能重新排序。

\item 
memory\_order\_release: 这是一个存储型操作。当前线程中的访问不能在此存储之后重新排序。

\item 
memory\_order\_acq\_rel: 这既是获取型,也是释放型。当前线程中的访问不能在此存储之前或之后重新排序。

\item 
memory\_order\_seq\_cst: 这是顺序一致的排序,根据上下文获取或释放。加载执行获取,存储执行释放,读/写/修改同时执行。所有线程以相同的顺序观察所有修改。

若没有指定memory\_order,则memory\_order\_seq\_cst为默认值。
\end{itemize}





