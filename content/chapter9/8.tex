
可能需要在许多线程中运行相同的代码,但只能初始化该代码一次。

一种解决方案是在运行线程之前调用初始化代码。这种方法可以工作,但有一些缺点。通过分离初始化,可以在不必要的时候调用它,也可以在必要的时候忽略它。

std::call\_once函数提供了一个更健壮的解决方案。call\_once在<mutex>头文件中声明。

\subsubsection{How to do it…}

本节中,我们使用print函数进行初始化,所以可以清楚地看到函数什么时候调用:

\begin{itemize}
\item 
我们可以使用一个常量来表示要生成的线程数:

\begin{lstlisting}[style=styleCXX]
constexpr size_t max_threads{ 25 };
\end{lstlisting}

还需要一个std::once\_flag来同步std::call\_once函数:

\begin{lstlisting}[style=styleCXX]
std::once_flag init_flag;
\end{lstlisting}

\item 
初始化函数只是打印一个字符串,让我们知道其调用了:

\begin{lstlisting}[style=styleCXX]
void do_init(size_t id) {
	cout << format("do_init ({}): ", id);
}
\end{lstlisting}

\item 
我们的工作函数do\_print()使用std::call\_once来调用初始化函数,然后打印其id:

\begin{lstlisting}[style=styleCXX]
void do_print(size_t id) {
	std::call_once(init_flag, do_init, id);
	cout << format("{} ", id);
}
\end{lstlisting}

\item 
在main()中,使用列表容器来管理线程对象:

\begin{lstlisting}[style=styleCXX]
int main() {
	list<thread> spawn;
	for (size_t id{}; id < max_threads; ++id) {
		spawn.emplace_back(do_print, id);
	}
	for (auto& t : spawn) t.join();
	cout << '\n';
}
\end{lstlisting}

输出显示初始化首先发生,并且只发生一次:

\begin{tcblisting}{commandshell={}}
do_init (8): 12 0 2 1 9 6 13 10 11 5 16 3 4 17 7 15 8 14
18 19 20 21 22 23 24
\end{tcblisting}

注意,最终调用初始化函数的并不总是第一个衍生线程(0),但总是第一个调用。若重复运行这个命令,会看到线程0得到初始化,但不是每次都这样。在内核较少的系统中,初始化时更经常看到线程0。
\end{itemize}

\subsubsection{How it works…}

std::call\_once是一个模板函数,它接受一个标志、一个可调用对象(函数或函子)和一个参数形参包:

\begin{lstlisting}[style=styleCXX]
template<class Callable, class... Args>
void call_once(once_flag& flag, Callable&& f, Args&&... args);
\end{lstlisting}

可调用的f只调用了一次。即使call\_once在多个线程中并发使用,f仍然只调用了一次。

这需要一个std::once\_flag对象进行协调,once\_flag构造函数将其状态设置为指示可调用对象尚未调用。

当call\_once调用可调用对象时,对同一个once\_flag的其他调用都将阻塞,直到可调用对象返回。可调用对象返回后,可以设置once\_flag,从而使后面的call\_once直接返回,而不在调用f。



















