
The <thread> header provides two functions for putting a thread to sleep, sleep\_for() and sleep\_until(). Both functions are in the std::this\_thread namespace.

This recipe explores the use of these functions, as we will be using them later in this chapter.

\subsubsection{How to do it…}

Let's look at how to use the sleep\_for() and sleep\_until() functions:

\begin{itemize}
\item 
The sleep-related functions are in the std::this\_thread namespace. Because it has just a few symbols, we'll go ahead and issue using directives for std::this\_thread and std::chrono\_literals:

\begin{lstlisting}[style=styleCXX]
using namespace std::this_thread;
using namespace std::chrono_literals;
\end{lstlisting}

The chrono\_literals namespace has symbols for representing durations, such as 1s for one second, or 100ms for 100 milliseconds.

\item 
In main(), we'll mark a point in time with steady\_clock::now(), so we can time our test:

\begin{lstlisting}[style=styleCXX]
int main() {
	auto t1 = steady_clock::now();
	cout << "sleep for 1.3 seconds\n";
	sleep_for(1s + 300ms);
	cout << "sleep for 2 seconds\n";
	sleep_until(steady_clock::now() + 2s);
	duration<double> dur1 = steady_clock::now() - t1;
	cout << format("total duration: {:.5}s\n",
		dur1.count());
}
\end{lstlisting}

The sleep\_for() function takes a duration object to specify the amount of time to sleep. The argument (1s + 300ms) uses chrono\_literal operators to return a duration object representing 1.3 seconds.

The sleep\_until() function takes a time\_point object to specify a specific time to resume from sleep. In this case, the chrono\_literal operators are used to modify the time\_point object returned from steady\_clock::now().

This is our output:

\begin{tcblisting}{commandshell={}}
sleep for 1.3 seconds
sleep for 2 seconds
total duration: 3.3005s
\end{tcblisting}

\end{itemize}

\subsubsection{How it works…}

The sleep\_for(duration) and sleep\_until(time\_point) functions suspend execution of the current thread for the specified duration, or until the time\_point is reached. The sleep\_for() function will use the steady\_clock implementation, if supported.

Otherwise, the duration may be subject to time adjustments. Both functions may block for longer due to scheduling or resource delays.

\subsubsection{There's more…}

Some systems support a POSIX function, sleep(), which suspends execution for the number of seconds specified:

\begin{lstlisting}[style=styleCXX]
unsigned int sleep(unsigned int seconds);
\end{lstlisting}

The sleep() function is part of the POSIX standard and is not part of the C++ standard.







