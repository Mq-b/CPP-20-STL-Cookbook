
三向比较操作符(<=>),通常称为“宇宙飞船”操作符,因为它看起来像一个飞碟,是C++20添加的新功能。您可能想知道,是因为现有的六个比较操作符有什么问题吗?并不是,可以继续使用它们。宇宙飞船的目的是提供统一的比较操作符。

常见的双向比较操作符根据比较的结果返回两种状态之一,true或false。例如:

\begin{lstlisting}[style=styleCXX]
const int a = 7;
const int b = 42;
static_assert(a < b);
\end{lstlisting}

a < b表达式使用小于比较操作符(<)来测试a是否小于b。若条件满足,比较操作符返回true,不满足则返回false。这种情况下,返回true,因为7小于42。

这三种方式的比较方式有所不同,会返回三种状态之一。若操作数相等,飞船操作符将返回一个等于0的值,若左操作数小于右操作数则返回负数,若左操作数大于右操作数则返回正数。

\begin{lstlisting}[style=styleCXX]
const int a = 7;
const int b = 42;
static_assert((a <=> b) < 0);
\end{lstlisting}

返回值不是整数,是<compare>头中的对象,与0进行比较。

若操作数为整型,则操作符从<compare>库返回strong\_ordered对象。

\begin{lstlisting}[style=styleCXX]
strong_ordering::equal // operands are equal
strong_ordering::less // lhs is less than rhs
strong_ordering::greater // lhs is greater than rhs
\end{lstlisting}

若操作数为浮点类型,则操作符返回partial\_ordered对象:

\begin{lstlisting}[style=styleCXX]
partial_ordering::equivalent // operands are equivelant
partial_ordering::less // lhs is less than rhs
partial_ordering::greater // lhs is greater than rhs
partial_ordering::unordered // if an operand is unordered
\end{lstlisting}

这些对象通常与0进行比较,从而区别于常规比较操作符(例如,(a <=> b) < 0)。这使得三向比较的结果比常规比较更精确。

若这些看起来有点复杂,没关系。对于大多数应用程序,可能不会直接使用太空船操作符,其主要作用在于为对象提供统一的比较运算符。

\subsubsection{How to do it…}

来看一个简单的类,封装了一个整数,并提供了比较运算符:

\begin{lstlisting}[style=styleCXX]
struct Num {
	int a;
	constexpr bool operator==(const Num& rhs) const
	{ return a == rhs.a; }
	constexpr bool operator!=(const Num& rhs) const
	{ return !(a == rhs.a); }
	constexpr bool operator<(const Num& rhs) const
	{ return a < rhs.a; }
	constexpr bool operator>(const Num& rhs) const
	{ return rhs.a < a; }
	constexpr bool operator<=(const Num& rhs) const
	{ return !(rhs.a < a); }
	constexpr bool operator>=(const Num& rhs) const
	{ return !(a < rhs.a); }
};
\end{lstlisting}

这样的比较操作符重载并不少见。若比较操作符并非两侧对象的友元函数,情况会变得更加复杂。

有了新的宇宙飞船操作符,就可以通过一次重载来完成:

\begin{lstlisting}[style=styleCXX]
#include <compare>
struct Num {
	int a;
	constexpr Num(int a) : a{a} {}
	auto operator<=>(const Num&) const = default;
};
\end{lstlisting}

注意,我们需要为三向操作符返回类型包含<compare>头文件。现在可以声明一些变量,并通过比较来进行测试:

\begin{lstlisting}[style=styleCXX]
constexpr Num a{ 7 };
constexpr Num b{ 7 };
constexpr Num c{ 42 };

int main() {
	static_assert(a < c);
	static_assert(c > a);
	static_assert(a == b);
	static_assert(a <= b);
	static_assert(a <= c);
	static_assert(c >= a);
	static_assert(a != c);
	puts("done.");
}
\end{lstlisting}

对于每一次比较,编译器都会自动使用<=>操作符。

因为默认的<=>操作符已经是constexpr安全的,所以不需要在成员函数中这样声明。

\subsubsection{How it works…}

<=>重载利用了C++20的一个新概念,即重写的表达式。在重载解析期间,编译器根据一组规则重写表达式。例如,a < b,编译器会将其重写为(a <=> b < 0),以便它与成员操作符可以一起工作。编译器将重写<=>操作符的每个相关比较表达式,其中不包含具体的操作符。

事实上,我们不再需要非成员函数来处理与左侧兼容类型的比较。编译器将合成一个与成员操作符一起工作的表达式。例如,42 > a,编译器将合成一个操作符颠倒的表达式(a <=> 42 < 0),以便与我们的成员操作符一起工作。

\begin{tcolorbox}[colback=webgreen!5!white,colframe=webgreen!75!black,title=Note]
<=>的优先级高于其他比较运算符,因此它总是先求值。所有比较运算符都从左到右计算。
\end{tcolorbox}

\subsubsection{There's more…}

默认操作符可以很好地处理各种类,包括多个不同类型成员的类:

\begin{lstlisting}[style=styleCXX]
struct Nums {
	int i;
	char c;
	float f;
	double d;
	auto operator<=>(const Nums&) const = default;
};
\end{lstlisting}

但若有一个更复杂的类型呢?下面是一个简单分数类:

\begin{lstlisting}[style=styleCXX]
struct Frac {
	long n;
	long d;
	constexpr Frac(int a, int b) : n{a}, d{b} {}
	constexpr double dbl() const {
		return static_cast<double>(n) /
		static_cast<double>(d);
	}
	constexpr auto operator<=>(const Frac& rhs) const {
		return dbl() <=> rhs.dbl();
	};
	constexpr auto operator==(const Frac& rhs) const {
		return dbl() <=> rhs.dbl() == 0;
	};
};
\end{lstlisting}

在本例中,需要定义运算符<=>的重载,因为数据成员不是独立的标量。重载很简单,而且效果很好。

注意,还需要一个运算符==重载。因为表达式重写规则不会使用自定义操作符<=>重载重写==和!=,所以需要定义操作符==,从而编译器会根据需要重写!=表达式。

现在可以定义一些对象:

\begin{lstlisting}[style=styleCXX]
constexpr Frac a(10,15); // compares equal with 2/3
constexpr Frac b(2,3);
constexpr Frac c(5,3);
\end{lstlisting}

可以用正常的比较运算符来测试,会如预期的一样:

\begin{lstlisting}[style=styleCXX]
int main() {
	static_assert(a < c);
	static_assert(c > a);
	static_assert(a == b);
	static_assert(a <= b);
	static_assert(a <= c);
	static_assert(c >= a);
	static_assert(a != c);
}
\end{lstlisting}

宇宙飞船操作符的强大之处在于,能够简化类中的比较重载。与单独重载每个操作符相比,其简单并且高效。
















