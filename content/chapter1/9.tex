
范围库是C++20中添加的重要特性,可为过滤和处理容器提供了一种新的范例。范围为更有效和可读的代码提供了清晰和直观的构建块。

先来定义几个术语:

\begin{itemize}
\item 
“范围”是一个可以迭代的对象的集合,支持begin()和end()迭代器的结构都是范围。这包括大多数STL容器。

\item 
“视图”是转换另一个基础范围的范围。视图是惰性的,只在范围迭代时操作。视图从底层范围返回数据,不拥有任何数据。视图的运行时间复杂度是O(1)。

\item 
视图适配器是一个对象,可接受一个范围,并返回一个视图对象。视图适配器可以使用|操作符连接到其他视图适配器。
\end{itemize}

\begin{tcolorbox}[colback=webgreen!5!white,colframe=webgreen!75!black,title=Note]
<ranges>中定义了std::ranges和std::ranges::view命名空间。这貌似有些复杂,标准包含了std::ranges::view的别名,即std::view。我还是觉得那很麻烦。对于这个示例,将使用以下别名,我觉得这种方式更优雅:

\begin{lstlisting}[style=styleCXX]
namespace ranges = std::ranges; // save the fingers!
namespace views = std::ranges::views;
\end{lstlisting}

这适用于本节中的代码。
\end{tcolorbox}

\subsubsection{How to do it…}

范围和视图类在<ranges>头文件中,看看如何使用它们:

\begin{itemize}
\item 
将视图(View)应用于范围(Range),如下所示:

\begin{lstlisting}[style=styleCXX]
const vector<int> nums{ 1, 2, 3, 4, 5, 6, 7, 8, 9, 10 };
auto result = ranges::take_view(nums, 5);
for (auto v: result) cout << v << " ";
\end{lstlisting}

输出为:

\begin{tcblisting}{commandshell={}}
1 2 3 4 5
\end{tcblisting}

ranges::take\_view(range, n):返回前n个元素的视图。

也可以使用视图适配器版本的take\_view():

\begin{lstlisting}[style=styleCXX]
auto result = nums | views::take(5);
for (auto v: result) cout << v << " ";
\end{lstlisting}

输出为:

\begin{tcblisting}{commandshell={}}
1 2 3 4 5
\end{tcblisting}

视图适配器位于std::ranges::views命名空间中。视图适配器从|操作符的左侧获取范围操作数,很像<{}<操作符的iostreams用法。|操作赋会从左向右求值。

\item 
因为视图适配器可迭代,所以也有资格作为范围。这使得它们可以连续使用:

\begin{lstlisting}[style=styleCXX]
const vector<int> nums{ 1, 2, 3, 4, 5, 6, 7, 8, 9, 10 };
auto result = nums | views::take(5) | views::reverse;
\end{lstlisting}

输出为:

\begin{tcblisting}{commandshell={}}
5 4 3 2 1
\end{tcblisting}

\item 
filter()视图使用了一个谓词函数:

\begin{lstlisting}[style=styleCXX]
auto result = nums |
	views::filter([](int i){ return 0 == i % 2; });
\end{lstlisting}

输出为:

\begin{tcblisting}{commandshell={}}
2 4 6 8 10
\end{tcblisting}

\item 
transform()视图使用了一个转换函数:

\begin{lstlisting}[style=styleCXX]
auto result = nums |
	views::transform([](int i){ return i * i; });
\end{lstlisting}

输出为:

\begin{tcblisting}{commandshell={}}
1 4 9 16 25 36 49 64 81 100
\end{tcblisting}

\item 
视图和适配器可适用于任何范围:

\begin{lstlisting}[style=styleCXX]
cosnt vector<string>
words{ "one", "two", "three", "four", "five" };
auto result = words | views::reverse;
\end{lstlisting}

输出为:

\begin{tcblisting}{commandshell={}}
five four three two one
\end{tcblisting}

\item 
范围库还包括一些工厂函数。iota工厂将生成一系列递增的值:

\begin{lstlisting}[style=styleCXX]
auto rnums = views::iota(1, 10);
\end{lstlisting}

输出为:

\begin{tcblisting}{commandshell={}}
1 2 3 4 5 6 7 8 9
\end{tcblisting}

\item 
iota(value, bound)函数的作用是:生成一个从value开始,到bound之前结束的序列。若省略了bound,序列则为无穷大:

\begin{lstlisting}[style=styleCXX]
auto rnums = views::iota(1) | views::take(200);
\end{lstlisting}

输出为:

\begin{tcblisting}{commandshell={}}
1 2 3 4 5 6 7 8 9 10 11 12 […] 196 197 198 199 200
\end{tcblisting}
\end{itemize}

范围、视图和视图适配器灵活好用。让我们更深入地了解一下。

\subsubsection{How it works…}

为了满足范围的基本要求,对象必须至少有两个迭代器begin()和end(),其中end()迭代器是一个哨兵,用于确定Range的端点。大多数STL容器都符合范围的要求,包括string、vector、array、map等。不过,容器适配器除外,如stack和queue,因为它们没有起始迭代器和结束迭代器。

视图是一个对象,操作一个范围并返回一个修改后的范围。视图为惰性操作的,不包含自己的数据。不保留底层数据的副本,只是根据需要返回底层元素的迭代器。来看个代码段:

\begin{lstlisting}[style=styleCXX]
vector<int> vi { 0, 1, 2, 3, 4, 5 };
ranges::take_view tv{vi, 2};
for(int i : tv) {
	cout << i << " ";
}
cout << "\n";
\end{lstlisting}

输出为:

\begin{tcblisting}{commandshell={}}
0 1
\end{tcblisting}

本例中,take\_view对象接受两个参数,一个范围(在本例中是vector<int>对象)和一个计数,结果是一个包含vector中第一个count对象的视图。在for循环的迭代过程求值时,take\_view对象会根据需要返回指向vector对象元素的迭代器。

在此过程中不修改vector对象。

范围命名空间中的许多视图在视图命名空间中都有相应的范围适配器,这些适配器可以使用按位或(|)操作符作为管道:

\begin{lstlisting}[style=styleCXX]
vector<int> vi { 0, 1, 2, 3, 4, 5 };
auto tview = vi | views::take(2);
for(int i : tview) {
	cout << i << " ";
}
cout << "\n";
\end{lstlisting}

输出为:

\begin{tcblisting}{commandshell={}}
0 1
\end{tcblisting}

如预期的那样,|操作符从左到右求值。因为范围适配器的结果是另一个范围,所以这些适配器表达式可以链接起来:

\begin{lstlisting}[style=styleCXX]
vector<int> vi { 0, 1, 2, 3, 4, 5, 6, 7, 8, 9 };
auto tview = vi | views::reverse | views::take(5);
for(int i : tview) {
	cout << i << " ";
}
cout << "\n";
\end{lstlisting}

输出为:

\begin{tcblisting}{commandshell={}}
9 8 7 6 5
\end{tcblisting}

标准库包括一个过滤视图,用于定义简单的过滤器:

\begin{lstlisting}[style=styleCXX]
vector<int> vi { 0, 1, 2, 3, 4, 5, 6, 7, 8, 9 };
auto even = [](long i) { return 0 == i % 2; };
auto tview = vi | views::filter(even);
\end{lstlisting}

输出为:

\begin{tcblisting}{commandshell={}}
0 2 4 6 8
\end{tcblisting}

还包括一个transform视图,与transform函数一起用于转换结果:

\begin{lstlisting}[style=styleCXX]
vector<int> vi { 0, 1, 2, 3, 4, 5, 6, 7, 8, 9 };
auto even = [](int i) { return 0 == i % 2; };
auto x2 = [](auto i) { return i * 2; };
auto tview = vi | views::filter(even) | views::transform(x2);
\end{lstlisting}

输出为:

\begin{tcblisting}{commandshell={}}
0 4 8 12 16
\end{tcblisting}

库中有相当多有用的视图和视图适配器。可访问(\url{https://j.bw.org/ranges})获取完整的列表。

\subsubsection{There's more…}

从C++20开始,<algorithm>头文件中的大多数算法都会基于范围。这些版本在<algorithm>头文件中,但是在std::ranges命名空间中,这将它们与传统算法区别开来。

所以,无需再调用带有两个迭代器的算法:

\begin{lstlisting}[style=styleCXX]
sort(v.begin(), v.end());
\end{lstlisting}

现在可以用范围来调用:

\begin{lstlisting}[style=styleCXX]
ranges::sort(v);
\end{lstlisting}

这当然更方便,但它真正意义在哪里呢?

回想一下,要对vector的一部分排序时的情况。可以用老方法来做:

\begin{lstlisting}[style=styleCXX]
sort(v.begin() + 5, v.end());
\end{lstlisting}

这将对vector的前5个元素进行排序。范围版本中,可以使用视图来跳过前5个元素:

\begin{lstlisting}[style=styleCXX]
ranges::sort(views::drop(v, 5));
\end{lstlisting}

甚至可以组合视图:

\begin{lstlisting}[style=styleCXX]
ranges::sort(views::drop(views::reverse(v), 5));
\end{lstlisting}

也可以使用范围适配器作为ranges::sort的参数:

\begin{lstlisting}[style=styleCXX]
ranges::sort(v | views::reverse | views::drop(5));
\end{lstlisting}

用传统的排序算法和vector迭代器来完成:

\begin{lstlisting}[style=styleCXX]
ranges::sort(v.rbegin() + 5, v.rend());
\end{lstlisting}

虽然这更简单,也很好理解,但我觉得范围适配器版本的会更直观。

可以在cppreference(\url{https://j.bw.org/algoranges})上找到限制使用范围的完整算法列表。

这个示例中,我们只触及了范围和视图的冰山一角。这个特性是许多不同团队十多年来工作的结晶,我希望它能从根本上改变我们使用STL容器的方式。
























