
模板对于编写适用于不同类型的代码非常有用。例如,此函数将适用于任何数字类型:

\begin{lstlisting}[style=styleCXX]
template <typename T>
T arg42(const T & arg) {
	return arg + 42;
}
\end{lstlisting}

当尝试用非数字类型调用它时,会发生什么呢?

\begin{lstlisting}[style=styleCXX]
const char * n = "7";
cout << "result is " << arg42(n) << "\n";
\end{lstlisting}

输出为:

\begin{tcblisting}{commandshell={}}
Result is ion
\end{tcblisting}

这样编译和运行没有错误,但结果无法预测。该调用非常危险,很容易造成崩溃或成为漏洞。我更希望编译器生成一个错误消息,这样就可以提前修复代码。

有了概念后,就可以这样写:

\begin{lstlisting}[style=styleCXX]
template <typename T>
requires Numeric<T>
T arg42(const T & arg) {
	return arg + 42;
}
\end{lstlisting}

require关键字是C++20的新特性,将约束应用于模板。Numeric是一个只接受整数和浮点类型的概念的名称。现在,当用非数字参数编译这段代码时,就会得到编译错误:

\begin{tcblisting}{commandshell={}}
error: 'arg42': no matching overloaded function found
error: 'arg42': the associated constraints are not satisfied
\end{tcblisting}

这样的错误消息比大多数编译器错误有用得多。

仔细看看如何在代码中使用概念和约束。

\subsubsection{How to do it…}

概念只是一个命名的约束。上面的Numeric概念是这样的:

\begin{lstlisting}[style=styleCXX]
#include <concepts>
template <typename T>
concept Numeric = integral<T> || floating_point<T>;
\end{lstlisting}

此概念需要类型T,满足std::integral或std::float\_point预定义概念。这些概念包含在<concepts>头文件中。

概念和约束可以用在类模板、函数模板或变量模板中。我们已经看到了一个约束函数模板,这里是一个简单的约束类模板示例:

\begin{lstlisting}[style=styleCXX]
template<typename T>
requires Numeric<T>
struct Num {
	T n;
	Num(T n) : n{n} {}
};
\end{lstlisting}

下面是一个简单的变量模板示例:

\begin{lstlisting}[style=styleCXX]
template<typename T>
requires floating_point<T>
T pi{3.1415926535897932385L};
\end{lstlisting}

可以在任何模板上使用概念和约束。简单起见,我们将在这些示例中使用函数模板。

\begin{itemize}
\item 
约束可以使用概念或类型特征来评估类型的特征。可以使用<type\_traits>头文件中找到的任何类型特征,只要可以返回bool类型。

例如:

\begin{lstlisting}[style=styleCXX]
template<typename T>
requires is_integral<T>::value // value is bool
constexpr double avg(vector<T> const& vec) {
	double sum{ accumulate(vec.begin(), vec.end(),
		0.0)
	};
	return sum / vec.size();
}
\end{lstlisting}

\item 
require关键字是C++20中新出现的,为模板参数引入了一个约束。本例中,约束表达式根据类型特征is\_integral测试模板参数。

\item 
可以使用<type\_traits>头文件中预定义的特性,或者自定义的特性,就像模板变量一样。为了在约束中使用,该变量必须返回constexpr bool。例如:

\begin{lstlisting}[style=styleCXX]
template<typename T>
constexpr bool is_gt_byte{ sizeof(T) > 1 };
\end{lstlisting}

这定义了一个名为is\_gt\_byte的类型特征,该特性使用sizeof操作符来测试类型T是否大于1字节。

\item 
概念只是一组命名的约束。例如:

\begin{lstlisting}[style=styleCXX]
template<typename T>
concept Numeric = is_gt_byte<T> &&
	(integral<T> || floating_point<T>);
\end{lstlisting}

这定义了一个名为Numeric的概念,使用is\_gt\_byte约束,以及<concepts>头文件中的floating\_point和integral概念。可以用它来约束模板,使其只接受大于1字节的数字类型。

\begin{lstlisting}[style=styleCXX]
template<Numeric T>
T arg42(const T & arg) {
	return arg + 42;
}
\end{lstlisting}

有读者会注意到,我在模板声明中应用了约束,而不是在require表达式中的单独一行中。有几种方法可以使用概念,让我们看看它是如何工作的。
\end{itemize}

\subsubsection{How it works…}

使用概念或约束有几种不同的方法:

\begin{itemize}
\item 
可以用require关键字定义一个概念或约束:

\begin{lstlisting}[style=styleCXX]
template<typename T>
requires Numeric<T>
T arg42(const T & arg) {
	return arg + 42;
}
\end{lstlisting}

\item 
可以在模板声明中使用概念:

\begin{lstlisting}[style=styleCXX]
template<Numeric T>
T arg42(const T & arg) {
	return arg + 42;
}
\end{lstlisting}

\item 
可以在函数签名中使用require关键字:

\begin{lstlisting}[style=styleCXX]
template<typename T>
T arg42(const T & arg) requires Numeric<T> {
	return arg + 42;
}
\end{lstlisting}

\item 
可以在参数列表中使用概念来简化函数模板:

\begin{lstlisting}[style=styleCXX]
auto arg42(Numeric auto & arg) {
	return arg + 42;
}
\end{lstlisting}
\end{itemize}

如何进行选择,可能就只是偏好问题。

\subsubsection{There's more…}

该标准使用术语连接、分离和原子来描述可用于构造约束的表达式类型。我们来定义一下这些术语。

可以使用\&\&和||操作符组合概念和约束。这些组合分别称为连接词和析取词,可以把它们看成逻辑的AND和OR。

约束连接符是通过使用带有两个约束的\&\&操作符形成的:

\begin{lstlisting}[style=styleCXX]
Template <typename T>
concept Integral_s = Integral<T> && is_signed<T>::value;
\end{lstlisting}

只有当\&\&运算符的两边都满足时,连接符才满足。从左到右求值。连接的操作数可短路,若左边的约束不满足,右边的约束就不会求值。

约束析取通过使用||操作符和两个约束形成:

\begin{lstlisting}[style=styleCXX]
Template <typename T>
concept Numeric = integral<T> || floating_point<T>;
\end{lstlisting}

若||运算符的任意一边满足,则析取条件满足。从左到右求值。连接的操作数是短路的,若左边的约束得到满足,右边的约束就不会求值。

原子约束是返回bool类型的表达式,不能进一步分解。换句话说,不是一个连接或析取。

\begin{lstlisting}[style=styleCXX]
template<typename T>
concept is_gt_byte = sizeof(T) > 1;
\end{lstlisting}

也可以在原子约束中使用逻辑!(NOT)操作符:

\begin{lstlisting}[style=styleCXX]
template<typename T>
concept is_byte = !is_gt_byte<T>;
\end{lstlisting}

果然,!运算符将bool表达式的值颠倒到!的右侧。

当然,可以将所有这些表达式类型组合成一个更大的表达式。可以在下面的例子中看到这些约束表达式的例子:

\begin{lstlisting}[style=styleCXX]
template<typename T>
concept Numeric = is_gt_byte<T> &&
(integral<T> || floating_point<T>);
\end{lstlisting}

来分析一下。

子表达式(integral<T> || floating\_point<T>)是一个析取。子表达式is\_gt\_byte<T> \&\&(…)是一个连接。每个子表达式integral<T>,float \_point<T>和is\_gt\_byte<T>,都是原子表达式。

这些区别主要是为了描述的目的。虽然了解细节是很好的习惯,但在编写代码时,可将它们视为逻辑||,\&\&和!操作符。

概念和约束是C++标准的一个很受欢迎的补充,非常期待在未来的项目中使用到它们。







