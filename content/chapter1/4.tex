
比较不同类型的整数,可能并不总是生成预期的结果。例如:

\begin{lstlisting}[style=styleCXX]
int x{ -3 };
unsigned y{ 7 };
if(x < y) puts("true");
else puts("false");
\end{lstlisting}

可能希望此代码输出true,-3通常小于7,但代码也会输出false。

问题是x是有符号的y是无符号的,标准化的行为是将有符号类型转换为无符号类型进行比较。这似乎违反直觉,不能可靠地将无符号值转换为相同大小的有符号值,因为有符号整数使用2的补数表示(使用最高位作为符号)。给定相同大小的整数,最大有符号值是无符号值的一半。使用这个例子,若整数是32位,-3(有符号)变成FFFF FFFD(十六进制),或者4,294,967,293(无符号十进制),所以不小于7。

尝试比较有符号整数值和无符号整数值时,一些编译器可能会发出警告,但大多数编译器不会。

C++20标准在<utility>头文件中包含了一组整数安全的比较函数。

\subsubsection{How to do it…}

新的整数比较函数可以在<utility>头文件找到。每个函数都有两个参数,分别对应于运算符的左边和右边。

\begin{lstlisting}[style=styleCXX]
#include <utility>
int main() {
	int x{ -3 };
	unsigned y{ 7 };
	if(cmp_less(x, y)) puts("true");
	else puts("false");
}
\end{lstlisting}

cmp\_less()函数给出了我们所期望的结果。-3小于7,程序现在输出true。

<utility>头文件提供了完整的整数比较函数。假设有x和y值,可以得到这些比较的结果:

\begin{lstlisting}[style=styleCXX]
cmp_equal(x, y) // x == y is false
cmp_not_equal(x, y) // x != y is true
cmp_less(x, y) // x < y is true
cmp_less_equal(x, y) // x <= y is true
cmp_greater(x, y) // x > y is false
cmp_greater_equal(x, y) // x >= y is false
\end{lstlisting}

\subsubsection{How it works…}

下面是C++20标准中cmp\_less()函数的示例实现,可以了解它是如何工作的:

\begin{lstlisting}[style=styleCXX]
template< class T, class U >
constexpr bool cmp_less( T t, U u ) noexcept
{
	using UT = make_unsigned_t<T>;
	using UU = make_unsigned_t<U>;
	if constexpr (is_signed_v<T> == is_signed_v<U>)
		return t < u;
	else if constexpr (is_signed_v<T>)
		return t < 0 ? true : UT(t) < u;
	else
		return u < 0 ? false : t < UU(u);
}
\end{lstlisting}

UT和UU别名声明为make\_unsigned\_t,这是C++17引入的一种辅助类型。其允许有符号类型到无符号类型的安全转换。

函数首先测试两个参数是有符号的,还是无符号的。然后,返回一个简单的比较。

然后,测试两边是否有符号。若该带符号值小于零,则可以返回true或false而不执行比较。否则,将有符号值转换为无符号值进行比较。

相同的逻辑也适用于其他比较函数。















