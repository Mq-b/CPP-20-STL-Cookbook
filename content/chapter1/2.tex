
若想格式化文本,可以使用传统的printf函数或STL iostream库,两者各有优缺点。

printf函数继承自C语言,50多年的发展,已让其很高效、灵活和方便。就是格式语法看起来有点晦涩,但习惯过后觉得还行。

\begin{lstlisting}[style=styleCXX]
printf("Hello, %s\n", c_string);
\end{lstlisting}

printf的缺点是弱类型安全。printf()函数(及其相关函数),使用C的可变参数模型将参数传递给格式化程序。在正常情况下非常高效,但参数类型与其对应的格式说明符不匹配时,可能会产生严重的问题。现代编译器会尽可能地进行类型检查,但编程模型本身存在缺陷,保护也只能到此为止。

STL的iostream库以可读性和运行时性能为代价确保了类型安全。iostream语法不常见,但很易懂。其重载按位左移操作符(<<),以允许产生格式化输出一连串的对象、操作数和格式化操纵符。

\begin{lstlisting}[style=styleCXX]
cout << "Hello, " << str << endl;
\end{lstlisting}

iostream的缺点在于语法和实现方面的复杂性,构建格式化字符串可能会冗长而晦涩。许多格式操纵符在使用后必须重置,否则会产生难以调试的级联格式错误。该库本身庞大而复杂,导致代码比printf等效代码大得多,速度也慢很多。

这种令人沮丧的情况让C++程序员别无选择,只能在两个有缺陷的方法中选择其中一个。

\subsubsection{How to do it…}

新格式库位于<format>头文件中。撰写本书时,其仅在MSVC(Microsoft)编译器中实现。当阅读本书时,应该在更多的编译器上实现了。如若不然,可以使用其参考实现进行理解fmt.dev(\url{j.bw.org/fmt})。

格式库基于Python3中的str.format()方法建模。格式字符串基本上与Python中的格式字符串相同,通常可以互换。来看一些简单的例子:

\begin{itemize}
\item 
format()函数可以接受一个string\_view格式的字符串和一个可变参数参数包,并返回一个字符串。其函数签名为:

\begin{lstlisting}[style=styleCXX]
template<typename... Args>
string format(string_view fmt, const Args&... args);
\end{lstlisting}

\item 
format()返回类型或值的字符串表示形式。例如:

\begin{lstlisting}[style=styleCXX]
string who{ "everyone" };
int ival{ 42 };
double pi{ std::numbers::pi };

format("Hello, {}!\n ", who); // Hello, everyone!
format("Integer: {}\n ", ival); // Integer: 42
format("π: {}\n", pi); // π: 3.141592653589793
\end{lstlisting}

格式化字符串使用大括号\{\}作为类型安全的占位符,可以将任何兼容类型的值转换为合理的字符串表示形式。

\item 
可以在格式字符串中包含多个占位符:

\begin{lstlisting}[style=styleCXX]
format("Hello {} {}", ival, who); // Hello 42
								  // everyone
\end{lstlisting}

\item 
可以指定替换值的顺序,这对本地化很有用:

\begin{lstlisting}[style=styleCXX]
format("Hello {1} {0}", ival, who); // Hello everyone 42
format("Hola {0} {1}", ival, who); // Hola 42 everyone
\end{lstlisting}

\item 
这也可以进行对齐,左(<)、右(>)或中心(\^{})对齐,可以选择性使用填充字符:

\begin{lstlisting}[style=styleCXX]
format("{:.<10}", ival); // 42........
format("{:.>10}", ival); // ........42
format("{:.^10}", ival); // ....42....
\end{lstlisting}

\item 
可以设置十进制数值的精度:

\begin{lstlisting}[style=styleCXX]
format("π: {:.5}", pi); // π: 3.1416
\end{lstlisting}
\end{itemize}

这是一个丰富而完整的格式化方式,具有iostream的类型安全,以及printf的性能和简单性,达到了鱼和熊掌兼得的目的。

\subsubsection{How it works…}

格式库不包括print()函数,这是C++23计划的一部分。format()函数本身返回一个字符串对象。若想打印字符串,需要使用iostream或cstdio。:(

可以使用iostream输出字符串:

\begin{lstlisting}[style=styleCXX]
cout << format("Hello, {}", who) << "\n";
\end{lstlisting}

或者使用cstdio:

\begin{lstlisting}[style=styleCXX]
puts(format("Hello, {}", who).c_str());
\end{lstlisting}

这两种方法都不理想,但是编写一个简单的print()函数并不难。在这个过程中,来了解一些格式库的工作方式。

下面是print()函数使用格式库的简单实现:

\begin{lstlisting}[style=styleCXX]
#include <format>
#include <string_view>
#include <cstdio>

template<typename... Args>
void print(const string_view fmt_str, Args&&... args) {
	auto fmt_args{ make_format_args(args...) };
	string outstr{ vformat(fmt_str, fmt_args) };
	fputs(outstr.c_str(), stdout);
}
\end{lstlisting}

使用与format()函数相同的参数。第一个参数是格式字符串的string\_view对象,后面作为参数的可变参数包。

make\_format\_args()函数的作用:接受参数包并返回一个对象,该对象包含适合格式化的已擦除类型的值。然后,将该对象传递给vformat(),vformat()再返回适合打印的字符串。我们使用fputs()将值输出到控制台上(这比cout高效得多)。

现在可以使用print()函数,来代替cout <{}< format()的组合:

\begin{lstlisting}[style=styleCXX]
print("Hello, {}!\n", who);
print("π: {}\n", pi);
print("Hello {1} {0}\n", ival, who);
print("{:.^10}\n", ival);
print("{:.5}\n", pi);
\end{lstlisting}

输出为:

\begin{tcblisting}{commandshell={}}
Hello, everyone!
π: 3.141592653589793
Hello everyone 42
....42....
3.1416
\end{tcblisting}

编译器支持C++23的print()时,使用\texttt{std::print;}就能完成所有工作,并且print()也会正常工作。

\subsubsection{There's more…}

能够格式化字符串和原语已经非常好了,但是要使格式库具有完整的功能,就需要对其进行定制,以便与自定义的类一起工作。

例如,这里有两个成员的简单结构体:分子和分母。将其输出为分数:

\begin{lstlisting}[style=styleCXX]
struct Frac {
	long n;
	long d;
};

int main() {
	Frac f{ 5, 3 };
	print("Frac: {}\n", f);
}
\end{lstlisting}

编译时,会出现一系列错误,例如:“没有定义的转换运算符…”。让我们来解决它!

当格式系统遇到要转换的对象时,其会寻找具有相应类型的格式化程序对象的特化。标准的特化对于常见的对象,如字符串和数字等。Frac类型的特化非常简单:

\begin{lstlisting}[style=styleCXX]
template<>
struct std::formatter<Frac>
{
	template<typename ParseContext>
	constexpr auto parse(ParseContext& ctx) {
		return ctx.begin();
	}
	template<typename FormatContext>
	auto format(const Frac& f, FormatContext& ctx) {
		return format_to(ctx.out(), "{0:d}/{1:d}",
		f.n, f.d);
	}
};
\end{lstlisting}

格式化特化,是具有两个简短模板函数的类:

\begin{itemize}
\item 
parse()函数解析格式字符串,从冒号之后(若没有冒号,则在开大括号之后)直到但不包括结束大括号(就是指定对象类型的部分)。其接受一个ParseContext对象,并返回一个迭代器。这里,可以只返回begin()迭代器。因为我们的类型不需要新语法,所以无需准备任何东西。

\item 
format()函数接受一个Frac对象和一个FormatContext对象,返回结束迭代器。format\_to()函数可使这变得很容易,其可以接受一个迭代器、一个格式字符串和一个参数包。本例中,参数包是Frac类的两个属性,分子和分母。

需要做的就是提供一个简单的格式字符串“\{0\}/\{1\}”以及分子和分母的值(0和1表示参数的位置)。
\end{itemize}

现在有了Frac的特化,可以将对象传递给print()从而获得一个可读的结果:

\begin{lstlisting}[style=styleCXX]
int main() {
	Frac f{ 5, 3 };
	print("Frac: {}\n", f);
}
\end{lstlisting}

输出为:

\begin{tcblisting}{commandshell={}}
Frac: 5/3
\end{tcblisting}

C++20通过提供高效、方便的类型安全文本格式库,解决了一个长期存在的问题。

