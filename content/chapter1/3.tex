
C++20允许在新的上下文中使用constexpr。因为这些可以在编译时计算,而非运行时,从而提高了效率。


\subsubsection{How to do it…}

其中包括在constexpr上下文中使用string和vector对象的能力。需要注意的是,这些对象本身可能不声明为constexpr,但可以在编译时上下文中使用:

\begin{lstlisting}[style=styleCXX]
constexpr auto use_string() {
	string str{"string"};
	return str.size();
}
\end{lstlisting}

也可以在constexpr上下文中使用算法:

\begin{lstlisting}[style=styleCXX]
constexpr auto use_vector() {
	vector<int> vec{ 1, 2, 3, 4, 5};
	return accumulate(begin(vec), end(vec), 0);
}
\end{lstlisting}

累加算法的结果在编译时和constexpr上下文中可用。

\subsubsection{How it works…}

constexpr说明符声明了一个可以在编译时求值的变量或函数。在C++20之前,这仅限于用文字值初始化的对象,或在有限约束条件下的函数。C++17允许某种程度上的扩展,而C++20进一步对其进行了扩展。

C++20开始,标准string和vector类具有限定的构造函数和析构函数,这是可在编译时使用的前提。所以分配给string或vector对象的内存,必须在编译时释放。

例如,constexpr函数返回一个vector,编译时不会出错:

\begin{lstlisting}[style=styleCXX]
constexpr auto use_vector() {
	vector<int> vec{ 1, 2, 3, 4, 5};
	return vec;
}
\end{lstlisting}

但是若试图在运行时环境中使用结果,会得到一个在常量求值期间分配内存的错误:

\begin{lstlisting}[style=styleCXX]
int main() {
	constexpr auto vec = use_vector();
	return vec[0];
}
\end{lstlisting}

因为在编译期间分配和释放了vector对象,所以该对象在运行时不再可用。

另外,可以在运行时使用一些vector对象的适配constexpr的方法,比如size():

\begin{lstlisting}[style=styleCXX]
int main() {
	constexpr auto value = use_vector().size();
	return value;
}
\end{lstlisting}

因为size()方法是constexpr限定的,所以表达式可以在编译时求值。












