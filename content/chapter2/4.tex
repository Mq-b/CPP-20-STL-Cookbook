
C++17起,if和switch现在有初始化语法,就像C99开始的for循环一样。可以在限制条件中,确定变量的使用范围。

\subsubsection{How to do it…}

可能习惯于这样的代码:

\begin{lstlisting}[style=styleCXX]
const string artist{ "Jimi Hendrix" };
size_t pos{ artist.find("Jimi") };
if(pos != string::npos) {
	cout << "found\n";
} else {
	cout << "not found\n";
}
\end{lstlisting}

这使得变量pos暴露在条件语句的作用域之外,需要对其进行管理,否则它可能与使用相同符号的其他变量发生冲突。

现在可以把初始化表达式放在if条件中:

\begin{lstlisting}[style=styleCXX]
if(size_t pos{ artist.find("Jimi") }; pos != string::npos) {
	cout << "found\n";
} else {
	cout << "not found\n";
}
\end{lstlisting}

现在pos变量的作用域限制在条件变量的作用域内,这使得命名空间内保持干净和可管理。

\subsubsection{How it works…}

初始化表达式可以在if或switch语句中使用,以下是其各自的例子。

\begin{itemize}
\item 
使用带有if语句的初始化表达式:

\begin{lstlisting}[style=styleCXX]
if(auto var{ init_value }; condition) {
	// var is visible
} else {
	// var is visible
}
// var is NOT visible
\end{lstlisting}

在初始化表达式中定义的变量,在整个if语句的范围内是可见的,包括else子句。出了if语句的作用域,该变量将不再可见,并且将调用相关的析构函数。

\item 
使用switch语句的初始化表达式:

\begin{lstlisting}[style=styleCXX]
switch(auto var{ init_value }; var) {
	case 1: ...
	case 2: ...
	case 3: ...
	...
	Default: ...
}
// var is NOT visible
\end{lstlisting}

初始化表达式中定义的变量在整个switch语句的作用域中可见,包括所有case子句和default子句(如果包含的话)。当出了switch语句的作用域,变量将不再可见,并且将调用相关的析构函数。

\end{itemize}

\subsubsection{There's more…}

一个有趣的用例是限制锁定互斥锁的lock\_guard的作用域。使用初始化表达式,会让代码变得更简单:

\begin{lstlisting}[style=styleCXX]
if (lock_guard<mutex> lg{ my_mutex }; condition) {
	// interesting things happen here
}
\end{lstlisting}

lock\_guard在构造函数中锁定互斥量,在析构函数中解锁互斥量。现在,当lock\_guard超出if语句的作用域时,将自动销毁。过去,必须删除它或将整个if语句括在一个额外的大括号块中。

另一个用例可能是使用使用输出参数的遗留接口,就像下面示例,来自于SQLite:

\begin{lstlisting}[style=styleCXX]
if(
	sqlite3_stmt** stmt,
	auto rc = sqlite3_prepare_v2(db, sql, -1, &_stmt,
	nullptr);
	!rc) {
	// do SQL things
} else { // handle the error
	// use the error code
	return 0;
}
\end{lstlisting}

这里,可以将语句句柄和错误代码本地化到if语句的范围内。否则,需要全局化地管理这些对象。

使用初始化表达式将有助于保持代码紧凑、整洁、紧凑、易阅读。重构和管理代码也会变得更加容易。













