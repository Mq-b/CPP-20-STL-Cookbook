
当模板函数或类模板构造函数(C++17起)的实参类型足够清楚,无需使用模板实参,编译器就能理解时,就会进行模板实参推导。这个功能有一定的规则,但主要规则是很直观的。

\subsubsection{How to do it…}

通常,当使用具有明确兼容参数的模板时,模板参数推断会自动发生。让我们看一些例子。

\begin{itemize}
\item 
函数模板中,参数推断通常是这样:

\begin{lstlisting}[style=styleCXX]
template<typename T>
const char * f(const T a) {
	return typeid(T).name();
}
int main() {
	cout << format("T is {}\n", f(47));
	cout << format("T is {}\n", f(47L));
	cout << format("T is {}\n", f(47.0));
	cout << format("T is {}\n", f("47"));
	cout << format("T is {}\n", f("47"s));
}
\end{lstlisting}

输出:

\begin{tcblisting}{commandshell={}}
T is int
T is long
T is double
T is char const *
T is class std::basic_string<char...
\end{tcblisting}

因为类型很容易识别,所以没有理由在函数调用中指定f<int>(47)这样的模板形参。编译器可以从实参中推导出<int>类型。

\begin{tcolorbox}[colback=webgreen!5!white,colframe=webgreen!75!black,title=Note]
上面的输出显示了有意义的类型名,大多数编译器都会使用简写,比如i表示int, PKc表示const char *,等等。
\end{tcolorbox}

\item 
这同样适用于多个模板参数:

\begin{lstlisting}[style=styleCXX]
template<typename T1, typename T2>
string f(const T1 a, const T2 b) {
	return format("{} {}", typeid(T1).name(),
	typeid(T2).name());
}

int main() {
	cout << format("T1 T2: {}\n", f(47, 47L));
	cout << format("T1 T2: {}\n", f(47L, 47.0));
	cout << format("T1 T2: {}\n", f(47.0, "47"));
}
\end{lstlisting}

输出:

\begin{tcblisting}{commandshell={}}
T1 T2: int long
T1 T2: long double
T1 T2: double char const *
\end{tcblisting}

这里编译器同时推导出了T1和T2的类型。

\item 
注意,类型必须与模板兼容。例如,不能从字面量中获取引用:

\begin{lstlisting}[style=styleCXX]
template<typename T>
const char * f(const T& a) {
	return typeid(T).name();
}
int main() {
	int x{47};
	f(47); // this will not compile
	f(x); // but this will
}
\end{lstlisting}

\item 
C++17起,还可以对类使用模板参数推断:

\begin{lstlisting}[style=styleCXX]
pair p(47, 47.0); // deduces to pair<int, double>
tuple t(9, 17, 2.5); // deduces to tuple<int, int, double>
\end{lstlisting}

这消除了对std::make\_pair()和std::make\_tuple()的需求,现在可以直接初始化这些类,而无需显式的模板参数。std::make\_*工厂函数则保持向后兼容性可用。
\end{itemize}

\subsubsection{How it works…}

我们定义一个类,这样就可以了解它是如何工作的:

\begin{lstlisting}[style=styleCXX]
template<typename T1, typename T2, typename T3>
class Thing {
	T1 v1{};
	T2 v2{};
	T3 v3{};
	public:
	explicit Thing(T1 p1, T2 p2, T3 p3)
	: v1{p1}, v2{p2}, v3{p3} {}
	string print() {
		return format("{}, {}, {}\n",
			typeid(v1).name(),
			typeid(v2).name(),
			typeid(v3).name()
		);
	}
};
\end{lstlisting}

这是一个具有三种类型和三个相应数据成员的模板类。有一个print()函数,该函数返回一个带有三个类型名称的格式化字符串。

若没有模板参数推导,则需要实例化一个这种类型的对象:

\begin{lstlisting}[style=styleCXX]
Things<int, double, string> thing1{1, 47.0, "three" }
\end{lstlisting}

现在可以这样做:

\begin{lstlisting}[style=styleCXX]
Things thing1{1, 47.0, "three" }
\end{lstlisting}

既简单又不易出错。

当在thing1对象上使用print()函数时,得到了这样的结果:

\begin{lstlisting}[style=styleCXX]
cout << thing1.print();
\end{lstlisting}

输出:

\begin{tcblisting}{commandshell={}}
int, double, char const *
\end{tcblisting}

STL包含了一些这样的辅助函数,比如make\_pair()和make\_tuple()等。这些代码现在已经过时了,但是为了与旧代码兼容,会保留这些代码。

\subsubsection{There's more…}

考虑带有参数包的构造函数:

\begin{lstlisting}[style=styleCXX]
template <typename T>
class Sum {
	T v{};
public:
	template <typename... Ts>
	Sum(Ts&& ... values) : v{ (values + ...) } {}
	const T& value() const { return v; }
};
\end{lstlisting}

注意构造函数中的折叠表达式(values + …)。这是C++17的一个特性,它将操作符应用于一个形参包的所有成员。本例中,将v初始化为参数包的和。

该类的构造函数接受任意数量的形参,其中每个形参可以是不同的类。可以这样调用:

\begin{lstlisting}[style=styleCXX]
Sum s1 { 1u, 2.0, 3, 4.0f }; // unsigned, double, int,
							 // float
Sum s2 { "abc"s, "def" }; // std::sring, c-string
\end{lstlisting}

当然,这无法编译。模板实参推导不能为所有这些不同的形参找到一个公共类型,编译器将会报出一个错误消息:

\begin{tcblisting}{commandshell={}}
cannot deduce template arguments for 'Sum'
\end{tcblisting}

可以用模板推导指南来解决这个问题。推导指南是一种辅助模式,用于协助编译器进行复杂的推导。下面是构造函数的指南:

\begin{lstlisting}[style=styleCXX]
template <typename... Ts>
Sum(Ts&& ... ts) -> Sum<std::common_type_t<Ts...>>;
\end{lstlisting}

这告诉编译器使用std::common\_type\_t的特征,试图为包中的所有参数找到一个公共类型。现在我们的参数推导工作了,可以看到其确切的类型:

\begin{lstlisting}[style=styleCXX]
Sum s1 { 1u, 2.0, 3, 4.0f }; // unsigned, double, int,
							 // float
Sum s2 { "abc"s, "def" }; // std::sring, c-string

auto v1 = s1.value();
auto v2 = s2.value();
cout << format("s1 is {} {}, s2 is {} {}",
		typeid(v1).name(), v1, typeid(v2).name(), v2);
\end{lstlisting}

输出:

\begin{tcblisting}{commandshell={}}
s1 is double 10, s2 is class std::string abcdef
\end{tcblisting}



