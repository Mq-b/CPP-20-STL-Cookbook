
对于C++20,std::span类是一个包装器,可在连续的对象序列上创建视图。span没有属于自己的数据,其引用底层结构中的数据。可以把它看作C数组的string\_view,底层结构可以是C数组、vector或STL array。

\subsubsection{How to do it…}

可以使用兼容的连续存储结构创建span,最常见的的是C数组。例如,将一个C数组传递给函数,该数组将降级为指针,并且函数没有办法直接了解数组的长度:

\begin{lstlisting}[style=styleCXX]
void parray(int * a); // loses size information
\end{lstlisting}

若用span形参定义函数,可以传递一个C数组,其将会升级为span。下面是一个模板函数,其接受一个span参数,并以元素和字节为单位输出:

\begin{lstlisting}[style=styleCXX]
template<typename T>
void pspan(span<T> s) {
	cout << format("number of elements: {}\n", s.size());
	cout << format("size of span: {}\n", s.size_bytes());
	for(auto e : s) cout << format("{} ", e);
	cout << "\n";
}
\end{lstlisting}

可以传递一个C数组给这个函数,会自动提升为span:

\begin{lstlisting}[style=styleCXX]
int main() {
	int carray[] { 1, 2, 3, 4, 5, 6, 7, 8, 9, 10 };
	pspan<int>(carray);
}
\end{lstlisting}

输出为:

\begin{tcblisting}{commandshell={}}
number of elements: 10
number of bytes: 40
1 2 3 4 5 6 7 8 9 10
\end{tcblisting}

span的目的是封装原始数据,并以最小的开销保证安全性和实用性。

\subsubsection{How it works…}

span类本身不拥有数据,数据属于底层数据结构。span本质上是底层数据的视图,并提供了一些有用的成员函数。

<span>头文件中定义的span类看起来像这样:

\begin{lstlisting}[style=styleCXX]
template<typename T, size_t Extent = std::dynamic_extent>
class span {
	T * data;
	size_t count;
	public:
	...
};
\end{lstlisting}

Extent参数是一个constexpr size\_t类型的常量,在编译时计算。其要么是底层数据中的元素数量,要么是std::dynamic\_extent常量,这表明其大小可变。这允许span使用底层结构(如vector),但大小并不总是相同。

所有成员函数都是constexpr和const限定的。成员函数包括:

\begin{table}[H]
\centering
\begin{tabular}{|l|l|}
\hline
\textbf{公共成员函数} & \textbf{返回值}                                  \\ \hline
T\& front()            & 第一个元素                             \\ \hline
T\& back()             & 最后一个元素                              \\ \hline
T\& operator{[}{]}     & 索引位置的元素                            \\ \hline
T* data()              & 指向序列头部的指针    \\ \hline
iterator begin()       & 指向第一个元素的迭代器              \\ \hline
iterator end()         & 指向最后一个元素后面的迭代器        \\ \hline
iterator rbegin()      & 指向第一个元素的反向迭代器       \\ \hline
iterator rend()        & 指向最后一个元素后面的反向迭代器 \\ \hline
size\_t size()         & 序列中元素的数量        \\ \hline
size\_t size\_bytes()  & 序列所占字节数           \\ \hline
bool empty()           & 空为真                                 \\ \hline
\begin{tabular}[c]{@{}l@{}}span\textless{}T\textgreater first\textless{}count\textgreater{}()\\ span\textless{}T\textgreater first(count)\end{tabular} &
\begin{tabular}[c]{@{}l@{}}由序列的第一个计数元素组成\\ 的子span\end{tabular} \\ \hline

\end{tabular}
\end{table}

\begin{table}[H]
\centering
\begin{tabular}{|l|l|}
\hline
\begin{tabular}[c]{@{}l@{}}span\textless{}T\textgreater last\textless{}count\textgreater{}()\\ span\textless{}T\textgreater last(count)\end{tabular} &
返回最后一个count元素的子span \\ \hline
\begin{tabular}[c]{@{}l@{}}span\textless{}T\textgreater subspan(offset, \\ count)\end{tabular} &
\begin{tabular}[c]{@{}l@{}}返回以offset偏移对头进行偏移后的count个\\ 元素组成的子span\end{tabular} \\ \hline
\end{tabular}
\end{table}

\begin{tcolorbox}[colback=webgreen!5!white,colframe=webgreen!75!black,title=重要的Note]
span类只是一个简单的包装器,不执行边界检查。若尝试访问n个元素中的元素n+1,结果就是未定义的,所以最好不要这样做。
\end{tcolorbox}

