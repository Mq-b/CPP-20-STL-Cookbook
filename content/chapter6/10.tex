
The std::merge() algorithm takes two sorted sequences and creates a third merged and sorted sequence. This technique is often used as part of a merge sort, allowing very large amounts of data to be broken down into chunks, sorted separately, and merged into one sorted target.

\subsubsection{How to do it…}

For this recipe, we'll take two sorted vector containers and merge them into a third vector using std::merge().

\begin{itemize}
\item 
We'll start with a simple function to print the contents of a container:

\begin{lstlisting}[style=styleCXX]
void printc(const auto& c, string_view s = "") {
	if(s.size()) cout << format("{}: ", s);
	for(auto e : c) cout << format("{} ", e);
	cout << '\n';
}
\end{lstlisting}

We'll use this to print the source and destination sequences.

\item 
In the main() function, we'll declare our source vectors, along with the destination vector, and print them out:

\begin{lstlisting}[style=styleCXX]
int main() {
	vector<string> vs1{ "dog", "cat",
		"velociraptor" };
	vector<string> vs2{ "kirk", "sulu", "spock" };
	vector<string> dest{};
	printc(vs1, "vs1");
	printc(vs2, "vs2");
	...
}
\end{lstlisting}

The output is:

\begin{tcblisting}{commandshell={}}
vs1: dog cat velociraptor
vs2: kirk sulu spock
\end{tcblisting}

\item 
Now we can sort our vectors and print them again:

\begin{lstlisting}[style=styleCXX]
sort(vs1.begin(), vs1.end());
sort(vs2.begin(), vs2.end());
printc(vs1, "vs1 sorted");
printc(vs2, "vs2 sorted");
\end{lstlisting}

Output:

\begin{tcblisting}{commandshell={}}
vs1 sorted: cat dog velociraptor
vs2 sorted: kirk spock sulu
\end{tcblisting}

\item 
Now that our source containers are sorted, we can merge them for our final merged result:

\begin{lstlisting}[style=styleCXX]
merge(vs1.begin(), vs1.end(), vs2.begin(), vs2.end(),
	back_inserter(dest));
printc(dest, "dest");
\end{lstlisting}

Output:

\begin{tcblisting}{commandshell={}}
dest: cat dog kirk spock sulu velociraptor
\end{tcblisting}

This output represents the merge of the two sources into one sorted vector.
\end{itemize}

\subsubsection{How it works…}

The merge() algorithm takes begin() and end() iterators from both the sources and an output iterator for the destination:

\begin{lstlisting}[style=styleCXX]
OutputIt merge(InputIt1, InputIt1, InputIt2, InputIt2, OutputIt)
\end{lstlisting}

It takes the two input ranges, performs its merge/sort operation, and sends the resulting sequence to the output iterator.

















