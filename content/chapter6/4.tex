
我们早已经解决了“如何高效地对可比元素进行排序“的问题,对于大多数应用程序,没有理由重新发明这个轮子。STL通过std::sort()算法提供了一个优秀的排序解决方案。虽然标准没有指定排序算法,但当应用于具有n个元素的范围时,最差情况的时间复杂度为O(nlog(n))。

几十年前,快速排序算法被认为是大多数用途的一个很好的折衷,通常比其他可比算法更快。现在,我们有混合算法,可以根据情况在不同的方法之间进行选择,经常在运行中切换算法。目前大多数C++库都使用一种混合方法,将插入排序和插入排序相结合。通常,std::sort()也具有出色的性能。

\subsubsection{How to do it…}

这个示例中,我们将检查std::sort()算法。sort()算法适用于具有随机访问迭代器的容器。这里,我们将使用int类型的vector:

\begin{itemize}
\item 
我们会从一个函数开始测试容器是否已排序:

\begin{lstlisting}[style=styleCXX]
void check_sorted(auto &c) {
	if(!is_sorted(c.begin(), c.end())) cout << "un";
	cout << "sorted: ";
}
\end{lstlisting}

使用std::is\_sorted()算法,并根据结果打印“sorted:”或“unsorted:”。

\item 
这里需要一个函数来打印vector:

\begin{lstlisting}[style=styleCXX]
void printc(const auto &c) {
	check_sorted(c);
	for(auto& e : c) cout << e << ' ';
	cout << '\n';
}
\end{lstlisting}

这个函数使用check\_sorted(),来查看容器排序的状态。

\item 
现在,可以在main()函数中定义并打印一个int类型的vector:

\begin{lstlisting}[style=styleCXX]
int main() {
	vector<int> v{ 1, 2, 3, 4, 5, 6, 7, 8, 9, 10 };
	printc(v);
	…
}
\end{lstlisting}

输出如下所示:

\begin{tcblisting}{commandshell={}}
sorted: 1 2 3 4 5 6 7 8 9 10
\end{tcblisting}

\item 
为了测试std::sort()算法,需要一个未排序的vector。下面是一个简单的随机容器的函数:

\begin{lstlisting}[style=styleCXX]
void randomize(auto& c) {
	static std::random_device rd;
	static std::default_random_engine rng(rd());
	std::shuffle(c.begin(), c.end(), rng);
}
\end{lstlisting}

std::random\_device类使用系统的硬件熵源。大多数现代系统都有一个,否则库将对其进行模拟。std::default\_random\_engine()函数从熵源生成随机数,std::shuffle()用其来随机化容器。

现在,可以对容器调用randomize()并打印结果:

\begin{lstlisting}[style=styleCXX]
randomize(v);
printc(v);
\end{lstlisting}

输出为:

\begin{tcblisting}{commandshell={}}
unsorted: 6 3 4 8 10 1 2 5 9 7
\end{tcblisting}

因为它是随机的,每个人的输出可能都会不同。事实上,我每次运行它,也会得到不同的结果:

\begin{lstlisting}[style=styleCXX]
for(int i{3}; i; --i) {
	randomize(v);
	printc(v);
}
\end{lstlisting}

输出为:

\begin{tcblisting}{commandshell={}}
unsorted: 3 1 8 5 10 2 7 9 6 4
unsorted: 7 6 5 1 3 9 10 2 4 8
unsorted: 4 2 3 10 1 9 5 6 8 7
\end{tcblisting}

\item 
只需使用std::sort()就能对vector进行排序:

\begin{lstlisting}[style=styleCXX]
std::sort(v.begin(), v.end());
printc(v);
\end{lstlisting}

输出为:

\begin{tcblisting}{commandshell={}}
sorted: 1 2 3 4 5 6 7 8 9 10
\end{tcblisting}

默认情况下,sort()算法使用小于操作符对提供的迭代器指定范围内的元素进行排序。

\item 
partial\_sort()算法将对容器的一部分进行排序:

\begin{lstlisting}[style=styleCXX]
cout << "partial_sort:\n";
randomize(v);
auto middle{ v.begin() + (v.size() / 2) };
std::partial_sort(v.begin(), middle, v.end());
printc(v);
\end{lstlisting}

partial\_sort()接受三个迭代器:起始、中间和末尾。它对容器进行排序,使中间之前的元素排序。中间之后的元素不保证是原来的顺序。输出如下:

\begin{tcblisting}{commandshell={}}
unsorted: 1 2 3 4 5 10 7 6 8 9
\end{tcblisting}

注意,前五个元素是有序的。

\item 
partition()算法不进行任何排序,其会重新排列容器,使某些元素出现在容器的前面:

\begin{lstlisting}[style=styleCXX]
coutrandomize(v);
printc(v);
partition(v.begin(), v.end(), [](int i)
	{ return i > 5; });
printc(v);
\end{lstlisting}

第三个参数是谓词lambda,将决定哪些元素会移到前面。

输出为:

\begin{tcblisting}{commandshell={}}
unsorted: 4 6 8 1 9 5 2 7 3 10
unsorted: 10 6 8 7 9 5 2 1 3 4
\end{tcblisting}

注意,小于5的值会移动到容器的前面。

\item 
sort()算法支持可选的比较函数,该函数可用于非标准比较。例如,给定一个名为things的类:

\begin{lstlisting}[style=styleCXX]
struct things {
	string s_;
	int i_;
	string str() const {
		return format("({}, {})", s_, i_);
	}
};
\end{lstlisting}

可以创建一个vector:

\begin{lstlisting}[style=styleCXX]
vector<things> vthings{ {"button", 40},
	{"hamburger", 20}, {"blog", 1000},
	{"page", 100}, {"science", 60} };
\end{lstlisting}

这里需要一个函数对其进行打印:

\begin{lstlisting}[style=styleCXX]
void print_things(const auto& c) {
	for (auto& v : c) cout << v.str() << ' ';
	cout << '\n';
}
\end{lstlisting}

\item 
现在可以对vector进行排序和打印:

\begin{lstlisting}[style=styleCXX]
std::sort(vthings.begin(), vthings.end(),
		[](const things &lhs, const things &rhs) {
	return lhs.i_ < rhs.i_;
});
print_things(vthings);
\end{lstlisting}

输出为:

\begin{tcblisting}{commandshell={}}
(hamburger, 20) (button, 40) (science, 60) (page, 100)
(blog, 1000)
\end{tcblisting}

注意比较函数在i\_成员上排序,所以结果是按i\_排序的,可以对s\_成员进行排序:

\begin{lstlisting}[style=styleCXX]
std::sort(vthings.begin(), vthings.end(),
		[](const things &lhs, const things &rhs) {
	return lhs.s_ < rhs.s_;
});
print_things(vthings);
\end{lstlisting}

现在,得到了这样的输出:

\begin{tcblisting}{commandshell={}}
(blog, 1000) (button, 40) (hamburger, 20) (page, 100)
(science, 60)
\end{tcblisting}

\end{itemize}

\subsubsection{How it works…}

sort()函数的工作原理是对由两个迭代器指示的元素范围应用排序算法,用于该范围的开始和结束。

通常,这些算法使用小于操作符来比较元素。当然,也可以使用比较函数,通常以lambda形式提供:

\begin{lstlisting}[style=styleCXX]
std::sort(vthings.begin(), vthings.end(),
[](const things& lhs, const things& rhs) {
	return lhs.i_ < rhs.i_;
});
\end{lstlisting}

比较函数接受两个参数,并返回bool类型,其签名相当于:

\begin{lstlisting}[style=styleCXX]
bool cmp(const Type1& a, const Type2& b);
\end{lstlisting}

sort()函数使用std::swap()来移动元素。这在计算周期和内存使用方面都很高效,这减轻了为读写排序对象分配空间的需要。这也是为什么partial\_sort()和partition()函数,不能保证未排序元素顺序的原因。











