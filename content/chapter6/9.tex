
排列有许多用例,包括测试、统计、研究等。next\_permutation()算法通过将容器重新排序到下一个字典排列来生成排列。

\subsubsection{How to do it…}

对于这个示例,我们将打印出一组三个字符串的排列:

\begin{itemize}
\item 
首先创建一个简短的函数来打印容器的内容:

\begin{lstlisting}[style=styleCXX]
void printc(const auto& c, string_view s = "") {
	if(s.size()) cout << format("{}: ", s);
	for(auto e : c) cout << format("{} ", e);
	cout << '\n';
}
\end{lstlisting}

使用简单的函数来打印数据集和排列。

\item 
main()函数中,声明了一个字符串对象的vector,并使用sort()算法对其进行排序。

\begin{lstlisting}[style=styleCXX]
int main() {
	vector<string> vs{ "dog", "cat", "velociraptor" };
	sort(vs.begin(), vs.end());
	...
}
\end{lstlisting}

next\_permutation()函数需要一个已排序的容器。

\item 
现在可以在do循环中使用next\_permutation()列出排布情况:

\begin{lstlisting}[style=styleCXX]
do {
	printc(vs);
} while (next_permutation(vs.begin(), vs.end()));
\end{lstlisting}

next\_permutation()函数修改容器,若有另一个排列则返回true若没有则返回false。

输出列出了三种宠物的六种排列:

\begin{tcblisting}{commandshell={}}
cat dog velociraptor
cat velociraptor dog
dog cat velociraptor
dog velociraptor cat
velociraptor cat dog
velociraptor dog cat
\end{tcblisting}
\end{itemize}

\subsubsection{How it works…}

std::next\_permutation()算法生成一组值的字典排列,即基于字典排序的排列。必须对输入进行排序,因为算法按字典顺序逐级进行排列。所以,若从3,2,1这样的集合开始,其会立即终止,因为这三个元素的最后一个字符是字典顺序的最后一个。

例如:

\begin{lstlisting}[style=styleCXX]
vector<string> vs{ "velociraptor", "dog", "cat" };
do {
	printc(vs);
} while (next_permutation(vs.begin(), vs.end()));
\end{lstlisting}

输出为:

\begin{tcblisting}{commandshell={}}
velociraptor dog cat
\end{tcblisting}

虽然术语lexicographically意味着字母顺序,但实现使用标准比较操作符,因此其适用于任何可排序的值。

同样,若集合中的值重复,则仅根据字典序对它们进行计数。这里有一个int类型的vector,其有两个五个值的重复序列:

\begin{lstlisting}[style=styleCXX]
vector<int> vi{ 1, 2, 3, 4, 5, 1, 2, 3, 4, 5 };
sort(vi.begin(), vi.end());
printc(vi, "vi sorted");
long count{};
do {
	++count;
} while (next_permutation(vi.begin(), vi.end()));
cout << format("number of permutations: {}\n", count);
\end{lstlisting}

输出为:

\begin{tcblisting}{commandshell={}}
Vi sorted: 1 1 2 2 3 3 4 4 5 5
number of permutations: 113400
\end{tcblisting}

这些值有113,400种排列。注意,不是10!(3,628,800),因为有些值重复。因为3,3和3,3排序相同,所以其是字典序。

换句话说,若列出这个短集合的排列:

\begin{lstlisting}[style=styleCXX]
vector<int> vi2{ 1, 3, 1 };
sort(vi2.begin(), vi2.end());
do {
	printc(vi2);
} while (next_permutation(vi2.begin(), vi2.end()));
\end{lstlisting}

由于有重复,所以只有三种排列,不是3的阶乘(9):

\begin{tcblisting}{commandshell={}}
1 1 3
1 3 1
3 1 1
\end{tcblisting}






