
std::transform()函数非常强大和灵活,是库中最常用的算法之一。它将函数或lambda应用于容器中的每个元素,将结果存储在另一个容器中,同时保留原始的元素。

\subsubsection{How to do it…}

这个示例中,将探索std::transform()函数的某些使用方式:

\begin{itemize}
\item 
从一个打印容器内容的函数开始:

\begin{lstlisting}[style=styleCXX]
void printc(auto& c, string_view s = "") {
	if(s.size()) cout << format("{}: ", s);
	for(auto e : c) cout << format("{} ", e);
	cout << '\n';
}
\end{lstlisting}

我们将使用它来查看转换的结果。

\item 
在main()函数中,声明两个vector:

\begin{lstlisting}[style=styleCXX]
int main() {
	vector<int> v1{ 1, 2, 3, 4, 5, 6, 7, 8, 9, 10 };
	vector<int> v2;
	printc(v1, "v1");
	...
}
\end{lstlisting}

输出v1的内容:

\begin{tcblisting}{commandshell={}}
v1: 1 2 3 4 5 6 7 8 9 10
\end{tcblisting}

\item 
现在可以使用transform()函数将每个值的平方插入到v2中:

\begin{lstlisting}[style=styleCXX]
cout << "squares:\n";
transform(v1.begin(), v1.end(), back_inserter(v2),
	[](int x){ return x * x; });
printc(v2, "v2");
\end{lstlisting}

transform()函数有四个参数。前两个是源范围的begin()和end()迭代器,第三个参数是目标范围的begin()迭代器。本例中,使用back\_inserter()算法将结果插入到v2中。第四个参数是变换函数,我们使用lambda来对值进行平方。

输出为:

\begin{tcblisting}{commandshell={}}
squares:
v2: 1 4 9 16 25 36 49 64 81 100
\end{tcblisting}

\item 
当然,可以对任何类型使用transform(),下面是一个将字符串对象的向量转换为小写的例子。首先,需要一个函数来返回字符串的小写值:

\begin{lstlisting}[style=styleCXX]
string str_lower(const string& s) {
	string outstr{};
	for(const char& c : s) {
		outstr += tolower(c);
	}
	return outstr;
}
\end{lstlisting}

现在可以在transform中使用str\_lower()函数了:

\begin{lstlisting}[style=styleCXX]
vector<string> vstr1{ "Mercury", "Venus", "Earth",
	"Mars", "Jupiter", "Saturn", "Uranus", "Neptune",
	"Pluto" };
vector<string> vstr2;
printc(vstr1, "vstr1");
cout << "str_lower:\n";
transform(vstr1.begin(), vstr1.end(),
	back_inserter(vstr2),
	[](string& x){ return str_lower(x); });
printc(vstr2, "vstr2");
\end{lstlisting}

这将为vstr1中的每个元素调用str\_lower(),并将结果插入vstr2。结果为:

\begin{tcblisting}{commandshell={}}
vstr: Mercury Venus Earth Mars Jupiter Saturn Uranus
Neptune Pluto
str_lower:
vstr: mercury venus earth mars jupiter saturn uranus
neptune pluto
\end{tcblisting}

(是的,冥王星对我来说永远都是行星。)

\item 
transform还有一个范围版本:

\begin{lstlisting}[style=styleCXX]
cout << "ranges squares:\n";
auto view1 = views::transform(v1, [](int x){
	return x * x; });
printc(view1, "view1");
\end{lstlisting}

范围版本具有更简洁的语法,并返回一个视图对象,而不是填充另一个容器。
\end{itemize}

\subsubsection{How it works…}

std::transform()函数的工作原理与std::copy()相似,只不过增加了用户提供的函数。输入范围内的每个元素都传递给函数,函数的返回值会复制赋值给目标迭代器。

值得注意的是,transform()并不保证元素将按顺序处理。若需要确保转换的顺序,需要使用for循环:

\begin{lstlisting}[style=styleCXX]
v2.clear(); // reset vector v2 to empty state
for(auto e : v1) v2.push_back(e * e);
printc(v2, "v2");
\end{lstlisting}

输出为:

\begin{tcblisting}{commandshell={}}
v2: 1 4 9 16 25 36 49 64 81 100
\end{tcblisting}
















