STL的强大之处在于容器接口的标准化。若容器具有特定的功能,那么该功能的接口很可能是跨容器类型标准化的。这种标准化使得容器可以直接支持某些算法。

例如,若想对int类型的vector中所有元素求和,可以使用循环:

\begin{lstlisting}[style=styleCXX]
vector<int> x { 1, 2, 3, 4, 5 };
long sum{};
for( int i : x ) sum += i; // sum is 15
\end{lstlisting}

或者一个算法:

\begin{lstlisting}[style=styleCXX]
vector<int> x { 1, 2, 3, 4, 5 };
auto sum = accumulate(x.begin(), x.end(), 0); // sum is 15
\end{lstlisting}

同样的语法也适用于其他容器:

\begin{lstlisting}[style=styleCXX]
deque<int> x { 1, 2, 3, 4, 5 };
auto sum = accumulate(x.begin(), x.end(), 0); // sum is 15
\end{lstlisting}

算法版本不一定更短,但更易于阅读和维护,而且算法通常比等效循环更有效。

C++20起,范围库提供了一组操作范围和视图的替代算法。本书将在适当的地方演示这些替代方案,有关范围和视图的更多信息,请参阅本书第1章中的示例。

大多数算法都在算法头中。一些数值算法,特别是accumulate(),在<numeric>头文件中,而一些与内存相关的算法在<memory>头文件中。

我们将使用以下示例中介绍STL算法:

\begin{itemize}
\item 
基于迭代器的复制

\item 
将容器元素连接到一个字符串中

\item 
std::sort——排序容器元素

\item 
std::transform——修改容器内容

\item 
查找特定项

\item 
将容器的元素值限制在std::clamp的范围内

\item 
std::sample——采集样本数据集

\item 
生成有序数据序列

\item 
合并已排序容器
\end{itemize}













