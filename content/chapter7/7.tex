
It is common for input from users to include extraneous whitespace at one or both ends of a string. This can be problematic, so we often need to remove it. In this recipe, we'll use the string class methods, find\_first\_not\_of() and find\_last\_not\_of(), to trim whitespace from the ends of a string.

\subsubsection{How to do it…}

The string class includes methods for finding elements that are, or are not, included in a list of characters. We'll use these methods to trim string:

\begin{itemize}
\item 
We start by defining string with input from a hypothetical ten-thumbed user:

\begin{lstlisting}[style=styleCXX]
int main() {
	string s{" \t ten-thumbed input \t \n \t "};
	cout << format("[{}]\n", s);
	...
\end{lstlisting}

Our input has a few extra tab \verb|\|t and newline \verb|\|n characters before and after the content. We print it with surrounding brackets to show the whitespace:

\begin{lstlisting}[style=styleCXX]
[           ten-thumbed input
       ]
\end{lstlisting}

\item 
Here's a trimstr() function to remove all the whitespace characters from both ends of string:

\begin{lstlisting}[style=styleCXX]
string trimstr(const string& s) {
	constexpr const char * whitespace{ " \t\r\n\v\f" };
	if(s.empty()) return s;
	const auto first{ s.find_first_not_of(whitespace) };
	if(first == string::npos) return {};
	const auto last{ s.find_last_not_of(whitespace) };
	return s.substr(first, (last - first + 1));
}
\end{lstlisting}

We defined our set of whitespace characters as space, tab, return, newline, vertical tab, and form feed. Some of these are more common than others, but that's the canonical set.

This function uses the find\_first\_not\_of() and find\_last\_not\_of() methods of the string class to find the first/last elements that are not a member of the set.

\item 
Now, we can call the function to get rid of all that unsolicited whitespace:

\begin{lstlisting}[style=styleCXX]
cout << format("[{}]\n", trimstr(s));
\end{lstlisting}

Output:

\begin{tcblisting}{commandshell={}}
[ten-thumbed input]
\end{tcblisting}
\end{itemize}


\subsubsection{How it works…}

The string class's various find...() member functions return a position as a size\_t value:

\begin{lstlisting}[style=styleCXX]
size_t find_first_not_of( const CharT* s, size_type pos = 0 );
size_t find_last_not_of( const CharT* s, size_type pos = 0 );
\end{lstlisting}

The return value is the zero-based position of the first matching character (not in the s list of characters) or the special value, string::npos, if not found. npos is a static member constant that represents an invalid position.

We test for (first == string::npos) and return an empty string \{\} if there is no match. Otherwise, we use the first and last positions with the s.substr() method to return the string without whitespace.














