
By default, the basic\_istream class reads one word at a time. We can take advantage of this property to use an istream\_iterator to count words.

\subsubsection{How to do it…}

This is a simple recipe to count words using an istream\_iterator:

\begin{itemize}
\item 
We'll start with a simple function to count words using an istream\_iterator object:

\begin{lstlisting}[style=styleCXX]
size_t wordcount(auto& is) {
	using it_t = istream_iterator<string>;
	return distance(it_t{is}, it_t{});
}
\end{lstlisting}

The distance() function takes two iterators and returns the number of steps between them. The using statement creates an alias it\_t for the istream\_iterator class with a string specialization. We then call distance() with an iterator, initialized with the input stream it\_t\{is\}, and another with the default constructor, which gives us an end-of-stream sentinel.

\item 
We call wordcount() from main():

\begin{lstlisting}[style=styleCXX]
int main() {
	const char * fn{ "the-raven.txt" };
	std::ifstream infile{fn, std::ios_base::in};
	size_t wc{ wordcount(infile) };
	cout << format("There are {} words in the
		file.\n", wc);
}
\end{lstlisting}

This calls wordcount() with our fstream object and prints the number of words in the file. When I call it with the text of Edgar Allan Poe's The Raven, we get this output:

\begin{tcblisting}{commandshell={}}
There are 1068 words in the file.
\end{tcblisting}
\end{itemize}

\subsubsection{How it works…}

Because basic\_istream defaults to word-by-word input, the number of steps in a file will be the number of words. The distance() function will measure the number of steps between two iterators, so calling it with the beginning and the sentinel of a compatible object will count the number of words in the file.
