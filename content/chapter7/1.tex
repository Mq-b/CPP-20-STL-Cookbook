String formatting has traditionally been a weak point with the STL. Until recently, we've been left with an imperfect choice between the cumbersome STL iostreams or the archaic legacy printf(). Beginning with C++20 and the format library, STL string formatting has finally grown up. Closely based on Python's str.format() method, the new format library is fast and flexible, providing many of the advantages of both iostreams and printf(), along with good memory management and type safety.

For more about the format library, see the Format text with the new format library recipe in Chapter 1, New C++20 Features.

While we no longer need to use iostreams for string formatting, it is still quite useful for other purposes, including file and stream I/O, and some type conversions.

In this chapter, we will cover these subjects and more in the following recipes:

\begin{itemize}
\item 
Use string\_view as a lightweight string object

\item 
Concatenate strings

\item 
Transform strings

\item 
Format text with C++20's format library

\item 
Trim whitespace from strings

\item 
Read strings from user input

\item 
Count words in a file

\item 
Initialize complex structures from file input

\item 
Customize a string class with char\_traits

\item 
Parse strings with Regular Expressions
\end{itemize}

















