
C++20引入了新的format()函数,该函数以字符串形式返回参数的格式化表示。format()使用python风格的格式化字符串,具有简洁的语法、类型安全,以及出色的性能。

format()函数接受一个格式字符串和一个模板形参包作为参数:

\begin{lstlisting}[style=styleCXX]
template< class... Args >
string format(const string_view fmt, Args&&... args );
\end{lstlisting}

格式化字符串使用大括号\{\}作为格式化参数的占位符:

\begin{lstlisting}[style=styleCXX]
const int a{47};
format("a is {}\n", a);
\end{lstlisting}

输出为:

\begin{tcblisting}{commandshell={}}
a is 47
\end{tcblisting}

还使用大括号作为格式说明符,例如:

\begin{lstlisting}[style=styleCXX]
	format("Hex: {:x} Octal: {:o} Decimal {:d} \n", a, a, a);
\end{lstlisting}

输出为:

\begin{tcblisting}{commandshell={}}
Hex: 2f Octal: 57 Decimal 47
\end{tcblisting}

本示例展示了如何将format()函数,用于一些常见的字符串格式化解决方案。

\begin{tcolorbox}[colback=webgreen!5!white,colframe=webgreen!75!black,title=Note]
本章是使用Windows 10上的Microsoft Visual C++编译器预览版开发的。在撰写本文时,这是唯一完全支持C++ 20 <format>库的编译器。最终的实现可能在某些细节上有所不同。
\end{tcolorbox}

\subsubsection{How to do it…}

使用format()函数来考虑一些常见的格式化解决方案:

\begin{itemize}
\item 
先从一些变量开始格式化:

\begin{lstlisting}[style=styleCXX]
const int inta{ 47 };
const char * human{ "earthlings" };
const string_view alien{ "vulcans" };
const double df_pi{ pi };
\end{lstlisting}

pi常数在<numbers>头文件和std::numbers命名空间中。

\item 
可以使用cout显示变量:

\begin{lstlisting}[style=styleCXX]
cout << "inta is " << inta << '\n'
<< "hello, " << human << '\n'
<< "All " << alien << " are welcome here\n"
<< "π is " << df_pi << '\n';
\end{lstlisting}

得到这样的输出:

\begin{tcblisting}{commandshell={}}
a is 47
hello, earthlings
All vulcans are welcome here
π is 3.14159
\end{tcblisting}

\item 
现在,来看看format()如何对它们进行处理:

\begin{lstlisting}[style=styleCXX]
cout << format("Hello {}\n", human);
\end{lstlisting}

这是format()函数的最简单形式。格式字符串有一个占位符\{\}和一个对应的变量human。输出结果为:

\begin{tcblisting}{commandshell={}}
Hello earthlings
\end{tcblisting}

\item 
format()函数返回一个字符串,我们使用cout<{}<来显示该字符串。

format()库的建议包括一个print()函数,使用与format()相同的参数,这就可以打印格式化的字符串:

\begin{lstlisting}[style=styleCXX]
print("Hello {}\n", cstr);
\end{lstlisting}

但print()没有进入C++20,但有望加入C++23。

我们可以用一个简单的函数,使用vformat()提供相同的功能:

\begin{lstlisting}[style=styleCXX]
template<typename... Args>
constexpr void print(const string_view str_fmt,
Args&&... args) {
	fputs(std::vformat(str_fmt,
	std::make_format_args(args...)).c_str(),
	stdout);
}
\end{lstlisting}

这个简单的单行函数提供了一个有用的print()函数,可以用它来代替cout <{}< format()组合:

\begin{lstlisting}[style=styleCXX]
print("Hello {}\n", human);
\end{lstlisting}

输出为:

\begin{tcblisting}{commandshell={}}
Hello earthlings
\end{tcblisting}

该函数的更完整版本可以在示例文件的include目录中找到。

\item 
格式字符串还提供了位置选项:

\begin{lstlisting}[style=styleCXX]
print("Hello {} we are {}\n", human, alien);
\end{lstlisting}

输出为:

\begin{tcblisting}{commandshell={}}
Hello earthlings we are vulcans
\end{tcblisting}

可以在格式字符串中使用位置选项来改变参数的顺序:

\begin{lstlisting}[style=styleCXX]
print("Hello {1} we are {0}\n", human, alien);
\end{lstlisting}

现在,可以得到这样的输出:

\begin{tcblisting}{commandshell={}}
Hello vulcans we are earthlings
\end{tcblisting}

注意,参数保持不变。只有大括号中的位置值发生了变化。位置索引是从零开始的,就像[]操作符一样。

这个特性对于国际化(或本地化)非常有用,因为不同的语言在句子中使用不同的词性顺序。

\item 
数字有很多格式化选项:

\begin{lstlisting}[style=styleCXX]
print("π is {}\n", df_pi);
\end{lstlisting}

输出为:

\begin{tcblisting}{commandshell={}}
π is 3.141592653589793
\end{tcblisting}

可以指定精度的位数:

\begin{lstlisting}[style=styleCXX]
print("π is {:.5}\n", df_pi);
\end{lstlisting}

输出为:

\begin{tcblisting}{commandshell={}}
π is 3.1416
\end{tcblisting}

冒号字符“:”用于分隔位置索引和格式化参数:

\begin{lstlisting}[style=styleCXX]
print("inta is {1:}, π is {0:.5}\n", df_pi, inta);
\end{lstlisting}

输出为:

\begin{tcblisting}{commandshell={}}
inta is 47, π is 3.1416
\end{tcblisting}

\item 
若想要一个值占用一定的空间,可以这样指定字符的数量:

\begin{lstlisting}[style=styleCXX]
print("inta is [{:10}]\n", inta);
\end{lstlisting}

输出为:

\begin{tcblisting}{commandshell={}}
inta is [    47]
\end{tcblisting}

可以向左或向右对齐:

\begin{lstlisting}[style=styleCXX]
print("inta is [{:<10}]\n", inta);
print("inta is [{:>10}]\n", inta);
\end{lstlisting}

输出为:

\begin{tcblisting}{commandshell={}}
inta is [47     ]
inta is [     47]
\end{tcblisting}

默认情况下,用空格字符填充,但可以进行修改:

\begin{lstlisting}[style=styleCXX]
print("inta is [{:*<10}]\n", inta);
print("inta is [{:0>10}]\n", inta);
\end{lstlisting}

输出为:

\begin{tcblisting}{commandshell={}}
inta is [47********]
inta is [0000000047]
\end{tcblisting}

还可以将值居中:

\begin{lstlisting}[style=styleCXX]
print("inta is [{:^10}]\n", inta);
print("inta is [{:_^10}]\n", inta);
\end{lstlisting}

输出为:

\begin{tcblisting}{commandshell={}}
inta is [    47    ]
inta is [____47____]
\end{tcblisting}

\item 
可以将整数格式化为十六进制、八进制或默认的十进制表示形式:

\begin{lstlisting}[style=styleCXX]
print("{:>8}: [{:04x}]\n", "Hex", inta);
print("{:>8}: [{:4o}]\n", "Octal", inta);
print("{:>8}: [{:4d}]\n", "Decimal", inta);
\end{lstlisting}

输出为:

\begin{tcblisting}{commandshell={}}
    Hex: [002f]
  Octal: [  57]
Decimal: [  47]
\end{tcblisting}

注意,这里使用右对齐来排列标签。

大写十六进制使用大写X:

\begin{lstlisting}[style=styleCXX]
print("{:>8}: [{:04X}]\n", "Hex", inta);
\end{lstlisting}

输出为:

\begin{tcblisting}{commandshell={}}
    Hex: [002f]
\end{tcblisting}

\begin{tcolorbox}[colback=gray!5!white,colframe=gray!75!black,title=Tip]
默认情况下,Windows使用不常见的字符编码。最新版本可能默认为UTF-16或UTF-8 BOM。旧版本可能默认为“代码页”1252,这是ISO 8859-1 ASCII标准的超集。Windows系统默认为更常见的UTF-8 (No BOM)。

默认情况下,Windows不会显示标准UTF-8 π字符。要使Windows兼容UTF-8编码(以及其他编码),请在测试时使用编译器开关/ UTF-8并在命令行上发出命令chcp 65001。现在,你也可以得到π并使用它。
\end{tcolorbox}
	
\end{itemize}

\subsubsection{How it works…}

<format>库使用模板形参包将参数传递给格式化器,需要单独检查参数的类和类型。标准库函数make\_format\_args()接受一个形参包并返回一个format\_args对象,该对象需要提供格式化参数的类型擦除列表。

可以在print()函数中看到这些:

\begin{lstlisting}[style=styleCXX]
template<typename... Args>
constexpr void print(const string_view str_fmt, Args&&... args)
{
	fputs(vformat(str_fmt,
	make_format_args(args...)).c_str(),
	stdout);
}
\end{lstlisting}

make\_format\_args()函数的作用是:接受一个参数包并返回format\_args对象。vformat()函数的作用是:接受格式字符串和format\_args对象,并返回一个std::string。然后,使用c\_str()方法来获取用于fputs()的C字符串。

\subsubsection{There's more…}

对于自定义类重载ostream <{}<操作符是常见的做法。例如,给定一个保存分数值的类Frac:

\begin{lstlisting}[style=styleCXX]
template<typename T>
struct Frac {
	T n;
	T d;
};
...
Frac<long> n{ 3, 5 };
cout << "Frac: " << n << '\n';
\end{lstlisting}

我们想把对象打印成3/5这样的分数。因此,需要编写一个简单的操作符<{}<特化,就像这样:

\begin{lstlisting}[style=styleCXX]
template <typename T>
std::ostream& operator<<(std::ostream& os, const Frac<T>& f) {
	os << f.n << '/' << f.d;
	return os;
}
\end{lstlisting}

现在输出是:

\begin{tcblisting}{commandshell={}}
Frac: 3/5
\end{tcblisting}

为了为我们的自定义类提供format()支持,需要创建一个特化的格式化器对象,如下所示:

\begin{lstlisting}[style=styleCXX]
template <typename T>
struct std::formatter<Frac<T>> : std::formatter<unsigned> {
	template <typename Context>
	auto format(const Frac<T>& f, Context& ctx) const {
		return format_to(ctx.out(), "{}/{}", f.n, f.d);
	}
};
\end{lstlisting}

formatter类的特化重载了它的format()方法。为了简单起见,我们继承了formatter<unsigned>特化。format()方法使用Context对象调用,该对象为格式化的字符串提供输出上下文。再将ctx.out传入format\_to()函数,将返回一个正常格式字符串。

现在,可以在Frac类中使用print()函数:

\begin{lstlisting}[style=styleCXX]
print("Frac: {}\n", n);
\end{lstlisting}

格式化器现在可以识别我们的类,并提供我们想要的输出:

\begin{tcblisting}{commandshell={}}
Frac: 3/5
\end{tcblisting}






