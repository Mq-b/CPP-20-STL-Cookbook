
STL使用std::cin对象从标准输入流提供基于字符的输入。cin对象是一个全局单例对象,可从控制台读取输入作为istream输入流。

默认情况下,cin每次读取一个单词,直到它到达流的末尾:

\begin{lstlisting}[style=styleCXX]
string word{};
cout << "Enter words: ";
while(cin >> word) {
	cout << format("[{}] ", word);
}
cout << '\n';
\end{lstlisting}

输出为:

\begin{tcblisting}{commandshell={}}
$ ./working
Enter words: big light in sky
[big] [light] [in] [sky]
\end{tcblisting}

这可能会让一些人认为cin功能很弱。

虽然cin确实有它的怪癖,但其更容易用来提供面向行的输入。

\subsubsection{How to do it…}

要从cin获得基本的面向行的功能,需要理解两个重要的行为。一种是一次获得一行字的能力,而不是一次获得一个字。另一个是在出现错误条件后重置流的能力。让我们来了解一下细节:

\begin{itemize}
\item 
首先,需要提示用户进行输入。下面是一个简单的提示函数:

\begin{lstlisting}[style=styleCXX]
bool prompt(const string_view s, const string_view s2 = "") {
	if(s2.size()) cout << format("{} ({}): ", s, s2);
	else cout << format("{}: ", s);
	cout.flush();
	return true;
}
\end{lstlisting}

调用count .flush()函数确保立即显示输出。有时,输出不包括换行符时,输出流可能不会自动刷新。

\item 
cin类有一个getline()方法,从输入流中获取一行文本,并将其放入C-string数组中:

\begin{lstlisting}[style=styleCXX]
constexpr size_t MAXLINE{1024 * 10};
char s[MAXLINE]{};
const char * p1{ "Words here" };
prompt(p1);
cin.getline(s, MAXLINE, '\n');
cout << s << '\n';
\end{lstlisting}

输出为:

\begin{tcblisting}{commandshell={}}
Words here: big light in sky
big light in sky
\end{tcblisting}

cin.getline()方法有三个参数:

\begin{lstlisting}[style=styleCXX]
getline(char* s, size_t count, char delim );
\end{lstlisting}

第一个参数是目标的C-string数组。第二个是数组的大小。第三个是行尾的分隔符。

该函数在数组中放置的字符不超过count-1,为空结束符留出空间。

分隔符默认为换行符'\verb|\|n'字符。

\item 
STL还提供了一个独立的getline()函数,用于STL字符串对象:

\begin{lstlisting}[style=styleCXX]
string line{};
const char * p1a{ "More words here" };
prompt(p1a, "p1a");
getline(cin, line, '\n');
cout << line << '\n';
\end{lstlisting}

输出为:

\begin{tcblisting}{commandshell={}}
$ ./working
More words here (p1a): slated to appear in east
slated to appear in east
\end{tcblisting}

std::getline()函数有三个参数:

\begin{lstlisting}[style=styleCXX]
getline(basic_istream&& in, string& str, char delim );
\end{lstlisting}

第一个参数是输出流,第二个参数是对字符串对象的引用,第三个参数是行结束分隔符。

若未指定,分隔符默认为换行符'\verb|\|n'字符。

我感觉,getline()比cin.getline()方法更方便。

\item 
可以使用cin从输入流中获取特定的类型。要做到这一点,必须能够处理错误条件。

当cin遇到错误时,其将流设置为错误条件并停止接受输入。要在错误后重试输入,必须重置流的状态。下面是一个在错误后重置输入流的函数:

\begin{lstlisting}[style=styleCXX]
void clearistream() {
	string s{};
	cin.clear();
	getline(cin, s);
}
\end{lstlisting}

clear()函数的作用是:重置输入流中的错误标志,但将文本保留在缓冲区中。然后,通过读取一行并丢弃,来清除缓冲区。

\item 
可以通过对数值类型变量使用cin来接受数值输入:

\begin{lstlisting}[style=styleCXX]
double a{};
double b{};
const char * p2{ "Please enter two numbers" };
for(prompt(p2); !(cin >> a >> b); prompt(p2)) {
	cout << "not numeric\n";
	clearistream();
}
cout << format("You entered {} and {}\n", a, b);
\end{lstlisting}

输出为:

\begin{tcblisting}{commandshell={}}
$ ./working
Please enter two numbers: a b
not numeric
Please enter two numbers: 47 73
You entered 47 and 73
\end{tcblisting}

cin >{}> a >{}> b表达式接受来自控制台的输入,并尝试将前两个单词转换为与a和b兼容的类型(double)。若失败了,则再次使用clearistream()。

\item 
可以使用getline()分隔符参数来获取逗号分隔的输入:

\begin{lstlisting}[style=styleCXX]
line.clear();
prompt(p3);
while(line.empty()) getline(cin, line);
stringstream ss(line);
while(getline(ss, word, ',')) {
	if(word.empty()) continue;
	cout << format("word: [{}]\n", trimstr(word));
}
\end{lstlisting}

输出为:

\begin{tcblisting}{commandshell={}}
$ ./working
Comma-separated words: this, that, other
word: [this]
word: [that]
word: [other]
\end{tcblisting}

因为这段代码是在数字代码之后运行的,并且因为cin是混乱的,所以缓冲区中可能仍然有结束的行。while(line.empty())循环将有选择地获取空行。

使用一个stringstream对象来处理单词,所以不必使用cin。所以可以使用getline()来获取一行,而无需等待文件结束状态。

然后,在stringstream对象上调用getline()来解析用逗号分隔的单词。这给了我们带有前导空格的单词。我们使用本章中的trimstr()函数来清理空白字符。
\end{itemize}

\subsubsection{How it works…}


std::cin对象比它看起来更有用,但使用它可能是一个挑战。其倾向于在流上留下行结束符,并且在出现错误的情况下,可能会忽略输入。

解决方案是使用getline(),并在必要时将行放入stringstream中,以便于解析。











