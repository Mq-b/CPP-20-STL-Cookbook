
输入流的优点是能够解析文本文件中不同类型的数据,并将它们转换为相应的基本类型。下面是一个使用输入流将数据导入结构容器的简单技术。

\subsubsection{How to do it…}

这个示例中,我们将获取一个数据文件,并将其不同的字段导入到struct对象的vector中。数据文件表示城市,及其人口和地图坐标:

\begin{itemize}
\item 
这是cities.txt,是要读取的数据文件:

\begin{tcblisting}{commandshell={}}
Las Vegas
661903 36.1699 -115.1398
New York City
8850000 40.7128 -74.0060
Berlin
3571000 52.5200 13.4050
Mexico City
21900000 19.4326 -99.1332
Sydney
5312000 -33.8688 151.2093
\end{tcblisting}

城市名称单独在一行上。第二行是人口,然后是经度和纬度。这一模式在五个城市中都重复出现。

\item 
我们将在一个常量中定义文件名,以便稍后可以打开它:

\begin{lstlisting}[style=styleCXX]
constexpr const char * fn{ "cities.txt" };
\end{lstlisting}

\item 
下面是一个保存数据的City结构体:

\begin{lstlisting}[style=styleCXX]
struct City {
	string name;
	unsigned long population;
	double latitude;
	double longitude;
};
\end{lstlisting}

\item 
我们希望读取文件并填充City对象的vector:

\begin{lstlisting}[style=styleCXX]
vector<City> cities;
\end{lstlisting}

\item 
这里是输入流使这变得容易的地方。可以简单地为City类特化操作符>{}>,如下所示:

\begin{lstlisting}[style=styleCXX]
std::istream& operator>>(std::istream& in, City& c) {
	in >> std::ws;
	std::getline(in, c.name);
	in >> c.population >> c.latitude >> c.longitude;
	return in;
}
\end{lstlisting}

std::ws输入操纵符将丢弃输入流中前面的空格。

我们使用getline()来读取城市名称,因为可能由多个单词组成。

这利用填充(unsigned long)的>{}>操作符,以及纬度和经度(都是double)元素来填充正确的类型。

\item 
现在,可以打开文件并使用>{}>操作符将文件直接读入City对象的vector:

\begin{lstlisting}[style=styleCXX]
ifstream infile(fn, std::ios_base::in);
if(!infile.is_open()) {
	cout << format("failed to open file {}\n", fn);
	return 1;
}
for(City c{}; infile >> c;) cities.emplace_back(c);
\end{lstlisting}

\item 
可以使用format()来显示vector:

\begin{lstlisting}[style=styleCXX]
for (const auto& [name, pop, lat, lon] : cities) {
	cout << format("{:.<15} pop {:<10} coords {}, {}\n",
	name, make_commas(pop), lat, lon);
}
\end{lstlisting}

输出为:

\begin{tcblisting}{commandshell={}}
$ ./initialize_container < cities.txt
Las Vegas...... pop 661,903 coords 36.1699, -115.1398
New York City.. pop 8,850,000 coords 40.7128, -74.006
Berlin......... pop 3,571,000 coords 52.52, 13.405
Mexico City.... pop 21,900,000 coords 19.4326, -99.1332
Sydney......... pop 5,312,000 coords -33.8688, 151.2093
\end{tcblisting}

\item 
make\_comma()函数也在第2章中使用,接受一个数值并返回一个字符串对象。为了可读性,添加了逗号:

\begin{lstlisting}[style=styleCXX]
string make_commas(const unsigned long num) {
	string s{ std::to_string(num) };
	for(int l = s.length() - 3; l > 0; l -= 3) {
		s.insert(l, ",");
	}
	return s;
}
\end{lstlisting}
\end{itemize}

\subsubsection{How it works…}

这个示例的核心是istream类操作符>{}>重载:

\begin{lstlisting}[style=styleCXX]
std::istream& operator>>(std::istream& in, City& c) {
	in >> std::ws;
	std::getline(in, c.name);
	in >> c.population >> c.latitude >> c.longitude;
	return in;
}
\end{lstlisting}

通过在函数头中指定我们的City类,每当一个City对象出现在输入流>{}>操作符的右侧时,将使用这个函数:

\begin{lstlisting}[style=styleCXX]
City c{};
infile >> c;
\end{lstlisting}

这允许我们精确地指定输入流如何将数据读入City对象。

\subsubsection{There's more…}

在Windows系统上运行这段代码时,会注意到第一行的第一个单词损坏了。这是因为Windows总是在任何UTF-8文件的开头包含一个字节顺序标记(BOM)。因此,当在Windows上读取文件时,BOM将包含在读取的第一个对象中。BOM是不合时宜的,但在撰写本文时,没有办法阻止Windows使用它。

解决方案是调用一个函数来检查文件的前三个字节的BOM。UTF-8的BOM是EF BB BF。下面是一个搜索并跳过UTF-8 BOM的函数:

\begin{lstlisting}[style=styleCXX]
// skip BOM for UTF-8 on Windows
void skip_bom(auto& fs) {
	const unsigned char boms[]{ 0xef, 0xbb, 0xbf };
	bool have_bom{ true };
	for(const auto& c : boms) {
		if((unsigned char)fs.get() != c) have_bom = false;
	}
	if(!have_bom) fs.seekg(0);
	return;
}
\end{lstlisting}

这将读取文件的前三个字节,并检查是否为UTF-8 BOM签名。若这三个字节中的一个不匹配,会将输入流重置到文件的开头。若文件没有BOM,则不有任何问题。

只需在开始读取文件之前调用这个函数:

\begin{lstlisting}[style=styleCXX]
int main() {
	...
	ifstream infile(fn, std::ios_base::in);
	if(!infile.is_open()) {
		cout << format("failed to open file {}\n", fn);
		return 1;
	}
	skip_bom(infile);
	for(City c{}; infile >> c;) cities.emplace_back(c);
	...
}
\end{lstlisting}

这将确保BOM不包含在文件的第一个字符串中。

\begin{tcolorbox}[colback=webgreen!5!white,colframe=webgreen!75!black,title=Note]
因为cin输入流不可定位,所以skip\_bom()函数将不能在cin流上工作,只适用于可搜索的文本文件。
\end{tcolorbox}











