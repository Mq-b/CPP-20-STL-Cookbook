
string类是一个连续容器,很像vector或数组。它支持contiguous\_iterator和所有相应的算法。

string类是具有char类型的basic\_string的特化,所以容器的元素是char类型的。其他特化也可用,但字符串是最常见的。

因为其本质上是一个连续的char元素容器,所以string可以与transform()算法或任何其他使用contiguous\_iterator的技术一起使用。

\subsubsection{How to do it…}

根据应用程序的不同,有几种方法可以进行转换。本示例将探索其中的一些。

\begin{itemize}
\item 
我们将从几个谓词函数开始,谓词函数接受一个转换元素并返回一个相关元素。例如,这是一个返回大写字符的简单谓词:

\begin{lstlisting}[style=styleCXX]
char char_upper(const char& c) {
	return static_cast<char>(std::toupper(c));
}
\end{lstlisting}

这个函数是对std::toupper()的包装。由于toupper()函数返回int类型,而字符串元素是char类型,因此不能在转换中直接使用toupper()函数。

下面是一个相应的char\_lower()函数:

\begin{lstlisting}[style=styleCXX]
char char_lower(const char& c) {
	return static_cast<char>(std::tolower(c));
}
\end{lstlisting}

\item 
rot13()函数是一个有趣的转换谓词,用于演示目的。这是一个简单的替换密码,不适合加密,但通常用于做模糊处理:

\begin{lstlisting}[style=styleCXX]
char rot13(const char& x) {
	auto rot13a = [](char x, char a)->char {
		return a + (x - a + 13) % 26;
	};
	if (x >= 'A' && x <= 'Z') return rot13a(x, 'A');
	if (x >= 'a' && x <= 'z') return rot13a(x, 'a');
	return x;
}
\end{lstlisting}

\item 
可以在transform()算法中使用这些谓词:

\begin{lstlisting}[style=styleCXX]
main() {
	string s{ "hello jimi\n" };
	cout << s;
	std::transform(s.begin(), s.end(), s.begin(),
		char_upper);
	cout << s;
	...
\end{lstlisting}

transform()函数对s中的每个元素调用char\_upper(),将结果放回s并将所有字符转换为大写:

\begin{tcblisting}{commandshell={}}
hello jimi
HELLO JIMI
\end{tcblisting}

\item 
除了transform(),还可以使用一个简单的带有谓词函数的for循环:

\begin{lstlisting}[style=styleCXX]
for(auto& c : s) c = rot13(c);
cout << s;
\end{lstlisting}

结果是:

\begin{tcblisting}{commandshell={}}
URYYB WVZV
\end{tcblisting}

\item 
rot13密码的有趣之处在于它可以快速破解。因为ASCII字母表中有26个字母,旋转13然后再旋转13会得到原始字符串。让我们再次转换为小写字母和rot13来恢复我们的字符串:

\begin{lstlisting}[style=styleCXX]
for(auto& c : s) c = rot13(char_lower(c));
cout << s;
\end{lstlisting}

结果是:

\begin{tcblisting}{commandshell={}}
hello jimi
\end{tcblisting}

由于其统一接口,谓词函数可以作为彼此的参数链接起来。也可以使用char\_lower(rot13(c))得到同样的结果。

\item 
若需求过于复杂,无法进行简单的逐字符转换,则可以像处理任何连续容器一样使用字符串迭代器。下面是一个简单的函数,通过大写第一个字符和空格后面的每个字符将小写字符串转换为Title Case:

\begin{lstlisting}[style=styleCXX]
string& title_case(string& s) {
	auto begin = s.begin();
	auto end = s.end();
	*begin++ = char_upper(*begin); // first element
	bool space_flag{ false };
	for(auto it{ begin }; it != end; ++it) {
		if(*it == ' ') {
			space_flag = true;
		} else {
			if(space_flag) *it = char_upper(*it);
			space_flag = false;
		}
	}
	return s;
}
\end{lstlisting}

因为它返回一个对转换后的字符串的引用,可以用cout调用,像这样:

\begin{lstlisting}[style=styleCXX]
cout << title_case(s);
\end{lstlisting}

输出为:

\begin{tcblisting}{commandshell={}}
Hello Jimi
\end{tcblisting}

\end{itemize}

\subsubsection{How it works…}

std::basic\_string类及其特化(包括string)由完全兼容contiguous\_iterator的迭代器支持,所以适用于任何连续容器的技术,也适用于字符串。

\begin{tcolorbox}[colback=webgreen!5!white,colframe=webgreen!75!black,title=Note]
这些转换将不适用于string\_view对象,因为底层数据是const限定的。
\end{tcolorbox}






















